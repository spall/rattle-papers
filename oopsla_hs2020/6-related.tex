\section{Related work}
\label{sec:related}

The vast majority of existing build systems are \emph{backward build systems} -- they start at the final target, and recursively determine the dependencies required for that target. In contrast, \Rattle is a \emph{forward build system}---the script executes sequentially in the order given by the user.

\subsection{Comparison to forward build systems}

The idea of treating a script as a build system, omitting commands that have not changed, was pioneered by \Memoize \cite{memoize} and popularised by \Fabricate \cite{fabricate}. In both cases the build script was written in Python, where cheap logic was specified in Python and commands were run in a traced environment. If a traced command hadn't changed inputs  since it was last run, then it was skipped. We focus on \Fabricate, which came later and offered more features. \Fabricate uses \texttt{strace} on Linux, and on Windows uses either file access times (which are either disabled or very low resolution on modern Windows installations) or a proprietary and unavailable \texttt{tracker} program. Parallelism can be annotated explicitly, but often is omitted.

These systems did not seem to have much adoption -- in both cases the original sources are no longer available, and knowledge of them survives only as GitHub copies. \Rattle differs from an engineering perspective by the tracing mechanisms available (see \S\ref{sec:tracing}) and the availability of cloud build (see \S\ref{sec:cloud_builds}) -- both of which are likely just a consequence of being developed a decade later. \Rattle extends these systems in a more fundamental way with the notion of hazards, which both allows the detection of bad build scripts, and allows for speculation -- overcoming the main disadvantage of earlier forward build systems. Stated alternatively, \Rattle takes the delightfully simple approach of these build systems, and tries a more sophisticated execution strategy.

Recently there have been three other implementations of forward build systems we are aware of.

\begin{enumerate}
\item \Shake \cite{shake} provides a forward mode implemented in terms of a backwards build system. The approach is similar to the \Fabricate design, offering skipping of repeated commands and explicit parallelism. In addition, \Shake allows caching custom functions as though they were commands, relying on the explicit dependency tracking functions such as \texttt{need} already built into \Shake. The forward mode has been adopted by a few projects, notably a library for generating static websites/blogs. The tracing features are provided by a combination of \Shake and \Fsatrace, and are the ones we reuse in \Rattle.
\item \Fac \cite{fac} is based on the \Bigbro tracing library. Commands are given in a custom file format as a static list of commands (i.e. no monadic expressive power as per \S\ref{sec:monadic}), but may optionally include a subset of their inputs or outputs. The commands are not given in any order, but the specified inputs/outputs are used to form a dependency order which \Fac uses. If the specified inputs/outputs are insufficient to give a working order, then \Fac will fail but record the \emph{actual} dependencies which will be used next time -- a build with no dependencies can usually be made to work by running \Fac multiple times.
\item \Stroll \cite{stroll} takes a static set of commands, without either a valid sequence or any input/output information, and keeps running commands until they have all succeeded. As a consequence, \Stroll may run the same command multiple times, using tracing to figure out what might have changed to turn a previous failure into a future success. \Stroll also reuses the tracing features of \Shake and \Fsatrace.
\end{enumerate}

There are significantly fewer forward build systems than backwards build systems, but the interesting dimension starting to emerge is how an ordering is specified. The three current alternatives are the user specifies a valid order (\Fabricate and \Rattle), the user specifies partial dependencies which are used to calculate an order (\Fac) or the user specifies no ordering and search is used (\Stroll and some aspects of \Fac).

Other build systems, such as \Pluto~\cite{erdweg2015sound} (and
\textsc{PIE}~\cite{Konat_2018,10.1145/3238147.3238196}, a front-end to
\Pluto), simulate some of the convenience of a forward build system by
constructing the dependency graph during the execution of a
host-language program. However, these systems, similar to \Shake, then
rely on that dependency graph for rebuilding. In contrast, \Rattle and
other forward build systems do not construct or use such a dependency
graph.

\subsection{Comparison to backward build systems}
\label{sec:remote_execution}

The design space of backward build systems is discussed in \cite{build_systems_a_la_carte}. In that paper it is notable that forward build systems do not naturally fit into the design space, lacking the features that a build system requires. We feel that omission points to an interesting gap in the landscape of build systems. We think that it is likely forward build systems could be characterised similarly, but that we have yet to develop the necessary variety of forward build systems to do so. There are two dimensions used to classify backward build systems:

\textbf{Ordering} Looking at the features of \Rattle, the ordering is a sequential list, representing an excessively strict ordering given by the user. The use of speculation is an attempt to weaken that ordering into one that is less linear and more precise.

\textbf{Rebuild} For rebuilding, \Rattle looks a lot like the constructive trace model -- traces are made, stored in a cloud, and available for future use. The one wrinkle is that a trace may be later invalidated if it turns out a hazard occurred (see \S\ref{sec:choices}).
% In particular, the correspondence to constructive traces illuminates the consequences of moving to a deep constructive trace model (see \S\ref{sec:forward_hashes}) -- it solves non-determinism at the cost of losing unchanging builds.

\postparagraphs

\noindent There are three features present in some backward build systems that are particularly relevant to \Rattle:

\textbf{Sandboxing} Some backward build systems (e.g. \Bazel \cite{bazel}) run processes in a sandbox, where access to files which weren't declared as dependencies are blocked -- ensuring dependencies are always sufficient. A consequence is that it can be harder to write a \Bazel build script, requiring users to declare dependencies like \texttt{gcc} and system headers that are often overlooked. The sandbox doesn't prevent the reverse problem of too many dependencies.

The \Nix build system \cite{nix} solves the problem of sandboxing in a slightly different way. \Nix is a build tool aimed at package management, replacing tools such as \texttt{apt} and RPM. Before running any binary \Nix rewrites the \texttt{\$PATH} to remove everything apart from declared dependencies, ensuring no undeclared dependencies can be accessed. \Nix also installs each package into a location which includes a cryptographic hash of its contents, ensuring that even packages with the same name end up at different locations unless they are identical.

\textbf{Remote Execution} Some build systems (e.g. \Bazel and \BuildXL \cite{buildxl}) allow running commands on a remote machine, usually with a much higher degree of parallelism than is available on the users machine. If \Rattle was able to leverage remote execution then speculative commands could be used to fill up the cloud cache, and \emph{not} cause local writes to disk, eliminating all speculative hazards -- a very attractive property. Remote execution in \Bazel sends all required files along with the command, but since \Rattle doesn't know the files accessed in advance, that model is infeasible. Remote execution in \BuildXL sends the files it thinks the command will need, augments the file system to block if other files are accessed, and sends requests for additional files back to the originating machine -- which would fit nicely with \Rattle.

\textbf{Hazards} Some build systems detect when certain types of hazard occur. For example, \Pluto \cite{erdweg2015sound} builds are constructed from builders and it is a requirement that no two builders generate the same file, and if they do the build is aborted. Similarly, if two commands in a \Rattle build write to the same file a write-write hazard occurs and the build is terminated. If a command in a \Rattle build writes to a file after another command has already read it a read-write hazard occurs. Analogously in \Pluto, if a builder requires a file generated by another builder, then the builder which generated the file must be required first and if it is not the build is aborted. Building on \Pluto, \textsc{PIE}~\cite{10.1145/3238147.3238196, Konat_2018} optimizes computing the dependency tree. \Rattle does not use the same dependency tree structure of \Pluto, and we have not found computing internal datastructures to be a bottleneck and have not attempted to avoid it.

% This related work isn't relevant, and doesn't say anything interesting. It feels like a forced way of including Pluto.
%
% \textbf{Fine-grained dependencies} Pluto \cite{erdweg2015sound} is another backward build system that aims to
% provide fine-grained dependencies and optimal incremental rebuilding.  It supports dynamic
% dependencies, allows users to specify how they would like to depend on a file, and track build
% rule definitions so a build will re-run correctly when changed.  Rattle in contrast doesn't need
% to explicitly track whether a build script changes, and because all dependencies are implicitly
% tracked \Rattle does not provide the ability for a build script author to specify how a command
% should depend on any file.

\subsection{Build correctness}

We aren't the first to observe that existing build systems often have incorrect dependencies.  \citet{bezemer2017empirical} performed an analysis of the missing dependencies in \Make build scripts, finding over \emph{1.2 million unspecified dependencies} among four projects. To detect missing dependencies, \citet{detecting_incorrect_build_rules} introduced a concept called \emph{build fuzzing}, finding race-conditions and errors in 30 projects. It has also been shown that build maintenance requires as much as a 27\% overhead on software development \cite{build_maintenance}, a substantial proportion of which is devoted to dependency management. Our anecdotes from \S\ref{sec:evaluation} all reinforce these messages.

% SS TODO maybe mention something about a project like node trying to generate dependencies within make;
\subsection{Speculation}

Speculation is used extensively in many other areas of computer science, from processors to distributed systems. If an action is taken before it is known to be required, that can increase performance, but undesired side-effects can occur. Most speculating systems attempt to block the side-effects from happening, or roll them back if they do.

The most common use of speculation in computer science is the CPU -- \citet{swanson_cpu_speculation} found that 93\% of useful CPU instructions were evaluated speculatively. CPUs use hazards to detect incorrect speculation, with similar types of read/write, write/write and write/read hazards \cite{patterson_cpu_design} -- our terminology is inspired by their approaches. For CPUs many solutions to hazards are available, e.g. stalling the pipeline if a hazard is detected in advance or the Tomasulo algorithm \cite{tomasulo}. \Rattle also stalls the pipeline (stops speculating) if it detects potential hazards, although does so with incomplete information, unlike a CPU. The Tomasulo algorithm involves writing results into temporary locations and the moving them over afterwards -- use of remote execution (\S\ref{sec:remote_execution}) might be a way to apply similar ideas to build systems.

Looking towards software systems, \citet{welc2005safe} showed how to add speculation to Java programs, by marking certain parts of the program as worth speculating with a future. Similar to our work, they wanted Java with speculation to respect the semantics of the sequential version of the program, which required two main techniques. First, all data accesses to shared state are tracked and recorded, and if a dependency violation occurs, the offending code is restarted.  Second, shared state is versioned using a copy-on-write invariant to ensure threads write to their own copies, preventing a future from seeing its continuation's writes.

Thread level speculation \cite{steffan1998potential} is used by compilers to automatically parallelise programs, often by executing multiple iterations of a loop body simultaneously. As before, the goal is to maintain the semantics of single-threaded execution. Techniques commonly involve buffering speculative writes \cite{steffan2000scalable} and ensuring that a read reflects the speculative writes of threads that logically precede it.

Speculation has also been investigated for distributed systems. \citet{nightingale2005speculative} showed that adding speculation to distributed file systems such as NFS can make some benchmarks over 10 times faster, by allowing multiple file system operations to occur concurrently. A model allowing more distributed speculation, even in the presence of message passing between speculated distributed processes, is presented by \citet{tapus2006distributed}. Both these pieces of work involve modifying the Linux kernel with a custom file system to implement roll backs transparently.

All these approaches rely on the ability to trap writes, either placing them in a buffer and applying them later or rolling them back. Unfortunately, such facilities, while desirable, are currently difficult to achieve in portable cross-platform abstractions (\S\ref{sec:tracing}). We have used the ideas underlying speculative execution, but if the necessary trapping/rollback facilities became available in future, it's possible we could follow the approaches more directly.
