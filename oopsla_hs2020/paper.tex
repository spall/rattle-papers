% DEADLINE: Mon 14 Sep 2020
% LIMITS: 27 pages, bibliography does not count
% APPROACH: Double blind
% CFP: https://2020.splashcon.org/track/splash-2020-oopsla#Call-for-Papers

\documentclass[acmsmall,screen]{acmart}
\settopmatter{printacmref=false}
\renewcommand\footnotetextcopyrightpermission[1]{} % Remove after submission

\bibliographystyle{ACM-Reference-Format}
\citestyle{acmauthoryear}

\usepackage{microtype}
\usepackage{xspace}
\usepackage{pgfplots}
\pgfplotsset{compat=1.13}
\pgfplotsset{title style={at={(0.5,0.7)}}}

%%% If you see 'ACMUNKNOWN' in the 'setcopyright' statement below,
%%% please first submit your publishing-rights agreement with ACM (follow link on submission page).
%%% Then please update our instructions page and copy-and-paste the NEW commands into your article.
%%% Please contact us in case of questions; allow up to 10 min for the system to propagate the information.
%%%
%%% The following is specific to OOPSLA '20 and the paper
%%% 'Build Scripts with Perfect Dependencies'
%%% by Sarah Spall, Neil Mitchell, and Sam Tobin-Hochstadt.
%%%
% \setcopyright{ACMUNKNOWN}
\setcopyright{rightsretained}
\acmPrice{}
\acmDOI{10.1145/3428237}
\acmYear{2020}
\copyrightyear{2020}
\acmSubmissionID{oopsla20main-p169-p}
\acmJournal{PACMPL}
\acmVolume{4}
\acmNumber{OOPSLA}
\acmArticle{169}
\acmMonth{11}

\begin{document}

\newcommand{\Make}{\textsc{Make}\xspace}
\newcommand{\Rattle}{\textsc{Rattle}\xspace}
\newcommand{\Fabricate}{\textsc{Fabricate}\xspace}
\newcommand{\Bazel}{\textsc{Bazel}\xspace}
\newcommand{\Buck}{\textsc{Buck}\xspace}
\newcommand{\Shake}{\textsc{Shake}\xspace}
\newcommand{\Bigbro}{\textsc{BigBro}\xspace}
\newcommand{\Fac}{\textsc{Fac}\xspace}
\newcommand{\Fsatrace}{\textsc{Fsatrace}\xspace}
\newcommand{\tracedfs}{\textsc{Traced-Fs}\xspace}
\newcommand{\BuildXL}{\textsc{BuildXL}\xspace}
\newcommand{\Nix}{\textsc{Nix}\xspace}
\newcommand{\Memoize}{\textsc{Memoize}\xspace}
\newcommand{\Stroll}{\textsc{Stroll}\xspace}
\newcommand{\Pluto}{\textsc{Pluto}\xspace}
\newcommand{\PIE}{\textsc{PIE}\xspace}

\newcommand{\postparagraphs}{\vspace{3mm}\noindent}


\title{Build Scripts with Perfect Dependencies}

\author{Sarah Spall}
\affiliation{
  \institution{Indiana University}
  \country{USA}
}
\email{sjspall@iu.edu}

\author{Neil Mitchell}
\affiliation{
  \institution{Facebook}
  \country{UK}
}
\email{ndmitchell@gmail.com}

\author{Sam Tobin-Hochstadt}
\affiliation{
  \institution{Indiana University}
  \country{USA}
}
\email{samth@cs.indiana.edu}


\begin{abstract}
Build scripts for most build systems describe the actions to run, and the dependencies between those actions---but often build scripts get those dependencies wrong.
Most build scripts have both \emph{too few dependencies} (leading to incorrect build outputs) and \emph{too many dependencies} (leading to excessive rebuilds and reduced parallelism). Any programmer who has wondered why a small change led to excess compilation, or who resorted to a ``clean'' step, has suffered the ill effects of incorrect dependency specification.
We outline a build system where dependencies are \emph{not specified}, but instead \emph{captured by tracing execution}.
The consequence is that dependencies are always correct by construction and build scripts are easier to write.
The simplest implementation of our approach would lose parallelism, but we are able to recover parallelism using speculation.
\end{abstract}

%% 2012 ACM Computing Classification System (CSS) concepts
%% Generate at 'http://dl.acm.org/ccs/ccs.cfm'.
\begin{CCSXML}
<ccs2012>
<concept>
<concept_id>10011007.10011006.10011073</concept_id>
<concept_desc>Software and its engineering~Software maintenance tools</concept_desc>
<concept_significance>500</concept_significance>
</concept>
</ccs2012>
\end{CCSXML}
\ccsdesc[500]{Software and its engineering~Software maintenance tools}
%% End of generated code
\keywords{build systems, functional programming}

\maketitle

\section{Introduction}
\label{sec:introduction}


Every non-trivial piece of software includes a ``build system'', describing how to set up the system from source code.
Build scripts \cite{build_systems_a_la_carte} describe \emph{commands to run} and \emph{dependencies to respect}. For example, using the \Make build system \cite{make}, a build script might look like:

\vspace{3mm}
\begin{verbatim}
main.o: main.c
    gcc -c main.c
util.o: util.c
    gcc -c util.c
main.exe: main.o util.o
    gcc -o main.exe main.o util.o
\end{verbatim}
\vspace{3mm}

This script contains three rules. Looking at the first rule, it says \texttt{main.o} depends on \texttt{main.c}, and is produced by running \texttt{gcc -c main.c}. What if we copied the commands into a shell script? We get:

\vspace{3mm}
\begin{verbatim}
gcc -c main.c
gcc -c util.c
gcc -o main.exe main.o util.o
\end{verbatim}
\vspace{3mm}

That's shorter, simpler and easier to follow. Instead of declaring the outputs and dependencies of each command, we've merely given one valid ordering of the commands (we could equally have put \texttt{gcc -c util.c} first). This simpler specification has additional benefits. First, we've fixed an inadvertent bug -- these commands depend on the undeclared dependency \texttt{gcc}, and potentially whatever header files are used by \texttt{main.c} and \texttt{util.c}. Furthermore, as the files \texttt{main.c} and \texttt{util.c} evolve, and their dependencies change (by changing the \texttt{\#include} directives), the shell script remains correct, while the \Make script \emph{must} be kept consistent or builds will become incorrect.

\subsection{Why dependencies?}

Given the manifest simplicity of the shell script, why write a \texttt{Makefile}? Build systems such as \Make have two primary advantages, both provided by dependency specification: \textbf{incrementality} and \textbf{parallelism}. \Make is able to re-run only the commands needed when only some of the files change, saving significant work when only \texttt{util.c} changes. \Make can also run multiple commands in parallel when neither depends on the other, such as the two invocations of \texttt{gcc -c}. With a shell script, these are both challenging research problems in incremental computing and automatic parallelization, respectively, and unlikely to be solvable for arbitrary programs such as \texttt{gcc}. 

\subsection{Builds without dependencies}

In this paper we show how to take the above shell script and gain the benefits of a \Make build (\S\ref{sec:design}). Firstly, we can skip those commands whose dependencies haven't changed by \emph{tracing} which files they read and write (\S\ref{sec:skipping_unnecessary}) and keeping a history of such traces. Secondly, we can run some commands in parallel, using \emph{speculation} to guess which future commands won't interfere with things already running (\S\ref{sec:speculation}). The key to speculation is a robust model of what ``interfering'' means -- we call a problematic interference a \emph{hazard}, which we define  in \S\ref{sec:hazards}; we formalize hazards and show important properties about the safety of speculation in \S\ref{sec:proof}.

We have implemented these techniques in a build system called \Rattle\footnote{\url{https://github.com/author_name_omitted/rattle}}, introduced in \S\ref{sec:implementation}, which embeds commands in a Haskell script. A key part of the implementation is the ability to trace commands, using techniques we describe in \S\ref{sec:tracing}. To evaluate our claims, and properly understand the subtleties of our design, we converted existing \Make scripts into \Rattle scripts, and discuss the performance characteristics and \Make script issues uncovered in \S\ref{sec:evaluation}. We also implement two small but non-trivial builds from scratch in \Rattle and report on the lessons learned. Our design can be considered a successor to the \Memoize build system \cite{memoize}, and we compare \Rattle with it and other related work in \S\ref{sec:related}. Finally, in \S\ref{sec:conclusion} we conclude and describe future work.

\section{Build Scripts from Commands}
\label{sec:design}

Our goal is to design a build system where a build script is simply a list of commands. In this section we develop our design, starting with the simplest system that just executes all the commands in order, and ending up with the benefits of a conventional build system.

\subsection{Executing commands}
\label{sec:executing_commands}

Given a build script as a list of commands, like in \S\ref{sec:introduction}, the simplest execution model is to run each command sequentially in the order they were given. Importantly, we require that list of commands to be ordered, such that any dependencies are produced before they are used. We consider this sequential execution the reference semantics, and as we develop our design further, require that any optimised/cached implementation gives the same results.

\subsection{Value-dependent commands}
\label{sec:monadic}

While a static list of commands is sufficient for simple builds, it is limited in its expressive power. Taking the build system from \S\ref{sec:introduction}, it would be better to compile and link \emph{all} \texttt{.c} files -- not just those explicitly listed by the script. A more realistic script might be\footnote{We take some liberties with shell scripts around how we replace the extensions, so as not to obscure the main focus.}:

\vspace{3mm}
\begin{verbatim}
FILES=$(ls *.c)
for FILE in $FILES; do
    gcc -c $FILE
done
gcc -o main.exe ${FILES%.c}.o
\end{verbatim}
\vspace{3mm}

\begin{figure}
\begin{verbatim}
import Development.Rattle
import System.FilePath
import System.FilePattern

main = runRattle $ do
    let toO x = takeBaseName x <.> "o"
    cs <- liftIO $ getDirectoryFiles "." [root </> "*.c"]
    forM_ cs $ \c ->
        cmd "gcc -c" c
    cmd "gcc -o main.exe" (map toO cs)
\end{verbatim}
\caption{A Haskell/\Rattle version of the script from \S\ref{sec:monadic}}
\label{fig:monadic}
\end{figure}

This script now has a curious mixture of commands (\texttt{ls}, \texttt{gcc}), control logic (\texttt{for}) and simple manipulation (changing file extension). Importantly, there is \emph{no fixed list of commands} -- the future commands are determined based on the results of previous commands. Concretely, in this example, the result of \texttt{ls} changes which \texttt{gcc} commands are executed. The transition from a fixed list of commands to a dynamic list matches the \texttt{Applicative} vs \texttt{Monadic} destinction of \citet{build_systems_a_la_carte} \S3.5.

To accomodate a dynamic list of commands, we adjust our model so that a build script is a series of commands, where future commands may depend on the results of previous commands in ways that are not visible to the build system. The commands are produced with ``cheap'' functions such as control logic and simple manipulations. We consider the cheap commands to be fixed overhead, run on every build, and not cached or parallelised in any way. If any of these cheap manipulations becomes expensive, they can be replaced by a command, which will then be handled properly. The simple list of commands from \S\ref{sec:executing_commands} is then a degenerate case of no interesting logic.

An important consequence of the control logic not being visible to the build system is that the build system has no prior knowledge of which commands are coming next, or if they have changed. As a result, even if you are using a simple static script such as from \S\ref{sec:introduction}, and then manually edit the script, the build will execute correctly. The build system is unaware if you edited the script, or if the commands were conditional on something that it cannot observe. Therefore, this model solves the problem of self-tracking from \citet{build_systems_a_la_carte} \S6.5.

\subsection{Dependency tracing}
\label{sec:assume_tracing}

For the rest of this section we assume the existence of \emph{dependency tracing} which can, after a command completes, tell us which files that command read and which files it wrote -- we discuss the implementation and limitations of dependency tracing in \S\ref{sec:tracing}. Concretely, we can run a command in a special mode such that when the command completes (and not before) we can determine which files it read and which it wrote. We cannot determine at which point during the execution these files were accessed, or which order they were accessed in. We cannot prevent or otherwise redirect an in-progress access. These limitations are a (frustrating!) consequence of the tracing technology used.

\subsection{Skipping unnecessary commands}
\label{sec:skipping_unnecessary}

When \Rattle runs a command, it uses tracing to capture the files read and written, and then records their hashes after the command finishes. If \Rattle subsequently runs the same command, and the inputs and outputs haven't changed (same hashes), it can be skipped. This approach is the fundamental aspect of \Fabricate\citep{fabricate}. However, this approach has some issues that can result in incorrect builds.  The following are assumptions \Rattle makes about a build to avoid the aforementioned issues.

% SS todo make specific issues more clear

\begin{itemize}
\item \Rattle assumes that each command is atomic - it cannot be subdivided into smaller parts. If a command is secretly two independent commands then they should usually be expressed as such so they can be individually skipped.

\item If a command is deterministic, then running it again will change nothing. However, many commands are not deterministic -- e.g. the output of \texttt{ghc} object files contains random values within it. \Rattle assumes that where such non-determinism exists any possible output is equally valid.

\item If a command incorporates information such as the timestamp, then a cached value will represent the first time the command was run, not the current time. For commands like compilation that embed the timestamp in an object header, the original timestamp is probably fine.  If a command wants to actually get the current time, we suggest it be made part of the control logic (as per \S\ref{sec:monadic}) so it will run everytime the build runs. The same applies to commands that require unique information, e.g. the generation of a GUID or random number.

\item If a command both reads and writes the same file, and the information written is fundamentally influenced by the file that was read, then the command never results in a stable state. As an example, \verb"echo x >> foo.txt" will append the character \texttt{x} every time the command is run. As another example, the GHC package database has additional entries added every time a package is installed, making the output a consequence of the original file\footnote{As a consequence many build systems, including \Bazel and \Rattle, use multiple package databases with only one entry per database}. Equally, there are also commands that read the existing file to avoid rewriting a file that hasn't changed (e.g. \texttt{ghc} generating a \texttt{.hi} file) and commands that can cheaply update an existing file in some circumstances (the Micrsoft C++ linker). \Rattle makes the assumption that if a command both reads and writes to a file that the read does not meaningfully influence the write. Thought of another way, \Rattle can consider the command to be non-deterministic and the previous points apply.

\item If a command reads a file that during a build is modified by something not under the observation of \Rattle (e.g. a human or the untracked control logic), then the command may have seen multiple distinct values of the file, and the final hash will not match what the command saw. Therefore, \Rattle makes the assumption that only \Rattle tracked commands are working on the relevant files during the build. \Rattle can detect such errors by checking the modification time of the read file pre-dates the start of the command execution.
\end{itemize}

In general we assume individual commands are well behaved and meet the above assumptions, and if not, they can be lifted into the control logic.

\subsection{Cloud builds}
\label{sec:cloud_builds}

When skipping an execution \Rattle checks that all files accessed have the same hashes as a previous execution. However, if the files read all match a previous execution, and the corresponding output files have been cached, those output files can be copied over \emph{without} rerunning the command. If that cache is in on a server, you end up with cloud build functionality. Assuming commands that can be cached, whether a command uses a cloud oracle or runs locally is not observable to the rest of the system, so we don't need to consider it for the rest of the system. While this approach works beautifully in theory, in practice it leads to at least three separate problems:

\paragraph{Relative build directories} Often the current directory, or users profile directory, will be accessed by commands. These change either if the user has two working directories, or if they use different machines. We solve this by having a substitution table, replacing values such as the users home directory with \texttt{\$HOME}.

\paragraph{Compound commands} Sometimes a command will produce something that is user specific (not great for caching), but the next step will remove the user specificity (good for caching). To fix that we allow compound commands, by conjoining two commands with \texttt{\&\&}. Sometimes the sole purpose of the second command can be to strip machine-unique data from the first command.

\paragraph{Non-deterministic builds} If a command has non-deterministic output then if a user is temporarily disconnected from the cloud storage, and try and build something, then it will result in a different hash. If the user then reconnects to the cloud all actions depending on that non-deterministic build will fail to get a cache hit. There is one solution in \Rattle, where the hash of an output can be considered to be the hash of its inputs, but this setting is enabled on a per-file basis, and has all the problems of deep-constructive traces from \cite{build_systems_a_la_carte} \S?? and cannot benefit from unchanging outputs.

As a technical note, currently \Rattle uses a shared drive for sharing build artefacts, using tools such as NFS or Samba to provide the internet style ``cloud'' part.

\subsection{Build consistency}
\label{sec:hazards}

For a \Make build system to be stable, it must be the case that after a build, a rebuild will not execute any further commands. It's easy to construct examples of sequences of commands that violate this property, for example:

\begin{verbatim}
echo x >> foo.txt
\end{verbatim}

% SS not very clear in skipping unnecessary that those things listed are ``Restrictions''
% NM tried to make it more obvious. It's actually an example in restrictions

% SS changed resctrictions -> assumption since afaik rattle won't produce an error if someone puts that in their build.

As discussed in \S\ref{sec:skipping_unnecessary}, this command does not meet our assumption that a command which both reads and writes a file should not be meaningfully influenced by the initial read. But we can find other sequences of commands, where each command is fine in isolation, but the conjunction is problematic, for example:

\begin{verbatim}
echo 1 > foo.txt
echo 2 > foo.txt
\end{verbatim}

This program writes \texttt{1} to \texttt{foo.txt}, then writes \texttt{2}. If the commands are re-executed then the first command is invalid by virtue of its output changing, and after the first command has re-run, now the second command is invalid. More generally, if a build writes different values to the same file multiple times, it will not be consistent. But even without writing to the same file twice, it is possible to have an inconsistent build:

\begin{verbatim}
sha1sum foo.txt > bar.txt
sha1sum bar.txt > foo.txt
\end{verbatim}

Here \texttt{sha1sum} takes the SHA1 hash of a file, firstly taking the SHA of \texttt{foo.txt} and storing it in \texttt{bar.txt}, then taking the SHA of \texttt{bar.txt} and storing it in \texttt{foo.txt}. The root of the problem is that the script first reads from \texttt{foo.txt} on the first line, then writes to \texttt{foo.txt} on the second line, meaning that when the script is rerun the read of foo.txt will have to be repeated as its value has changed.

It turns out writing to a file after it has already been either read or written is the only circumstance in which a build, where every individual command is well-formed (as per \S\ref{sec:skipping_unnecessary}), is inconsistent. We define such a build as \emph{hazardous} if it violates one of the following consistency rules:

\begin{description}
\item[write after read] If one command reads from a file, and a subsequent command writes to that file, on a future build, the first command will have to be rerun because its input has changed.  This behavior causes a \emph{read-write hazard}.
\item[write after write] If two commands both write to the same file, on a future build, the first will be rerun (it's output has changed), which is likely to then cause the second to be rerun.  This behavior causes a \emph{write-write hazard}.
\end{description}

Using tracing \Rattle spots hazards and raises errors if they occur. We prove that a build system with deterministic control logic and with no hazards always results in no rebuilds in \S\ref{sec:proof:no_rebuild}. In a build system without hazards there is at most one write to any file, which occurs before any reads of that file. We can therefore prove there are no rebuilds by showing the first command can't rebuild, and proceeding by induction. It is not the case that the presence of hazards guarantees subsequent executions will definitely rebuild, for example if the write doesn't actually change the value, but such a build system is dangerous -- if the write definitely can't change the output, the write should not be present.

\subsection{Parallelism}

Given a script with no hazards when executed sequentially, we can show that any interleaving of those commands that also has no hazards will result in an equivalent output in \S\ref{sec:proof:reorder}. Moreover, any parallel or interleaved execution without hazards will also be equivalent, see \S\ref{sec:proof:parallel}. And even further, if \Rattle executes any additional commands that don't cause hazards, they can be shown not to change the results the normal build produced, see \S\ref{sec:proof:additional}.

As a consequence of the above, we have quite a lot of freedom to schedule the commands, provided they do not cause hazards.  There are two ways \Rattle enables parallelism.

\subsubsection{Explicit Parallelism}

In the Haskell API for \Rattle there is a parallel combinator \texttt{forP}. Replacing \texttt{forM} with \texttt{forP} in Figure \ref{fig:monadic} causes the commands to be given to \Rattle in parallel. As a consequence, they can be executed in parallel. The use of explicit parallelism is convenient when replacing a loop, but harder when expressing that three executables can be built in parallel, but that a single shared object common to all of them must complete before starting, so it is difficult to provide \emph{complete} parallelism annotations.

Interestingly, given complete \emph{dependency} information (e.g. as available to \Make) it is possible to infer complete \emph{parallelism} information, but the difficulty of specifying complete dependnecy information is the attraction of a tracing based approach to build systems.

\subsubsection{Implicit Parallelism}
\label{sec:speculation}

The other source of parallelism is implicit, using speculation. If we can predict what commands are coming up next, and predict that their execution will not cause hazards, then we can speculatively execute them.  Such predictions can be made by simply recording the last known execution and using that information to predict the next execution.  If speculation suggests that running a command would be beneficial there are two possible approaches to take.

Firstly, the command can execute remotely, or in a sandboxed manner - ensuring all writes do \emph{not} end up on the file system, but are recorded to the cloud cache. In such a mode we are filling up the coud cache speculative, with the hope that when the future command does arrive it can be satisfied from the cache. However, running remotely requires syncronising the files across, or using a sycronise on demand approach (CITE Microsoft Remote Execution talk). Running with a sandbox which intercepts writes is not an easily available cross-platform feature, so we have not explored it further.

Alternatively, the command can execute locally, which is what \Rattle does. If the build execution lead to a hazard, and commands were executed speculatively, then it is possible that the hazard is entirely an artefact of speculation.  A simple approach to resolving this is to re-run the build without speculation.  A more nuanced approach is to use the hazard's classification, which can be categorised as per \S\ref{sec:proof:classify_hazard}, to decide whether to raise the error immediately (if the speculation was not at fault) or take alternative measures to resolve the hazard, such as selectively eliminating a subset of commands from speculation, or re-executing specific commands.


% Rattle's requirements:
% deterministic, if reading adn writing same file write does not rely on read; aka no appending,

% So up above in the intro to this section we say that ``we have freedom to schedule commands provided they do not cause hazards''.  But there is the distinct possiblity that they WILL cause hazards,
% so what does Rattle do in that situation?

% Rattle is executing a build and speculatively executing commands and has encountered a hazard, what does it do?
% As we've mentioned previously usually when a hazard occurs Rattle terminates because the build violates Rattle's consistency requirements.  But, we've just added this new thing called ``speuculation''
% which changes that a bit.  If we were speculating commands then it is possible that the build doesn't have a consistency violation and Rattle just made a mistake it needs to correct.

% How does Rattle determine which of the two it is?  Consistency violation or artefact os speculation?
% If both commands were executed from the build script, then its a consistency violation and Rattle should error and terminate
% If at least one of the commands was speculated, then Rattle might be able to recover.

% Ways to recover:  re-run certain commands;
% have a read-write hazard:
% speculatively write then require a read: Read read bad information.... we need to have to read read the correct file so we just restart.... same thing as above about dleeting speculative commands

% require a read then speculatively write: need to undo-write some how.....

% require a read then require a write; error obviously.

% have a write-write hazard:
% speculatively write then require a write: Since the write should not fundamentally be influenced by the read, then the required write should just overwrite whatever the speculative write did and
% its ok.  If they happened in parallel, then obviously re-run the required write, which is i guess what we call re-starting since the script would have passed that point already....

% We can delete all speculation info in this case, or only delete the failed command. Or some other sophisticated algorithm that deletes all commands that refer to a specific file or something.
% require a write then speculatively write: need to re-execute the required write, but since we passed it already, restart script and delete speculative commands...

% speculate write speculate write: Rattle will restart right now and can delete these speculation commands, but if nobody else reads/writes these files do we even care?

% require write /require write: error obviously

% todo: need to prove all of the above produce equivalent outputs...

In the case that speculating caused a hazard, there are potential resolutions \Rattle can take that will still result in a correct build.  The following are situations in which \Rattle speculating command(s) could lead to a hazard.

\begin{description}
\item [Two speculated commands write to the same file]
  Two commands in a build script writing to the same file violates \Rattle's consistency requirement and is a \emph{write-write hazard}.  But, if both commands are speculated and no commands in the build script read or write to this file then this should not affect the output of the build.  These two commands can be deleted from the list of speculative commands though in case the build needs to later be restarted.  In the following example, both speculated commands \emph{cp foo.o baz.o} and \emph{cp bar.o baz.o} write to \emph{baz.o}.  As long as no later command reads or writes to baz.o the outcome of the build will not change and so \Rattle does not need to take any action.
\begin{verbatim}
  cp foo.o baz.o [speculate]
  cp bar.o baz.o [speculate]
  ...
  gcc -o Main foo.o bar.o
\end{verbatim}

\item [A speculated command and a command in build script write to the same file]
  If \Rattle speculated a command which writes to the same file as a command in the build script, then another \emph{write-write hazard} has occurred entirely due to speculation.  Normally a \emph{write-write hazard} indicates that the build script violates \Rattle's consistency requirement, but in this case the violation might not have occurred if \Rattle didn't speculation the offending command.  To recover from this \Rattle can re-execute the build script, and remove the offending command(s) from the speculation list.

  Continuing with the previous example, a \emph{write-write hazard} occurs between \emph{gcc -o baz.c} and both of the \emph{cp} commands.  Rattle can recover from this by re-executing the build and removing both offending \emph{cp} commands from the speculation list.

\begin{verbatim}
  cp foo.o baz.o [speculate]
  cp bar.o baz.o [speculate]
  gcc -o baz.c
  .
  .
  -- re-run build
  gcc -o baz.c [re-execute]
  .
  .
\end{verbatim}

\item [A speculated command writes to a file that a command in build script reads from]
  If \Rattle speculated a command which writes to the same file as a command in the build script reads from, then a \emph{read-write hazard} has occurred entirely due to speculation.  Normally a \emph{read-write hazard} indicated that the build script violates \Rattle's consistency requirement, but in this case the violation might not have occurred if \Rattle didn't speculate the offending command.  To recover from this, \Rattle can re-execute the build script, and remove the offending command from the speculation list.

\begin{verbatim}
gcc -c foo.c
gcc -o Main foo.o bar.o
gcc -o Main foo.o baz.o [speculate]
./Main
-- re-run build
gcc -c foo.c [ don't re-execute because up to date]
gcc -o Main foo.o bar.o [re-execute]
./Main [re-execute]
\end{verbatim}

SS TODO : fix above because it isn't an example of this, actually an example of write-write

% that is actually not an example of this and I can't think of one.

\item [Speculated command reads from a file that another command writes to]
    A \emph{read-write hazard} will occur if a second command executes concurrently with or after the speculated command which read the file.  If these two commands were executed as part of the build script, the hazard would be \emph{non-recoverable} and \Rattle would error and terminate, but because one of them was executed speculatively it is possible there is no consistency violation in the build script.  Because the speculated read executed too soon, it potentially read incorrect data from the file.  To get the speculated read command to read the correct file however, \Rattle can just re-execute the command. For example:

\begin{verbatim}
  cp foo.o baz.o [speculate]
  gcc -c foo.c
  cp foo.o baz.o [re-execute]
\end{verbatim}

In the above example, a \emph{read-write hazard} occurred when \emph{cp foo.o baz.o} ran and then \emph{gcc -c foo.c} ran.  \emph{cp foo.o baz.o} copied the wrong version of \emph{foo.o}, but if \Rattle
runs \emph{cp foo.o baz.o} again after \emph{gcc -c foo.c} has completed, \emph{cp foo.o baz.o} will copy the correct version of \emph{foo.o} and the build can just continue.
\end{description}

A more detailed explanation of hazards and proofs of the above claims follows in section \S\ref{sec:proof:classify_hazard}.

% NM TODO: Write a lot about selectively eliminate a subset of commands from speculation (if speculation was at fault).

\section{Implementing \Rattle}
\label{sec:implementation}

We have implemented the design from \S\ref{sec:design} in a program called \Rattle. We use Haskell as the host language, meaning that the control logic is in Haskell.

\subsection{A \Rattle example}

\begin{figure}
\begin{verbatim}
import Development.Rattle
import System.FilePath
import System.FilePattern

main = rattle $ do
    let toO x = takeBaseName x <.> "o"
    cs <- liftIO $ getDirectoryFiles "." [root </> "*.c"]
    forP cs $ \c ->
        cmd "gcc -c" c
    cmd "gcc -o main.exe" (map toO cs)
\end{verbatim}
\caption{A Haskell/\Rattle version of the script from \S\ref{sec:monadic}}
\label{fig:rattle_example}
\end{figure}

\begin{figure}
\begin{verbatim}
-- The Run monad
data Run a = ... deriving (Functor, Applicative, Monad, MonadIO)
rattle :: Run a -> IO a

-- Running commands
data CmdOption = Cwd FilePath | ...
cmd :: CmdArguments args => args

-- Reading/writing files
cmdReadFile :: FilePath -> Run String
cmdWriteFile :: FilePath -> String -> Run ()

-- Parallelism
forP :: [a] -> (a -> Run b) -> Run [b]
\end{verbatim}
\caption{The \Rattle API}
\label{fig:api}
\end{figure}

A complete \Rattle script that compiles all \texttt{.c} files as per \S\ref{sec:monadic} is given in Figure \ref{fig:rattle_example}, with the key API functions in Figure \ref{fig:api}. Looking at the example, we see:

\begin{itemize}
\item A \Rattle script is a Haskell program. It makes use of ordinary Haskell imports, and importantly includes \texttt{Development.Rattle} offering the API from Figure \ref{fig:api}.
\item The \texttt{rattle} function takes a value the \texttt{Run} monad and executes it in \texttt{IO}. The \texttt{Run} type is the \texttt{IO} monad, enriched with a \texttt{ReaderT} containing a reference to shared mutable state (e.g. what commands are in flight, where to store metadata, location of shared storage).
\item All the control logic is in Haskell and may make use of any external libraries -- e.g. \texttt{System.FilePath} for manipulating \texttt{FilePath} values and \texttt{System.FilePattern} for directory listing. Taking the example of replacing the extension from \texttt{.c} to \texttt{.o}, we are able to abstract out this pattern as \texttt{toO} and reuse it later. Abitrary Haskell IO can be embedded in the script using \texttt{liftIO}. All of the Haskell code is considered control logic and will be repeated in every execution.
\item Commands are given to the build system part of rattle using the \texttt{cmd}. We have implemented \texttt{cmd} as a variadic function \cite{variadic_functions} which takes a command as a series of \texttt{String} (a series of space-separated arguments), \texttt{[String]} (a list of arguments) and \texttt{CmdOption} (command execution modifiers, e.g. to change the current directory), returning a value of type \texttt{Run ()}. The function \texttt{cmd} only returns once the command has finished executing (whether that is by actual execution, skipping, or fetching from external storage).
\item We have used \texttt{forP} in the example, as opposed to \texttt{forM}, which causes the commands to be given to \Rattle in parallel. The use of explicit parallelism for distinct elements of a list is convenient.
\end{itemize}

Looking at the functions from Figure \ref{fig:api} there are two functions this example does not use. The \texttt{cmdWriteFile} and \texttt{cmdReadFile} are used to perform a read/write of the file system through Haskell code, causing hazards to arise if necessary. Apart from these functions, it is assumed that all Haskell control code only reads and writes files which are not involved in any commands.

\subsection{Alternative \Rattle wrappers}

Given the above API, combined with the choice to treat the control logic as opaque, it is possible to write wrappers that expose \Rattle in new ways. For example, to literally run a series of commands, it is possible to write use the \Rattle program:

\vspace{3mm}
\begin{verbatim}
main = rattle $ do
    [x] <- liftIO getArgs
    src <- readFile x
    forM_ (lines src) cmd
\end{verbatim}
\vspace{3mm}

Here we get single command line argument, read it as a file, then run each command sequentially using \texttt{forM\_}. We use this script in our evaluation in \S\ref{sec:evaluation}.

An alternative API could be provided by openning up a socket, and allowing a \Rattle server to take command invocations through that socket. Such an API would allow writting \Rattle scripts in other languages, making use of the existing \Rattle implementation. While such a design should be easy, we have not actually implemented it yet.

\subsection{Hash forwarding}
\label{sec:forward_hashes}

To cope with the problem of non-deterministic build results reducing external cache hits, as described in \S\ref{sec:cloud_builds}, \Rattle uses hash forwarding. If the output of a command produces a file \texttt{foo}, and also a file \texttt{foo.forward}, then the hash of \texttt{foo.forward} is used instead of the hash of \texttt{foo}. Provided \texttt{foo.forward} includes all information that \texttt{foo} depends on (or a hash thereof), then the forwarding hash can be treated as a proxy for the non-deterministic file. Moreover, since the most common forwarding is to take the hash of all inputs to the command, that feature is available through the \texttt{CmdOption} named \texttt{Forward}.

The disadvantage of hash forwarding (other than the complexity) is that if the resulting \texttt{foo} eliminates some information compared to the inputs, then there might be instances when two \texttt{foo} values would be equal despite their forwarding hashes being different, causing fewer cache hits. The use of hash forwarding is very similar to the idea of deep-constructive traces from \citet{build_systems_a_la_carte} \S4.2.4, including the inability to benefit from unchanging outputs.

\subsection{Specific design choices}

Relative to the reference design in \S\ref{sec:design} we have made a few specific design choices, mostly in the name of implementation simplicity:

\begin{itemize}
\item All our predictions (see \S\ref{sec:speculation}) only look at the very last run. This approach is simple, and in practice, seems to be sufficient -- most build scripts are run on very similar inputs most of the time.
\item We currently use a shared drive for sharing build artefacts, allow the use of tools such as NFS or Samba to provide remote connectivity and thus full ``cloud builds''.
\end{itemize}

Say how we chose what to speculate. A 5 point plan - find something that doesn't hazard out.


\subsection{Tracing approaches}
\label{sec:tracing}

In \S\ref{sec:assume_tracing} we assume the existence of \emph{dependency tracing} which can, after a command completes, tell us which files that command read and which files it wrote. Unfortunately, such an API is \emph{not} part of the POSIX standard, and is not easily available on any standard platform. We aim to make \Rattle work on Linux, Mac and Windows, which requires using a variety of approaches. In this section we outline some of the approaches that can be used for tracing, their advantages and disadvantages:

\begin{description}
\item[Syscall tracing] On Linux you can trace every system call made, examine it's arguments, and thus record which files are accessed. Moreover, you can tell which have information about them queried (via the \texttt{stat} system call). The syscall tracking approach can be made complete, but because \emph{every} syscall must be hooked, can end up being a bit slow. This approach is used by \libbigbro \cite{bigbro}.
\item[Library preload] On both Linux and Mac most programs refer to a dynamically linked C library for their file accesses. By using \texttt{LD\_LIBRARY\_PRELOAD} it is possible to inject another dynamic library into the program memory which intercepts the relevant C library calls, recording which files are read and written. This approach is simpler than hooking syscalls, but only works if all access to syscalls is made through the C library. For programs written in Go \cite{go}, the design is that the syscalls are invoked directly, meaning that nothing is observed by tracing the C library. For statically linked programs, the location of the C library is not available, and similarly they will not be traced. While the technique works on a Mac, from Mac OS X 1.10 or higher system binaries can't be traced due to System Integrity Protection\footnote{\url{https://developer.apple.com/library/content/documentation/Security/Conceptual/System_Integrity_Protection_Guide/ConfiguringSystemIntegrityProtection/ConfiguringSystemIntegrityProtection.html}}. As an example, the C compiler is typically installed as a system binary. It is possible to disable System Integrity Protection (but not recommended by Apple), to use non-system binaries (e.g. those supplied by \Nix \cite{nix} work well), or to copy the system binary to a temporary directory (which works provided the binary does not afterwards invoke another system binary to do its work). The library preload mechanism is implemented by \Fsatrace \cite{fsatrace} and the copying system binaries trick is implemented by \Shake \cite{shake}.
\item[File system tracing] An alternative approach on Linux would be to implement a file system that reported which files were accessed. We are aware of one such implementation, \tracedfs \cite{tracedfs}, which is unfortunately not yet complete. Such an approach would track all accesses, but may require privileges to mount a file system.
\item[Custom Mac tracing] \BuildXL \cite{buildxl}\footnote{\url{https://github.com/Microsoft/BuildXL/blob/master/Documentation/Specs/Sandboxing.md\#macos-sandboxing}} uses a Mac sandbox based on KAuth combined with TrustedBSD Mandatory Access Control (MAC) to both detect which files are accessed and also block access to specific files. The approach is based on internal Mac OS X detils which have been reversed engineered, some of which are deprecated and scheduled for removal.
\item[Windows Kernel API hooking] On Windows it is possible to hook the Kernel API, which can be used to detect when files are accessed. Hooking the kernel detects everything, but is not trivial. In particular, 32bit v 64bit differences are problematic, as Windows trampolines between the two and the hooks must cope with the differences using assembly code. Furthermore, some antivirus products are more likely to (incorrectly) detect such programs as viruses. The Windows kernel hooking is available in both \Fsatrace and \libbigbro, although without the 32bit calling 64bit assembly code.
\end{description}

\Rattle currently uses \Fsatrace as wrapped by \Shake for tracing. That means it uses library preloading on Linux/Mac and kernel hooking on Windows. The biggest limitations in practice vary by OS:

\begin{itemize}
\item On \textbf{Linux} it can't trace into Go programs and statically linked binaries. We are planning to integrate \libbigbro as an alternative, to address these concerns.
\item On \textbf{Mac} it can't trace into system binaries called from other system binaries. We recommend using \Nix binaries if this limitation is problematic.
\item On \textbf{Windows} it can't trace into 64bit programs invoked by 32bit programs. In most cases the 32bit binaries can easily be replaced by 64bit binaries. The only instance we've seen thus far was an outdated version of \texttt{sh} one of the authors of this paper downloaded over five years ago, which was easily remedied with a newer version.
\end{itemize}

However, in practice, none of the limitations have been overly problematic in the examples we have explored.

\subsection{Possible Tracing Enhancements}

Using the approaches listed above we are able to detect when files are opened for read or write access. After the command has finished, we then report which files were accessed and in what mode. That restricted interface has driven the design of \Rattle to ensure that it is possible to implement \Rattle robustly in realistic situations. However, there are places where a more refined interface would be useful.

Tracing file accesses is the most important file system information, but there is other information about a file a command may query -- for example the existence of a file or its modification time and size. We model these as reads of a file, with all the associated read/write hazard detection. % SS I am confused by the following line; the bit about being a proxy for incremental building is confusing.
We don't rerun if a file has changed modification time as we suspect it likely that modification time is used as a proxy for incremental building, which \Rattle already does. By virtue of \Rattle storing the hash and using it to determine if a file has changed, a size change is going to cause a rebuild. For the existence test, as a consequence of treating it as a read, \Rattle will rebuild slightly too much since the dependency is now on the entire contents rather than just the existence.  \Rattle also has to deal with files that are read but don't exist.

% SS elaborate what it means for rattle to deal with files that don't exist.  I'm assuming this just mean doing things like trying to get modtime/hash for files that don't exist etc.


The current mechanisms provide information about which files were accessed at the end. If they could interactively stream back which files were read, it would lead to better guesses in speculation (see \S\ref{sec:speculation}) about what files had been accessed by running system commands. With the tracing mechanisms available, there is no reason they couldn't report on which files had been read at any point via a socket, but such engineering work has not been done.

The approaches only report whether a file was accessed or not -- not at what time the access occurred. With a precise time it would be possible to have a more refined notion of hazard when commands are run in parallel. However, assuming commands are relatively short and that generally commands should not be relying on race conditions to avoid hazards, we consider the current approach quite reasonable. The enhancement could be implemented if desired.

It would be nice if the accesses could be intercepted and denied if there was a hazard violation, but that would require two way communication from the tracing to \Rattle. In many of the contexts being called (e.g. kernel tracing or syscall hooking) such communication is problematic, so while useful, we don't have it.

Finally, it would be useful to know the purpose of a file system access, but that is generally not available to the tracing system. As an example, many programs create, use and then delete temporary files. If multiple programs are run in a single \Rattle build then it is not unreasonable (and in fact, somewhat common) to see them access the \emph{same} temporary file, but distinct instances of it, which should not be reported as a hazard despite multiple writes. We solve such issues by ignoring reads and writes to the temporary directory.

Howver, most of all, standardisation would be fantastic. The tracing pieces of \Rattle have been some of the most difficult engineering pieces to develop. We suggest that such features should be standardised and made available in a cross-platform way.

NOTE: There are lots of issues around things like deleting, renaming files and directories which don't track as well as they should.

\section{Proofs}
\label{sec:proof}

The design of \Rattle relies on taking a sequence of commands and instead of running them all sequentially, rather running them in different orders. In this section we prove that the manipulations we perform are safe with respect to the reference semantics.

\subsection{Hazards, formally}

A hazard occurs when a Build command writes to a file that a previous command has already read from or written to.  Hazards can be classified as either: \emph{read-write hazards} or \emph{write-write hazards}.  In section \S\ref{sec:proof:classify_hazard} hazards are explained in greater detail, but for the following sections it suffices to say that in a build executed sequentially a
hazard occurs when a command read or writes a file and a later command writes to the same file.  In a build executed in parallel with speculation, a hazard can occur when a command reads or writes a file and a later command writes to the same file, or if two commands running concurrently access the same file and are not both just reading the file.  Therefore, for a sequential \Rattle build to have no hazards means for every command $c$ in the build, if $c$ reads or writes a file, no later command in the build will write to that file.  For a parallel \Rattle build to have no hazards means for every command $c$ in the build, if $c$ reads or writes a file, no later command will write to that file, and no command will concurrently read the file if it is being written, or write to the file if it is being read.

% SS seems this section should just be a brief explanation of hazards so the following claims of ``no hazards'' makes sense.

\subsection{No rebuilds}
\label{sec:proof:no_rebuild}

We prove that a build system with deterministic control logic and with no hazards always results in no rebuilds when no changes have occurred.

\begin{align}
  build :: \bar{command} \\
  command :: (cmd, reads :: \bar{file}, writes:: \bar{file}) \\
  file :: (name, content) \\
  eq(f1 :: file, f2 :: file) = name(f1) = name(f2) \&\& content(f1) = content(f2) \\
  hazard :: (file, command, command) \\
  files(command) = reads(command) \cup writes(command)
\end{align}

In a build system without hazards there is at most one write to any file, which occurs before any reads of that file. We can therefore prove there are no rebuilds by showing the first command can't rebuild, and proceeding by induction.

Let $B$ be a build with no hazards when executed sequentially.

\begin{description}
\item[Base case: $|B| = 1$]

  $\forall f \in reads(B[0]) \cup writes(B[0]), f \text{ has not changed}$, therefore, $B[0]$ does not run, and $\forall f \in writes(B[0])$ are not written to.

  % show that inputs of c = B[0] have not changed
  % therefore, c[0] will not run
  % therefore, the outputs of B, which are third(c[0]) will not have changed

\item[Induction step: $|B| = n+1$]
  Let us assume the above claim is true for a build, $|B| = n$.  Also, Let $D$ be the set of files written to during the build.
  It follows that, $\forall c \in B, \forall f \in writes(c), f \notin D$.
  Let us assume we have a build $A=B$, and $A$ has an extra command $A[n]=c1$.

  $\forall f \in files(c1), f \notin D, f \text{ has not changed}$, therefore, $A[n]$ does not run and $\forall f \in writes(A[n])$ are not written to.

\end{description}

\subsection{Reordered builds}
\label{sec:proof:reorder}

Given a script with no hazards when executed sequentially, we can show that any interleaving of those commands that also has no hazards will result in an equivalent output.

Let us assume we have a build $A$, with no hazards, meaning $\forall c = A[i] \in A, \forall f \in files(c), f \notin \bigcup^{|A|}_{j=i+1} writes(A[j])$.

Let us assume there exists a build $B$, that also has no hazards, meaning $\forall c = B[i] \in B, \forall f \in files(c), f \notin \bigcup^{|B|}_{j=i+1} writes(B[j])$.
Let us also assume that $|A|$ = $|B|$, and $\forall c \in A, \exists d \in B, s.t. cmd(c) = cmd(d)$.

Prove that $\forall c \in B, \exists d \in A, s.t. cmd(c) = cmd(d) \land reads(c) = reads(d) \implies writes(c) = writes(d) \implies \bigcup^{|B|}_{i=0} writes(B[i]) = \bigcup^{|A|}_{i=0} writes(A[i])$.

$\forall c = B[i] \in B, \forall f \in files(c), f \nexists d = B[j] \in B, s.t. i < j$

$\forall c \in B, \exists e = A[k] \in A, s.t. cmd(c) = cmd(e) \land \forall f \in files(c), \exists g \in files(e) \land \forall f \in files(e), \exists g \in files(c)$

$\implies \forall c \in B, \exists d \in A, s.t. cmd(c) = cmd(d) \land reads(c) = reads(d) \implies writes(c) = writes(d)$

$\implies \bigcup^{|B|}_{i=0} writes(B[i]) = \bigcup^{|A|}_{i=0} writes(A[i])$.
Therefore, for any script with no hazards when executed sequentially, any interleaving of those commands that also has no hazards will result in an equivalent output.

\subsection{Parallel commands}
\label{sec:proof:parallel}

Given a script with no hazards when executed sequentially, we can show that any parallel or interleaved execution without hazards will also be equivalent.

Proof by contradiction.

Let us assume we have a build $A$, which has no hazards.  Let us also assume there is an alternative execution of $A$, which has no hazards, but is not equivalent.
Recall that for the builds to be equivalent, they write to the same set of files and a file will have the same hash regardless of the build that wrote it.
Therefore, the parallel execution must write to a different set of files than $A$ or at least one of those files has a different hash.

\paragraph{Parallel build writes to a file $A$ does not write to, or $A$ writes to a file the parallel build does not write to.}
  There must exist a cmd $c$ which when run in parallel with one or more other commands, writes to a different set of files than when run sequentially.  The set of files written to by $c$ is affected by the content of the cmd, and the files it reads.  Since, the content of the cmd is the same between the parallel and sequential builds, the input files must differ.
  By the defintion of the parallel build of $A$ having no hazards, any files read by $c$ must be written to by commands that precede $c$ in the build.  Therefore, the files written by $c$ in the parallel build will be the same as the files written by $c$ in the sequential build.  Therefore, a contradiction.


\paragraph{Parallel build writes to a file $f$ whose hash is $h$ and $A$ writes to the same file but its hash is $k$; $h \neq k$}
  There must exist a cmd $c$ which when run in parallel with one or more other commands, writes to the file $f$ and produces the hash $h$ and when run sequentially in build $A$ produces the hash $k$.
  The files written to by $c$, and their hashes, is affected by the content of the command and the files read by the command.  Since $c$ is unchanged between $A$ the sequential run of $A$ and the parallel run of $A$, the files and their hashes read by $c$ must differ.  By the definition of parallel $A$ having no hazards, any files read by $c$ must be written to by commands that precede $c$ in the build.  And because parallel $A$ and sequential $A$ contain the same commands, $c$'s input must be the same regardless of whether $A$ was run in parallel or sequentially.  Therefore, a contradiction.

  % clean up

\subsection{Additional commands have no effect}
\label{sec:proof:additional}

Given a script with no hazards when executed sequentially, we can show that speculating unnecessary commands will not affect the build's output. % hmm

Proof by induction.

Let us assume we have a build $A$ which has no hazards when executed sequentially.

Base case:  Build $A$ has 1 command $c$.  Let us assume we have a command $d$ that does not write to any file read or written by $c$.  By this definition of $d$, it is obvious that $d$ running before or
concurrently with $c$ as part of build $A$, will not affect the files written by $c$ and therefore will not affect the output of $A$.

Inductive case: Let us assume the above claim is true for a build with $n$ commands.  Let us show the claim is true for a build of $n+1$ commands.

Let $A$ have $n+1$ command.  From the inductive hypothesis we know the output of the first $n$ commands is unchanged.  And, we know that $c$ does not write to any file read or written by command $n+1$.  And, because the build has no hazards, all files read by $n+1$ were written to before $n+1$ ran, so the output files of $n+1$ remain unchanged as well.  Therefore, the output of $A$ remained unaffected by $c$.

\subsection{Classifying Hazards}
\label{sec:proof:classify_hazard}

% SS An explanation of the classification of hazards and a proof of the claims in 2.7

In section \S\ref{sec:speculation} we describe how speculation can be used to execute builds in parallel.  Sometimes when commands are speculated, hazards can occur that wouldn't have if the build script was executed sequentially.  In that section we describe when speculation can lead to hazards and how they can be recovered from.  Here we offer a more precise classification of those hazards and proofs related to them.

Hazards are first classified as either \emph{read-write} or \emph{write-write}.  Hazards can be further classified by how the hazard occurred.

\paragraph{Non Recoverable}
A hazard is classified as \emph{non-recoverable} if it is triggered by a consistency violation in the build script.  A \emph{non-recoverable hazard} always results in the build terminating immediately with an error.  For two examples of \emph{non-recoverable hazards} see section \ref{sec:hazards}.

\paragraph{Recoverable}
A hazard is classified as \emph{recoverable} if it is caused by a speculated command that read a file which was concurrently or later written to by another command.  In this situation the speculated command likely read stale data and if re-executed would read up-to-date data.  Here is an example where \emph{cp foo.o baz.o} potentially copied the wrong version of \emph{foo.o} because it executed before \emph{gcc -c foo.c} completed.  If \emph{cp foo.o baz.o} re-executes it will read the new \emph{foo.o} produced by \emph{gcc -c foo.c}.

\begin{verbatim}
cp foo.o baz.o [speculate]
gcc -c foo.c
cp foo.o baz.o [re-execute]
\end{verbatim}

Proof that if the read command of a \emph{recoverable} hazard is re-executed the same hazard will not re-occur.

% SS todo above proof

\paragraph{Restartable} % when speculation causes a consistency violation. or a command to read incorrect data.
If a speculatively executed command wrote to a file that was later written to or read by another command, then incorrectness was potentially introduced into the build.  In the following example \emph{cp foo.o baz.o} was speculated, and wrote to \emph{baz.o}, then \emph{gcc -c baz.c} was executed and also wrote to \emph{baz.o}.  Normally a build script that executed these two commands would violate \Rattle's consistency properties, but in this case the consistency violation might have been avoided if \emph{cp foo.o baz.o} were not executed by the build.  Therefore, if the build script was re-executed and did not speculate \emph{cp foo.o baz.o} then the consistency violation might not occur.

\begin{verbatim}
cp foo.o baz.o [speculate]
gcc -c baz.c
\end{verbatim}

Another possible situation is \Rattle executing something like the following:

\begin{verbatim}
cp old-foo.c foo.c [speculate]
gcc -c foo.c
\end{verbatim}

Maybe the build script originally included \emph{cp old-foo.c foo.c}, but it was removed in the most recent version of the build.  By speculating this old command the build caused \emph{gcc -c foo.c} to
potentially build the wrong \emph{foo.c}.  If the build  were to re-execute and not speculate \emph{cp old-foo.c foo.c} then \emph{gcc -c foo.c} would build the correct \emph{foo.c} during the next build.  % This is a bad example since Rattle would have corrupted foo.c with the bad copy.  Need a new example or address this

%Proof that restartable hazard are handled

\section{Evaluation}
\label{sec:evaluation}

\subsection{Validating the claims from \S\ref{sec:introduction}}

In \S\ref{sec:introduction} we claimed that the following build script is ``just as good'' as a proper \Make script.

\begin{verbatim}
gcc -c main.c
gcc -c util.c
gcc -o main.exe main.o util.o
\end{verbatim}

There are two axes on which to measure ``just as good'' -- correctness and performance. Performance can be further broken down into how much rebuilds, how much parallelism can be acheived and overhead.

\paragraph{Correctness} \Rattle is correct, in that the reference semantics is running all the commands, and as we have shown in \S\ref{sec:rattle} and \S\ref{sec:proof}, \Rattle obeys that semantics. In contrast, the \Make version may have missing dependencies which causes it not to rebuild. Examples of failure to rebuild include both if \texttt{gcc} changes, one of the system headers used by \texttt{gcc} or any headers included but not listed in the \Make script.

\paragraph{Rebuilding too much} \Rattle never rebuilds too much, as if none of the inputs to a command have changed, then nothing rebuilds. As a matter of implementation, to implement cloud builders as per \ref{sec:cloud_builds}, \Rattle uses hashes of the file contents. In contrast, \Make uses the modification time, so if a file is resaved, but it's contents do not change, \Make will rebuild but \Rattle will not. It would be possible for \Make to use modification times, if it chose to store additional metadata not in the file system.

\paragraph{Parallelism} The script from \S\ref{sec:introduction} has three commands, the first two of which can run in parallel, the the third of which must wait for the first two to finish. \Make is given all this information by dependencies, so will always acheive as much parallelism as is available. In constrast, \Rattle has no such knowledge, so has to recover the parallelism by speculation, as per \ref{sec:speculation}. During the first execution, \Rattle has no knowledge about even which commands are coming next (as described in \S\ref{sec:monadic}), so has no choice but to execute each command serially, with less parallelism than \Make. In subsequent executions \Rattle will use speculation to always speculate on the second command (as it never has a hazard with the first), but never speculate on the third until the first two have finished (as they are known to conflict). Interesting, sometimes \Rattle executes the third command (because it got to that point in the script), and sometimes it speculates it (because the previous two have finished) -- it is a race condition where both alternatives are equivalent. So \Rattle has less parallelism on the first execution, but using the cloud features, that can be reduced to the first execution \emph{ever}, rather than the first execution for a given user.

\paragraph{Overhead} The overhead inherent in \Rattle is greater than that of \Make as it hashes files, traces command execution, computes potential hazards to figure out speculation and writes to a shared cloud store. To measure the overhead we created a very simple pair of \texttt{main.c} and \texttt{util.c} files where \texttt{main.c} calls \texttt{printf} using a number computed by a function in \texttt{util.c}. We then measured the time to do an initial build, a rebuild when nothing had changed, a rebuild with whitespace changes to \texttt{main.c}, and a rebuild with changes to both C files. We did all the above both single threaded and with two and three threads. The numbers are:

\begin{tabular}{|l|r|r|r|}
Number of threads & 1 & 2 & 3 \\
Initial build \\
Nothing changed \\
\texttt{main.c} changed \\
Both C files changed \\
\end{tabular}

We see \Rattle has low overhead, in the millisecond range. We see three threads has no change over two threads, as this build contains no more parallelism opportunities.

\subsection{Rest of the evaluation}

Let's consider the VIM build system. It's wrong in lots of ways Sarah can describe. But these were expert software engineers - has anyone got it right? To a first approximation, no. Unless you go for something like Bazel, and then it's tedious but sandboxing makes you get it right.

\section{Related work}
\label{sec:related}

Lot's on Fabricate.

We're not the first people to do that, but the previous solutions have been hampered by making parallelism hard. We solve that with speculation.

\section{Conclusion and future work}
\label{sec:conclusion}

In this paper we present a system that can take a sequence of actions and treat them as a build script. From the user perspective, they get most of the benefits of a conventional build system (incrementality, parallelism) but with lower cost (less time thinking about dependencies, only supplying one valid ordering). We believe this approach is viable for small projects and could be adopted immediately. For larger projects, it is as yet unclear whether the build can be described compositionally or not. \Rattle also has no story for how services such as remotely executing commands might work.

Comparing \Rattle to other build systems, e.g. \Make, it is fair to say that \Rattle presents a simpler interface (no dependencies), but a more complex implementation model. In particular the overhead of tracing, the overhead of the parts that are always run and the potential for speculation to not work out can mean that \Rattle is slower.

As future work, we think that \Rattle would benefit from real users.

\paragraph{Acknowledgements} We'd like to thank Jorge Acereda and David Roundy for their work on \Fsatrace and \libbigbro respectively.



\bibliography{paper}

\end{document}
