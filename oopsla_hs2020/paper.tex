% DEADLINE: Mon 14 Sep 2020
% LIMITS: 27 pages, bibliography does not count
% APPROACH: Double blind
% CFP: https://2020.splashcon.org/track/splash-2020-oopsla#Call-for-Papers

\documentclass[acmsmall,screen,review,10pt]{acmart}
\settopmatter{printacmref=false}
\renewcommand\footnotetextcopyrightpermission[1]{} % Remove after submission

\bibliographystyle{ACM-Reference-Format}
\citestyle{acmauthoryear}

\usepackage{xspace}
\usepackage{pgfplots}
\pgfplotsset{compat=1.13}
\pgfplotsset{title style={at={(0.5,0.7)}}}

\begin{document}

\newcommand{\Make}{\textsc{Make}\xspace}
\newcommand{\Rattle}{\textsc{Rattle}\xspace}
\newcommand{\Fabricate}{\textsc{Fabricate}\xspace}
\newcommand{\Bazel}{\textsc{Bazel}\xspace}
\newcommand{\Buck}{\textsc{Buck}\xspace}
\newcommand{\Shake}{\textsc{Shake}\xspace}
\newcommand{\Bigbro}{\textsc{BigBro}\xspace}
\newcommand{\Fac}{\textsc{Fac}\xspace}
\newcommand{\Fsatrace}{\textsc{Fsatrace}\xspace}
\newcommand{\tracedfs}{\textsc{Traced-Fs}\xspace}
\newcommand{\BuildXL}{\textsc{BuildXL}\xspace}
\newcommand{\Nix}{\textsc{Nix}\xspace}
\newcommand{\Memoize}{\textsc{Memoize}\xspace}
\newcommand{\Stroll}{\textsc{Stroll}\xspace}
\newcommand{\Pluto}{\textsc{Pluto}\xspace}
\newcommand{\PIE}{\textsc{PIE}\xspace}

\newcommand{\postparagraphs}{\vspace{3mm}\noindent}


\title{Build Scripts with Perfect Dependencies}

\author{Sarah Spall}
\affiliation{
  \institution{Indiana University}
  \country{United States}
}
\email{sjspall@iu.edu}

\author{Neil Mitchell}
\affiliation{
  \institution{Facebook}
  \country{UK}
}
\email{ndmitchell@gmail.com}

\author{Sam Tobin-Hochstadt}
\affiliation{
  \institution{Indiana University}
  \country{United States}
}
\email{samth@cs.indiana.edu}


\begin{abstract}
Build scripts for most build systems describe the actions to run, and the dependencies between those actions---but often build scripts get those dependencies wrong.
Most build scripts have both \emph{too few dependencies} (leading to incorrect build outputs) and \emph{too many dependencies} (leading to excessive rebuilds and reduced parallelism). Any programmer who has wondered why a small change led to excess compilation, or who resorted to a ``clean'' step, has suffered the ill effects of incorrect dependency specification.
We outline a build system where dependencies are \emph{not specified}, but instead \emph{captured by tracing execution}.
The consequence is that dependencies are always correct by construction and build scripts are easier to write.
The simplest implementation of our approach would lose parallelism, but we are able to recover parallelism using speculation.
\end{abstract}

%% 2012 ACM Computing Classification System (CSS) concepts
%% Generate at 'http://dl.acm.org/ccs/ccs.cfm'.
\begin{CCSXML}
<ccs2012>
<concept>
<concept_id>10011007.10011006.10011073</concept_id>
<concept_desc>Software and its engineering~Software maintenance tools</concept_desc>
<concept_significance>500</concept_significance>
</concept>
</ccs2012>
\end{CCSXML}
\ccsdesc[500]{Software and its engineering~Software maintenance tools}
%% End of generated code
\keywords{build systems, functional programming}

\maketitle

\section{Introduction}
\label{sec:introduction}

Build scripts \cite{build_systems_a_la_carte} describe \emph{commands to run} and \emph{dependencies to respect}. For example, using the \Make build system \cite{make}, a build script might look like:

\vspace{3mm}
\begin{verbatim}
main.o: main.c
    gcc -c main.c
util.o: util.c
    gcc -c util.c
main.exe: main.o util.o
    gcc -o main.exe main.o util.o
\end{verbatim}
\vspace{3mm}

This script contains three rules. Looking at the first rule, it says \texttt{main.o} depends on \texttt{main.c}, and is produced by running \texttt{gcc -c main.c}. But what if we copied the commands into a shell script? We get:

\vspace{3mm}
\begin{verbatim}
gcc -c main.c
gcc -c util.c
gcc -o main.exe main.o util.o
\end{verbatim}
\vspace{3mm}

That's shorter, simpler and easier to follow. Instead of declaring the outputs and dependencies of each command, we've merely given one valid ordering of the commands (we could equally have put \texttt{gcc -c util.c} first). There are two main downsides: 1) everything will always rebuild even if it's dependencies haven't changed; 2) the commands will run sequentially, while \Make can run the two \texttt{gcc -c} commands in parallel. But there are some benefits. We've fixed an inadvertent bug -- these commands depend on the undeclared dependency \texttt{gcc}, and potentially whatever header files are used by \texttt{main.c} and \texttt{util.c}. Furthermore, as the files \texttt{main.c} and \texttt{util.c} evolve, and their dependencies change (by changing the \texttt{\#include} directives), the shell script remains correct, while the \Make script \emph{must} be kept consistent (or builds will become stagnant).

\subsection{Overview}

% SS I changed script -> build; not sure if this is correct or wrong?
In this paper we show how to take the above shell script and gain the benefits of a \Make build (\S\ref{sec:design}). Firstly, we can skip those commands whose dependencies haven't changed by \emph{tracing} which files they read and write (\S\ref{sec:skipping_unnecessary}). Secondly, we can run some commands in parallel, using \emph{speculation} to guess which future commands won't interfere with things already running (\S\ref{sec:speculation}). The key to speculation is a robust model of what ``interfering'' means -- we call a problematic interference a \emph{hazard}, which we define in \S\ref{sec:hazards} and use to prove important properties about the safety of speculation in \S\ref{sec:proof}.

We have implemented these techniques in a build system called \Rattle\footnote{\url{https://github.com/author_name_omitted/rattle}}, introduced in \S\ref{sec:implementation}, which embeds commands in a Haskell script. A key part of the implementation is the ability to trace commands, using techniques we describe in \S\ref{sec:tracing}. To evaluate our claims, and properly understand the subtlties of our design, we converted existing \Make scripts into \Rattle scripts, and discuss the performance characteristics and \Make script bugs uncovered in \S\ref{sec:evaluation}. Our design can be considered a successor to the \Memoize build system \cite{memoize}, and we compare the differences between that and other related work in \S\ref{sec:related}. Finally, in \S\ref{sec:conclusion} we conclude and describe future work.

\section{Build Scripts from Commands}
\label{sec:design}

Our goal is to design a build system where a build script is simply a list of commands. In this section we develop our design, starting with the simplest system that just executes all the commands in order, and ending up with the benefits of a conventional build system.

\subsection{Executing commands}
\label{sec:executing_commands}

Given a build script as a list of commands, like in \S\ref{sec:introduction}, the simplest execution model is to run each command sequentially in the order they were given. Importantly, we require the list of commands is ordered, such that any dependencies are produced before they are used. We consider this sequential execution the reference semantics, and as we develop our design further, require that any optimised/cached implementation gives the same results.

\subsection{Value-dependent commands}
\label{sec:monadic}

While a static list of commands is sufficient for simple builds, it is
limited in its expressive power. Taking the build script from
\S\ref{sec:introduction}, the user might really want to compile and
link \emph{all} \texttt{.c} files -- not just those explicitly listed
by the script. A more powerful script, along with its translation to
working \Rattle code, is shown in figure~\ref{fig:getdir}.


\vspace{3mm}
\begin{figure}[h!]
\begin{minipage}{.35\textwidth}
  \begin{small}
\begin{verbatim}
FILES=$(ls *.c)                                       
for FILE in $FILES; do
    gcc -c $FILE
done
gcc -o main.exe ${FILES%.c}.o
\end{verbatim}
  \end{small}
\end{minipage}% This must go next to `\end{minipage}`
\begin{minipage}{.65\textwidth}
  \begin{small}
\begin{verbatim}
  main = rattle $ do
    cs <- liftIO $ getDirectoryFiles "." [root </> "*.c"]
    forP cs $ \c -> cmd "gcc -c" c
    let toO x = takeBaseName x <.> "o"
    cmd "gcc -o main.exe" (map toO cs)
\end{verbatim}
\end{small}
\end{minipage}
  
  \caption{Value-dependent build scripts, in Shell and Rattle}
  \label{fig:getdir}
\end{figure}

This script now has a curious mixture of commands (\texttt{ls}, \texttt{gcc}), control logic (\texttt{for}) and simple manipulation (changing file extension\footnote{We take some liberties with shell scripts around replacing extensions, so as not to obscure the main focus.}). Importantly, there is \emph{no fixed list of commands} -- the future commands are determined based on the results of previous commands. Concretely, in this example, the result of \texttt{ls} changes which \texttt{gcc} commands are executed. The transition from a fixed list of commands to a dynamic list matches the \texttt{Applicative} vs \texttt{Monadic} distinction of \citet[\S3.5]{build_systems_a_la_carte}.

There are three approaches to modelling a dynamic list of commands:

\begin{enumerate}
\item We could consider the commands as a stream given to the build system one by one as they are available. The build system has no knowledge of which commands are coming next or how they were created. In this model, a build script supplies a stream of commands, with the invariant that dependencies are produced before they are used, but provides no further information. The main downside is that it becomes impossible to perform any analysis that might guide optimisation.
\item We could expose the full logic of the script to the build system, giving a complete understanding of what commands are computed from previous output, and how that computation is structured. The main downside is that the logic between commands would need to be specified in some constrained domain-specific language (DSL) in order to take advantage of that information. Limiting build scripts to a specific DSL complicates writing such scripts.
\item It would be possible to have a hybrid approach, where dependencies between commands are specified, but the computation is not. Such an approach still complicates specification (some kind of dependency specification is required), but would allow some analysis to be performed.
\end{enumerate}

% SS build -> shell; I think this was what was meant but change back if I misunderstood.
In order to retain the desired simplicity of shell scripts, we have chosen the first option, modelling a build script as a sequence of commands given to the build system. Future commands may depend on the results of previous commands in ways that are not visible to the build system. The commands are produced with ``cheap'' functions such as control logic and simple manipulations, for example, using a for loop to build a list of object files. We consider the cheap commands to be fixed overhead, run on every build, and not cached or parallelised in any way. If any of these cheap manipulations becomes expensive, they can be replaced by a command, which will then be visible to the build system. The simple list of commands from \S\ref{sec:introduction} is a degenerate case of no interesting logic between commands.

An important consequence of the control logic not being visible to the build system is that the build system has no prior knowledge of which commands are coming next, or if they have changed since it last executed them. As a result, even when the build is a simple static script such as from \S\ref{sec:introduction}, when it is manually edited, the build will execute correctly. The build system is unaware if you edited the script, or if the commands were conditional on something that it cannot observe. Therefore, this model solves the problem of self-tracking from \citet[\S6.5]{build_systems_a_la_carte}.

\subsection{Dependency tracing}
\label{sec:assume_tracing}

For the rest of this section we assume the existence of \emph{dependency tracing} which can tell us which files a command accesses. Concretely, we can run a command in a special mode such that when the command completes (and not before) we can determine which files it read and wrote; these files are considered to be the command's inputs and outputs respectively. We cannot determine at which point during the execution these files were accessed, nor which order they were accessed in. We cannot prevent or otherwise redirect an in-progress access. We discuss the implementation of dependency tracing, and the reasons behind the (frustrating!) limitations, in \S\ref{sec:tracing}.

\subsection{Skipping unnecessary commands}
\label{sec:skipping_unnecessary}

When running a command, we can use dependency tracing to capture the files which that command reads and writes, and then after the command completes, record the cryptographic hashes of the contents of those files. If the same command is ever run again, and the inputs and outputs haven't changed (have the same hashes), it can be skipped. This approach is the key idea behind both \Memoize\cite{memoize} and \Fabricate\cite{fabricate}. However, this technique makes the assumption that commands are pure functions from their inputs to their outputs, meaning if a command's input files are the same as last time it executed, it will write the same values to the same set of files. Below are four ways that assumption can be violated, along with ways to work around it.

\textbf{Non-deterministic commands} Many commands are non-deterministic -- e.g. the output of \texttt{ghc} object files contains unpredictable values within it (a consequence of the technique described by \citet{lennart:unique_names}). We assume that where such non-determinism exists, any possible output is equally valid. %TODO R1 wants brief description of technique mentioned here

\textbf{Incorporating external information} Some commands incorporate system information such as a timestamp, so a cached value will be based on the first time the command was run, not the current time. For compilations that embed the timestamp in metadata, the first timestamp is probably fine. For commands that really want the current time, that step can be lifted into the control logic (as per \S\ref{sec:monadic}) so it will run each time the build runs. Similarly, commands that require unique information, e.g. a GUID or random number, can be moved into control logic and always run.

\textbf{Reading and writing the same file} If a command both reads and writes the same file, and the information written is fundamentally influenced by the file that was read, then the command never reaches a stable state. As an example, \verb"echo x >> foo.txt" will append the character \texttt{x} every time the command is run. Equally, there are also commands that read the existing file to avoid rewriting a file that hasn't changed (e.g. \texttt{ghc} generating a \texttt{.hi} file) and commands that can cheaply update an existing file in some circumstances (the Microsoft C++ linker in incremental mode). We make the assumption that if a command both reads and writes to a file, that the read does not meaningfully influence the write, otherwise it is not really suitable as part of a build system because the command never reaches a stable state and will re-run every time the build is run.
% SS is it really the reference semantics that are violated? Reference semantics are just whatever
% sequential version does right?

% SS I probably worte that assumptions are violated but now it doesnt make sense to me what assumptions were're talking about

\textbf{Simultaneous modification} If a command reads a file, but before the command completes something else modifies the file (e.g. a human or untracked control logic), then the final hash will not match what the command saw. It is possible to detect such problems with reads by ensuring that the modification time after computing the hash is before the command was started. For simultaneous writes the problem is much harder, so we require that all files produced by the build script are not simultaneously written to.

\postparagraphs In general we assume all commands given to the build system are well behaved and meet the above assumptions.

\subsection{Cloud builds}
\label{sec:cloud_builds}

We can skip execution of a command if all the files accessed have the same hashes as any previous execution (\S\ref{sec:skipping_unnecessary}). However, if \emph{only the files read} match a previous execution, and the files that were written have been stored away, those stored files can be copied over as outputs \emph{without} rerunning the command. If that storage is on a server, multiple users can share the results of one compilation, resulting in cloud build functionality. While this approach works well in theory, there are some problems in practice.

\textbf{Machine-specific outputs} Sometimes a generated output will only be applicable to the machine on which it was generated -- for example if a compiler auto-detects the precise chipset (e.g. presence of AVX2 instructions) or hardcodes machine specific details (e.g. the username). Such information can often be lifted into the command line, e.g. by moving chipset detection into the control logic and passing it explicitly to the compiler. Alternatively, such commands can be explicitly tagged as not being suitable for cloud builds.

\textbf{Relative build directories} Often the current directory, or user's profile directory, will be accessed by commands. These directories change if a user has two working directories, or if they use different machines. We can solve this problem by having a substitution table, replacing values such as the users home directory with \texttt{\$HOME}. If not rectified, this issue reduces the reusability of cloud results, but is otherwise not harmful.

\textbf{Non-deterministic builds} If a command has non-deterministic output, then every time it runs it may generate a different result. Anything that transitively depends on that output is likely to also vary on each run. If the user temporarily disconnects from the shared storage, and runs a non-deterministic command, even if they subsequently reconnect, it is likely anything transitively depending on that command will never match again until after a clean rebuild. There are designs to solve this problem (e.g. the modification comparison mechanism from \citet{erdweg2015sound}), but the issue only reduces the effectiveness of the cloud cache, and usually occurs with intermittent network access, so can often be ignored.

\subsection{Build consistency}
\label{sec:hazards}

As stated in \citet[\S3.6]{build_systems_a_la_carte}, a build is correct provided that:

\begin{quote}
\emph{If we recompute the value of the key (...), we should get exactly the same value as we see in the final store.}
\end{quote}

% SS just trying to make it clear nothign has changed between builds; although maybe that goes without saying?
Specified in terms more applicable to our design, it means that after a build completes, an immediate subsequent rebuild should have no effect because all commands are skipped (assuming the commands given to the build system are the same). However, there are sequences of commands, where each command meets our assumptions separately (as per \S\ref{sec:skipping_unnecessary}), but the combination is problematic:

\vspace{1mm}
\begin{small}
\begin{verbatim}
echo 1 > foo.txt
echo 2 > foo.txt
\end{verbatim}
\end{small}
\vspace{1mm}

\noindent
% consistent -> correct because consistent had never been defined and I think correct is what we meant
This program writes \texttt{1} to \texttt{foo.txt}, then writes \texttt{2}. If the commands are re-executed then the first command reruns because its output changed, and after the first command reruns, now the second commands output has changed. More generally, if a build writes different values to the same file multiple times, it is not correct by the above definition, because on a rebuild both commands would re-run. But even without writing to the same file twice, it is possible to have an incorrect build:

\vspace{1mm}
\begin{small}
\begin{verbatim}
sha1sum foo.txt > bar.txt
sha1sum bar.txt > foo.txt
\end{verbatim}
\end{small}
\vspace{1mm}

\noindent
Here \texttt{sha1sum} takes the SHA1 hash of a file, first taking the SHA1 of \texttt{foo.txt} and storing it in \texttt{bar.txt}, then taking the SHA1 of \texttt{bar.txt} and storing it in \texttt{foo.txt}. The problem is that the script first reads from \texttt{foo.txt} on the first line, then writes to \texttt{foo.txt} on the second line, meaning that when the script is rerun the read of \texttt{foo.txt} will have to be repeated as its value has changed.

Writing to a file after it has already been either read or written is the only circumstance in which a build, where every individual command is well-formed (as per \S\ref{sec:skipping_unnecessary}), is incorrect. We define such a build as \emph{hazardous} using the following rules:

\begin{description}
\item[Read then write] If one command reads from a file, and a later command writes to that file, on a future build, the first command will have to be rerun because its input has changed. This behaviour is defined as a \emph{read-write hazard}.  We assume the build author ordered the commands correctly, if not the author can edit the build script.
\item[Write then write] If two commands both write to the same file, on a future build, the first will be rerun (its output has changed), which is likely to then cause the second to be rerun. This behaviour is defined as a \emph{write-write hazard}.
\end{description}

Using tracing we can detect hazards and raise errors if they occur, detecting that a build system is incorrect before unnecessary rebuilds occur. We prove that a build system with deterministic control logic, given the same input the control logic will produce the same output,  and with no hazards always results in no rebuilds in \S\ref{sec:claims}.  The presence of hazards in a build does not guarantee that a rebuild will always occur, for example if the write of a file after it is read does not change the file's value.  But, such a build system is malformed by our definition and if the write definitely can't change the output, then it should not be present.

\subsection{Explicit Parallelism}
\label{sec:explicit_parallelism}

A build script can use explicit parallelism by giving further commands to the build system before previous commands have completed. For example, the script in \S\ref{sec:monadic} has a \texttt{for} loop where the inner commands are independent and could all be given to the build system simultaneously. Such a build system with explicit parallelism must still obey the invariant that the inputs of a command must have been generated before the command is given, requiring some kind of two-way feedback that an enqueued command has completed.

Interestingly, given complete \emph{dependency} information (e.g. as available to \Make) it is possible to infer complete \emph{parallelism} information. However, the difficulty of specifying complete dependency information is the attraction of a tracing based approach to build systems.

\subsection{Implicit Parallelism (Speculation)}
\label{sec:speculation}

While explicit parallelism is useful, it imposes a burden on the build script author. Alternatively we can use implicit parallelism, where some commands are executed speculatively, before they are required by the build script, in the hope that they will be executed by the script in the future and thus skipped by the build system. Importantly, such speculation can be shown to be safe by tracking hazards, provided we introduce a new hazard \emph{speculative-write-read}, corresponding to a speculative command writing to a file that is later read by a command required by the build script (defined precisely in \S\ref{sec:hazards_formally}).

Given a script with no hazards when executed sequentially, we show in \S\ref{sec:claims}: 1) that any ordering of those commands that also has no hazards will result in an equivalent output, see Claim \ref{claim:reorder}; 2) that any parallel or interleaved execution without hazards will also be equivalent, see Claim \ref{claim:parallel}; and 3) if any additional commands are run that don't cause hazards, they can be shown to not change the results the normal build produces, see Claim \ref{claim:additional}. Finally, we prove that if a series of commands contains hazards, so will any execution that includes those required commands, see Claim \ref{claim:keep_hazards}.

As a consequence, if we can predict what commands the build script will execute next, and predict that their execution will not cause hazards, it may be worth speculatively executing them. Effective speculation requires us to predict the following pieces of data.

\textbf{Future commands} The benefit of speculatively executing commands is that they will subsequently be skipped, which only happens if the speculative command indeed occurs in the build script. The simplest way to predict future commands is to assume that they will be the same as they were last time. Given more information about the build, e.g. all the control logic as per \S\ref{sec:monadic} choice 2, it would be possible to use static analysis to refine the prediction.

\textbf{Absence of hazards} If a hazard occurs the build is no longer correct, and remediation must be taken (e.g. rerunning the build without speculation, see \S\ref{sec:recovering}). Therefore, performance can be significantly diminished if a speculative command leads to a hazard. Given knowledge of the currently running commands, and the files all commands accessed in the last run, it is possible to predict whether a hazard will occur if the access patterns do not change. If tracing made it possible to abort runs that performed hazardous accesses then speculation could be unwound without restarting, but such implementations are difficult (see \S\ref{sec:tracing}).

\textbf{Recovering from Hazards caused by speculation}
If a build using speculative execution causes a hazard, it is possible that the hazard is entirely an artefact of speculation. There are a few actions the build system could take to recover and these are discussed in section \S\ref{sec:proof:classify_hazard}.

\section{Implementing \Rattle}
\label{sec:implementation}

We have implemented the design from \S\ref{sec:design} in a build system called \Rattle. We use Haskell as the host language and to write the control logic. Copying the design for \Shake \cite{shake}, a \Rattle build script is a Haskell program that uses the \Rattle library.

\subsection{A \Rattle example}

\begin{figure}
\begin{verbatim}
import Development.Rattle
import System.FilePath
import System.FilePattern

main = rattle $ do
    let toO x = takeBaseName x <.> "o"
    cs <- liftIO $ getDirectoryFiles "." [root </> "*.c"]
    forP cs $ \c ->
        cmd "gcc -c" c
    cmd "gcc -o main.exe" (map toO cs)
\end{verbatim}
\caption{A Haskell/\Rattle version of the script from \S\ref{sec:monadic}}
\label{fig:rattle_example}
\end{figure}

\begin{figure}
\begin{verbatim}
-- The Run monad
data Run a = ... deriving (Functor, Applicative, Monad, MonadIO)
rattle :: Run a -> IO a

-- Running commands
data CmdOption = Cwd FilePath | ...
cmd :: CmdArguments args => args

-- Reading/writing files
cmdReadFile :: FilePath -> Run String
cmdWriteFile :: FilePath -> String -> Run ()

-- Parallelism
forP :: [a] -> (a -> Run b) -> Run [b]
\end{verbatim}
\caption{The \Rattle API}
\label{fig:api}
\end{figure}

A complete \Rattle script that compiles all \texttt{.c} files like \S\ref{sec:monadic} is given in Figure \ref{fig:rattle_example}, with the key API functions in Figure \ref{fig:api}. Looking at the example, we see:

\begin{itemize}
\item A \Rattle script is a Haskell program. It makes use of ordinary Haskell imports, and importantly includes \texttt{Development.Rattle}, offering the API from Figure \ref{fig:api}.
\item The \texttt{rattle} function takes a value in the \texttt{Run} monad and executes it in \texttt{IO}. The \texttt{Run} type is the \texttt{IO} monad, enriched with a \texttt{ReaderT} \cite{mtl} containing a reference to shared mutable state (e.g. what commands are in flight, where to store metadata, location of shared storage).
\item All the control logic is in Haskell and can use external libraries -- e.g. \texttt{System.FilePath} for manipulating \texttt{FilePath} values and \texttt{System.FilePattern} for directory listing. Taking the example of replacing the extension from \texttt{.c} to \texttt{.o}, we are able to abstract out this pattern as \texttt{toO} and reuse it later. Arbitrary Haskell IO can be embedded in the script using \texttt{liftIO}. All of the Haskell code is considered control logic and will be repeated in every execution.
\item Commands are given to the build system part of \Rattle using \texttt{cmd}. We have implemented \texttt{cmd} as a variadic function \cite{variadic_functions} which takes a command as a series of \texttt{String} (a series of space-separated arguments), \texttt{[String]} (a list of arguments) and \texttt{CmdOption} (command execution modifiers, e.g. to change the current directory), returning a value of type \texttt{Run ()}. The function \texttt{cmd} only returns once the command has finished executing (whether that is by actual execution, skipping, or fetching from external storage).
\item We have used \texttt{forP} in the example, as opposed to \texttt{forM}, which causes the commands to be given to \Rattle in parallel (\S\ref{sec:explicit_parallelism}). We could have equally used \texttt{forM} and relied on speculation for parallelism (\S\ref{sec:speculation}).
\end{itemize}

Looking at the functions from Figure \ref{fig:api}, there are two functions this example does not use. The \texttt{cmdWriteFile} and \texttt{cmdReadFile} functions are used to perform a read/write of the file system through Haskell code, causing hazards to arise if necessary. Apart from these functions, it is assumed that all Haskell control code only reads and writes files which are not written to by any commands.

\subsection{Alternative \Rattle wrappers}

Given the above API, combined with the choice to treat the control logic as opaque, it is possible to write wrappers that expose \Rattle in new ways. For example, to run a series of commands from a file, we can write:

\vspace{3mm}
\begin{verbatim}
main = rattle $ do
    [x] <- liftIO getArgs
    src <- readFile x
    forM_ (lines src) cmd
\end{verbatim}
\vspace{3mm}

Here we take a command line argument, read the file it points to, then run each command sequentially using \texttt{forM\_}. We use this script for our evaluation in \S\ref{sec:evaluation}.

An alternative API could be provided by opening up a socket, and allowing a \Rattle server to take command invocations through that socket. Such an API would allow writing \Rattle scripts in other languages, making use of the existing \Rattle implementation. While such an implementation should be easy, we have not yet actually implemented it.

\subsection{Specific design choices and consequences}
\label{sec:choices}

Relative to the reference design in \S\ref{sec:design} we have made a few specific design choices, mostly in the name of implementation simplicity:

\begin{itemize}
\item All of our predictions (see \S\ref{sec:speculation}) only look at the very last run. This approach is simple, and in practice, seems to be sufficient -- most build scripts are run on very similar inputs most of the time.
\item We run a command speculatively if 1) it hasn't been run so far this build; 2) was required in the last run; and 3) doesn't cause a hazard relative to both the completed commands and predicted file accesses of the currently running commands. Importantly, if we are currently running a command we have never run before, we don't speculate anything -- the build system has changed in an unknown way so we take the cautious approach.
\item We never attempt to recover after speculation (see \S\ref{sec:recovering}), simply aborting the build and restarting without any speculation.
\item We have not yet implemented speculative-read-write hazards, and equally have not seen them occur. Under normal (non-adversarial) circumstances, given our cautious approach to speculation, we consider such hazards extremely unlikely.
\item We treat command lines as black boxes, never examining them to predict their outputs. For many programs a simple analysis (e.g. looking for \texttt{-o} flags) might be predictive.
\item We use a shared drive for sharing build artefacts, but allow the use of tools such as NFS or Samba to provide remote connectivity and thus full ``cloud builds''.
\item \Rattle can go wrong if a speculated command writes to an input file, as per \S\ref{sec:proof:restart_no_speculation}. This problem hasn't occurred in practice, but dividing files into inputs and outputs would be perfectly reasonable -- typically the inputs are either checked into version control or outside the project directory, so that information is readily available.
\item It is important that traces (as stored for \S\ref{sec:skipping_unnecessary}) are only recorded to disk when we can be sure they were not poisoned (see \S\ref{sec:proof:continue}) by any hazards. That determination requires waiting for all commands which ran at the same time as the command in question to have finished.
\item We model queries for information about a file (e.g. existence or modification time) as a read for tracing purposes, thus depending on the contents of the file. For queries about the existence of a file, we rerun if the file contents changes, which may be significantly more frequent than when the file is created or deleted. For queries about modification time, we don't rerun if the modification time changes but the file contents don't, potentially not changing when we should. In practice, most modification time queries are to implement rebuilding logic, so can be safely ignored if the file contents haven't changed.
\end{itemize}


\subsection{Hash forwarding}
\label{sec:forward_hashes}

One of the problems of non-deterministic builds is that they reduce external cache hits, as described in \S\ref{sec:cloud_builds}. As a concrete example, assuming we have the build script:

\begin{verbatim}
gcc -c main.c
gcc -o main.exe main.o
\end{verbatim}

Here we first compile \texttt{main.c} to produce \texttt{main.o}, then link \texttt{main.o} to produce \texttt{main.exe}. If we assume \texttt{gcc -c} is non-deterministic (e.g. embeds the current build time), and locally compile \texttt{gcc -c main.c}, then we will never get cache hits for the linking step. To avoid this problem, assuming all \texttt{main.o} values produced by the same \texttt{main.c} can be replaced for each other, we can use the hash of \texttt{main.c} to identify the output \texttt{main.o}.

\Rattle implements a more general version of this feature with the name ``hash forwarding''. If the output of a command produces files named \texttt{x} and \texttt{x.forward}, then the hash of \texttt{x.forward} is used to identify \texttt{x} instead of the hash of \texttt{x}. To modify the above example to take advantage of hash forwarding we can write:

\begin{verbatim}
gcc -c main.c && cp main.c main.o.forward
gcc -o main.exe main.o
\end{verbatim}

We require that if the ``important'' part of \texttt{main.o} changes then the file \texttt{main.o.forward} must also change. It would be equally possible to trim non-significant whitespace or comments from \texttt{main.c} to produce a more stable hash.

The disadvantage of hash forwarding (other than the complexity) is that if the output file eliminates some information compared to the inputs, e.g. is simply the number of lines in the input, then there might be instances where two output values would be equal despite their forwarding hashes being different, causing fewer cache hits. The use of hash forwarding is very similar to the idea of deep-constructive traces from \citet[\S4.2.4]{build_systems_a_la_carte}, including the inability to benefit from unchanging outputs.


\subsection{Tracing approaches}
\label{sec:tracing}

In \S\ref{sec:assume_tracing} we assume the existence of \emph{dependency tracing} which can, after a command completes, tell us which files that command read and wrote. Unfortunately, such an API is \emph{not} part of the POSIX standard, and is not easily available on any standard platform. We aim to make \Rattle work on Linux, Mac and Windows, which requires using a variety of approaches. In this section we outline some of the approaches that can be used for tracing, along with their advantages and disadvantages.

\paragraph{Syscall tracing} On Linux, \texttt{ptrace} allows tracing every system call made, examining the arguments, and thus recording the files accessed. Moreover, by tracing  the \texttt{stat} system call even file queries can be recorded. The syscall tracking approach can be made complete, but because \emph{every} syscall must be hooked, can end up imposing high overhead. This approach is used by \Bigbro \cite{bigbro} as well as numerous other debugging and instrumentation tools.

\paragraph{Library preload} On both Linux and Mac most programs use a dynamically linked C library to make file accesses. By using \texttt{LD\_LIBRARY\_PRELOAD} it is possible to inject a different library into the program memory which intercepts the relevant C library calls, recording which files are read and written. This approach is simpler than hooking syscalls, but only works if all syscall access is made through the C library. While normally true, that isn't the case for Go programs \cite{go} (syscalls are invoked directly) or statically linked programs (the C library cannot be replaced).

While the technique works on a Mac, from Mac OS X 1.10 onwards system binaries can't be traced due to System Integrity Protection\footnote{\url{https://developer.apple.com/library/content/documentation/Security/Conceptual/System_Integrity_Protection_Guide/ConfiguringSystemIntegrityProtection/ConfiguringSystemIntegrityProtection.html}}. As an example, the C compiler is typically installed as a system binary. It is possible to disable System Integrity Protection (but not recommended by Apple), to use non-system binaries (e.g. those supplied by \Nix \cite{nix}), or to copy the system binary to a temporary directory (which works provided the binary does not afterwards invoke another system binary to do its work). The library preload mechanism is implemented by \Fsatrace \cite{fsatrace} and the copying system binaries trick on Mac is implemented by \Shake \cite{shake}.

\paragraph{File system tracing} An alternative approach is to implement a custom file system and have that report which files are accessed. One such implementation for Linux is \tracedfs \cite{tracedfs}, which is unfortunately not yet complete. Such an approach can track all accesses, but may require administrator privileges to mount a file system.

\paragraph{Custom Mac tracing} \BuildXL \cite{buildxl}\footnote{\url{https://github.com/Microsoft/BuildXL/blob/master/Documentation/Specs/Sandboxing.md\#macos-sandboxing}} uses a Mac sandbox based on KAuth combined with TrustedBSD Mandatory Access Control (MAC) to both detect which files are accessed and also block access to specific files. The approach is based on internal Mac OS X details which have been reversed engineered, some of which are deprecated and scheduled for removal.

\paragraph{Windows Kernel API hooking} On Windows it is possible to hook the Kernel API, which can be used to detect when any files are accessed. Implementing such a hook is difficult, particularly around 32bit v 64bit differences, as custom assembly language trampolines must be used. Furthermore, some antivirus products (incorrectly) detect such programs as viruses. Windows kernel hooking is available in both \Fsatrace and \Bigbro, although without support for 32bit processes that spawn 64bit processes.

\postparagraphs

\Rattle currently uses \Fsatrace for the underlying tracing, with a Haskell interface provided by \Shake. That means it uses library preloading on Linux/Mac and kernel hooking on Windows. The biggest practical limitations vary by OS:

\begin{itemize}
\item On \textbf{Linux} it can't trace into Go programs (or other programs that use system calls directly) and statically linked binaries. We plan to integrate \Bigbro as an alternative, to address these concerns.
\item On \textbf{Mac} it can't trace into system binaries called from other system binaries. We recommend using \Nix binaries if this limitation is problematic.
\item On \textbf{Windows} it can't trace 64bit programs spawned by 32bit programs. In most cases the 32bit binaries can easily be replaced by 64bit binaries. The only problem we've seen was caused by a five year-old version of \texttt{sh}, which was easily remedied with a newer version.
\end{itemize}

\noindent In practice, none of the limitations have been overly problematic in the examples we have explored.

%\subsection{Possible Tracing Enhancements}

We designed \Rattle to work with the limitations of the best cross-platform tracing easily available -- but that involves trade-offs. An enhanced, portable system would be a significnat enabler for \Rattle.
Our wishlist for tracing would include a cross-platform API, precise detection times, detection as access happens, and executing code at interception points.

%% \begin{itemize}
%% \item A standardised cross-platform API for tracing, implemented by the operating system itself, would vastly simplify the work required and guarantee robustness.
%% \item If we could detect exactly when files were opened and closed we could report fewer hazards, as currently we assume all files are open during the entire command.
%% \item If we could detect file accesses as they happened we could report hazards sooner, and avoid speculating on commands that are guaranteed to conflict with those currently running.
%% \item If we could reject file writes, we could eliminate most hazards from speculation by rejecting the speculative write that leads to a read-write or write-write hazard.
%% \item If we could redirect file writes, we could eliminate all hazards from speculation by sending such writes to a shared cache, not directly to the output files.
%% \end{itemize}

\section{Proofs}
\label{sec:proof}

The design of \Rattle relies on taking a sequence of commands and instead of running them all sequentially, instead running different commands in different orders. In this section we prove that the manipulations we perform are safe with respect to the reference semantics.

\subsection{Hazards, formally}

A hazard occurs when a Build command writes to a file that a previous command has already read from or written to.  Hazards can be classified as either: \emph{read-write hazards} or \emph{write-write hazards}.  In section \S\ref{sec:proof:classify_hazard} hazards are explained in greater detail.  For the following sections though it suffices to say that in a build being executed sequentially a
hazard occurs when a command read or writes a file and a later command writes to the same file.  In a build being executed in parallel with speculation, a hazard can occur when a command reads or writes a file and a later command writes to the same file, or if two commands running concurrently access the same file and are not both just reading the file.  Therefore, for a sequential \Rattle build to have no hazards means for every command $c$ in the build, if $c$ reads or writes a file, no later command will write to that file.  For a parallel \Rattle build to have no hazards means for every command $c$ in the build, if $c$ reads or writes a file, no later command will write to that file, and no command will concurrently read the file if it is being written, or write to the file if it is being read.

% SS seems this section should just be a brief explanation of hazards so the following claims of ``no hazards'' makes sense.

\subsection{No rebuilds}
\label{sec:proof:no_rebuild}

We prove that a build system with deterministic control logic with no hazards always results in no rebuilds.
In a build system without hazards there is at most one write to any file, which occurs before any reads of that file. We can therefore prove there are no rebuilds by showing the first command can't rebuild, and proceeding by induction.

Let us have a build of \emph{n} commands that meets the above restrictions.

Base case: Command $1$ of $1$  will not be re-run because from the definition of the claim none of its inputs has changed, and the build system's control logic is deterministic.  Because, command $1$ is not re-running, it will not write to any files.

Induction step:  Let us assume the above claim is true for a build with $n$ commands.  Let us show the claim is true for a build of $n+1$ commands.

If the first $n$ commands of the build did not run and did not write to any files, then the dependencies of command $n+1$ could not have changed during the build, and from the claim above we know they did not change before the build, therefore, command $n+1$ will not run.

\subsection{Reordered builds}
\label{sec:proof:reorder}

Given a script with no hazards when executed sequentially, we can show that any interleaving of those commands that also has no hazards will result in an equivalent output.

Proof by contradiction.

Let us assume we have a build script $A$, with no hazards.  Let us also assume there exists a build script $B$, which is an alternative ordering of the build script $A$, which also has no hazards and produces an output different from the output produced by $A$.  Here, the output of a build script is the set of files written and their hashes.

Therefore, $B$ produces a different set of file and hashes than $A$ does when running the same set of commands.

\begin{description}
\item [$B$ writes to a file $A$ does not write to. Or $A$ writes to a file $B$ does not write to]
  There must exist a cmd $c$ which when run by $A$ and $B$ writes to a different set of files.  The set of files written to by $c$ is affected by the content of the cmd, and the set of input files.  Since, the content of the cmd is the same between $A$ and $B$, the input files must differ.

  By the definition of $A$ and $B$ having no hazards, any files read by $c$ must be written to by commands that precede $c$ in the build. And because $A$ and $B$ contain the same commands, $c$'s input files must be the same regardless of build.  Therefore, a contradiction.

\item [$B$ writes to a file $f$ whose hash is $h$ and $A$ writes to the same file but its hash is $K$; $h \neq k$]
  There must exist a cmd $c$ which when run by $A$ and $B$ writes to the file $f$ and produces the different hashes, $k$ and $h$ respectively.  The files written to by $c$ is affected by the content of the cmd and the set of input files. Since $c$ is the same between $A$ and $B$, the input files must differ.

  By the definition of $A$ and $B$ having no hazards, any files read by $c$ must be written to by commands that precede $c$ in the build.  And because $A$ and $B$ contain the same commands, $c$'s input files must be the same regardless of build.  Therefore, a contradiction.

\end{description}

\subsection{Parallel commands}
\label{sec:proof:parallel}

Given a script with no hazards when executed sequentially, we can show that any parallel or interleaved execution without hazards will also be equivalent.

Proof by contradiction.

Let us assume we have a build $A$, which has no hazards.  Let us also assume there is an alternative execution of $A$, which has no hazards, but is not equivalent.
Recall that for the builds to be equivalent, they write to the same set of files and a file will have the same hash regardless of the build that wrote it.
Therefore, the parallel execution must write to a different set of files than $A$ or at least one of those files has a different hash.

\begin{description}
\item [parallel build writes to a file $A$ does not write to, or $A$ writes to a file the parallel build does not write to.]
  There must exist a cmd $c$ which when run in parallel with one or more other commands, writes to a different set of files than when run sequentially.  The set of files written to by $c$ is affected by the content of the cmd, and the files it reads.  Since, the content of the cmd is the same between the parallel and sequential builds, the input files must differ.
  By the defintion of the parallel build of $A$ having no hazards, any files read by $c$ must be written to by commands that precede $c$ in the build.  Therefore, the files written by $c$ in the parallel build will be the same as the files written by $c$ in the sequential build.  Therefore, a contradiction.


\item [parallel build writes to a file $f$ whose hash is $h$ and $A$ writes to the same file but its hash is $k$; $h \neq k$]
  There must exist a cmd $c$ which when run in parallel with one or more other commands, writes to the file $f$ and produces the hash $h$ and when run sequentially in build $A$ produces the hash $k$.
  The files written to by $c$, and their hashes, is affected by the content of the command and the files read by the command.  Since $c$ is unchanged between $A$ the sequential run of $A$ and the parallel run of $A$, the files and their hashes read by $c$ must differ.  By the definition of parallel $A$ having no hazards, any files read by $c$ must be written to by commands that precede $c$ in the build.  And because parallel $A$ and sequential $A$ contain the same commands, $c$'s input must be the same regardless of whether $A$ was run in parallel or sequentially.  Therefore, a contradiction.

  % clean up
\end{description}

\subsection{Additional commands have no effect}
\label{sec:proof:additional}

Given a script with no hazards when executed sequentially, we can show that speculating unnecessary commands will not affect the build's output. % hmm

Proof by induction.

Let us assume we have a build $A$ which has no hazards when executed sequentially.

Base case:  Build $A$ has 1 command $c$.  Let us assume we have a command $d$ that does not write to any file read or written by $c$.  By this definition of $d$, it is obvious that $d$ running before or
concurrently with $c$ as part of build $A$, will not affect the files written by $c$ and therefore will not affect the output of $A$.

Inductive case: Let us assume the above claim is true for a build with $n$ commands.  Let us show the claim is true for a build of $n+1$ commands.

Let $A$ have $n+1$ command.  From the inductive hypothesis we know the output of the first $n$ commands is unchanged.  And, we know that $c$ does not write to any file read or written by command $n+1$.  And, because the build has no hazards, all files read by $n+1$ were written to before $n+1$ ran, so the output files of $n+1$ remain unchanged as well.  Therefore, the output of $A$ remained unaffected by $c$.

\subsection{Classifying Hazards}
\label{sec:proof:classify_hazard}

% SS An explanation of the classification of hazards and a proof of the claims in 2.7

In section \S\ref{sec:speculation} we describe how \Rattle uses speculation to execute builds in parallel.  Sometimes when \Rattle speculates commands, hazards can occur that wouldn't have if the build script was executed sequentially.  In that section we mention when speculation can lead to hazards and what \Rattle can do to recover from them.  Here we offer a more precise classification of othose hazards and proofs that \Rattle's actions recovering from them does not change the outcome of the build.

Hazards are first classified as either \emph{read-write} or \emph{write-write}.  Hazards can be further classified by how and if \Rattle can recover from the hazard.

\paragraph{Non Recoverable}
A hazard is classified as \emph{non-recoverable} if it is triggered by a consistency violation in the build script.  A \emph{non-recoverable hazard} always results in the build terminating immediateley with an error.  For two examples of \emph{non-recoverable hazards} see section \ref{sec:hazards}.

\paragraph{Recoverable}
A hazard is classified as \emph{recoverable} if it is caused by \Rattle speculating a command which read a file concurrently or later written to by another command.  In this situation the speculated command likely read stale data and if re-executed would read up-to-date data.  Here is an example where \emph{cp foo.o baz.o} potentially copied the wrong version of \emph{foo.o} because it executed before \emph{gcc -c foo.c} completed.  If \emph{cp foo.o baz.o} re-executes it will read the new \emph{foo.o} produced by \emph{gcc -c foo.c}.

\begin{verbatim}
cp foo.o baz.o [speculate]
gcc -c foo.c
cp foo.o baz.o [re-execute]
\end{verbatim}

Proof that \Rattle will produce an equivalent outcome to executing the build script sequentially if it re-executes the speculative read command of a \emph{recoverable read-write hazard}.

% proving that when a recoverable read-write hazard occurs the build outcome when re-executing the
% failed read will be the same as if the build was executed sequentially

% have a build A which when executed sequentially either results in a non-recoverable hazard
% or produces an output of a set of files with certain hashes

% prove that if rattle is speculating commands and there is a recoerable read write hazard
% that it can re-execute the read command and will still produce an outcome equivalent

% aka will prodcue a non-recoverable hazard ( do we care which one?) I say no
% produce the same set of output files with the same hashes

% SS writing this about the alternative scheduler I wrote since it doesn't really apply to the
%    original one. can be removed if it isnt relevant
Let us assume we have a build $A$ which when executed sequentially produces an output $O$.
There are two possible outputs:
\begin{description}
\item [Non-recoverable hazard] Proof by contradiction.
  Let us assume \Rattle executes commands speculatively and a \emph{recoverable read-write hazard} occurs.  All \emph{read-write hazards} are made up of two commands, $c1$, the command which read the file, and $c2$, the command which wrote to the file. In this case we also know $c1$ was executed speculatively.


\item [writes to set of files $F$]
\end{description}






% Proof that Rattle can re-execute the read portion of a recoverable hazard
A recoverable hazard is made up of two commands, $c1$ and $c2$, where $c1$ reads one or more files written by $c2$, and $c2$ does not read or write to any files written by $c1$.  Additionally, $c1$ ran before or at the same time as $c2$.  Because, $c1$ does not write to any files read or written to by $c2$, executing $c1$ again after $c2$ terminates will never cause $c2$ to need to rebuild.

% State something about re-executing c1 causing new hazards.
After re-executing $c1$ there is the possibility of a new hazard occuring.
\begin{description}
\item[hazard] The new hazard will be handled in the same manner as already described.
\item[no hazard] The build continues.
\end{description}

\paragraph{Restartable} % when speculation causes a consistency violation. or a command to read incorrect data.
If \Rattle speculatively executed a command that wrote to a file that was later written to or read by another command, then incorrectness was potentially introduced into the build.  In the following example \emph{cp foo.o baz.o} was speculated, and wrote to \emph{baz.o}, then \emph{gcc -c baz.c} was executed and also wrote to \emph{baz.o}.  Normally a build script that executed these two commands would violate \Rattle's consistency properties, but in this case the consistency violation might have been avoided if \emph{cp foo.o baz.o} were not executed by the build.  Therefore, if \Rattle were to re-execute the build script and not speculate \emph{cp foo.o baz.o} then the consistency violation might not occur.

\begin{verbatim}
cp foo.o baz.o [speculate]
gcc -c baz.c
\end{verbatim}

Another possible situation is \Rattle executing something like the following:

\begin{verbatim}
cp old-foo.c foo.c [speculate]
gcc -c foo.c
\end{verbatim}

Maybe the build script originally included \emph{cp old-foo.c foo.c}, but it was removed in the most recent version of the build.  By speculating this old command \Rattle caused \emph{gcc -c foo.c} to
potentially build the wrong \emph{foo.c}.  If \Rattle were to re-execute the build and not speculate \emph{cp old-foo.c foo.c} then \emph{gcc -c foo.c} would build the correct \emph{foo.c} during the next build.  % This is a bad example since Rattle would have corrupted foo.c with the bad copy.  Need a new example or address this

%Proof that restartable hazard are handled

\section{Evaluation}
\label{sec:evaluation}

In this section we evaluate the design from \S\ref{sec:design}, specifically our implemenation from \S\ref{sec:implementation}. We show how the implementation performs on the example from \S\ref{sec:introduction} in \S\ref{sec:eval:introduction}, on microbenchmarks in \S\ref{sec:eval:overhead}, and then on real projects that currently use \Make{} -- namely FSATrace (\S\ref{sec:eval:fsatrace}), Redis (\S\ref{sec:eval:redis}), Vim (\S\ref{sec:eval:vim}) and Node (\S\ref{sec:eval:node}). For larger projects we look at both whether the dependencies are correct and the performance.

For benchmarks, the first three (\S\ref{sec:eval:introduction}, \S\ref{sec:eval:fsatrace} and \S\ref{sec:eval:redis}) were run on a 4 core Intel i7-4790 3.6Ghz CPU, with 16Gb of RAM. The remaining benchmarks were run on TODO: SARAH MACHINE SPECS HERE.

\subsection{Validating the claims from \S\ref{sec:introduction}}
\label{sec:eval:introduction}

In \S\ref{sec:introduction} we claimed that the following build script is ``just as good'' as a proper \Make script.

\begin{verbatim}
gcc -c main.c
gcc -c util.c
gcc -o main.exe main.o util.o
\end{verbatim}

There are two axes on which to measure ``just as good'' -- correctness and performance. Performance can be further broken down into how much rebuilds, how much parallelism can be acheived, and how much overhead there is.

\paragraph{Correctness} \Rattle is correct, in that the reference semantics is running all the commands, and as we have shown in \S\ref{sec:design} and \S\ref{sec:proof}, and tested for with examples, \Rattle obeys those semantics. In contrast, the \Make version may have missing dependencies which causes it not to rebuild. Examples of failure to rebuild include both if \texttt{gcc} changes, one of the system headers used by \texttt{gcc} or any headers included but not listed in the \Make script.

\paragraph{Rebuilding too much} \Rattle only rebuilds a command if some of its inputs have changed. It is possible that a command only depends on a subset of those inputs, but at the level of abstraction \Rattle works, it never rebuilds too much. As a matter of implementation, to implement cloud builders as per \ref{sec:cloud_builds}, \Rattle uses hashes of the file contents. In contrast, \Make uses the modification time, so if a file is modified, but it's contents do not change (e.g. using \texttt{touch}), \Make will rebuild but \Rattle will not. It would be possible for \Make to use hashes, if it chose to store additional metadata.

\paragraph{Parallelism} The script from \S\ref{sec:introduction} has three commands, the first two of which can run in parallel, the the third must wait for the first two to finish. \Make is given all this information by dependencies, and will always acheive as much parallelism as possible. In constrast, \Rattle has no such knowledge, so has to recover the parallelism by speculation, as per \ref{sec:speculation}. During the first execution, \Rattle has no knowledge about even which commands are coming next (as described in \S\ref{sec:monadic}), so has no choice but to execute each command serially, with less parallelism than \Make. In subsequent executions \Rattle uses speculation to always speculate on the second command (as it never has a hazard with the first), but never speculate on the third until the first two have finished (as they are known to conflict). Interestingly, sometimes \Rattle executes the third command (because it got to that point in the script), and sometimes it speculates it (because the previous two have finished) -- it is a race condition where both alternatives are equivalent. While \Rattle has less parallelism on the first execution, using shared storage for speculation traces, that can be reduced to the first execution \emph{ever}, rather than the first execution for a given user.

\paragraph{Overhead} The overhead inherent in \Rattle is greater than that of \Make as it hashes files, traces command execution, computes potential hazards to figure out speculation and writes to a shared cloud store. To measure the overhead, and prove the other claims in this section, we created a very simple pair of \texttt{main.c} and \texttt{util.c} files where \texttt{main.c} calls \texttt{printf} using a string computed by a function in \texttt{util.c}. We then measured the time to do 1) an initial build; 2) a rebuild when nothing had changed; 3) a rebuild with whitespace changes to \texttt{main.c}; 4) a rebuild with meaningful changes to \texttt{main.c}; 5) a rebuild with meaningful changes to both C files. We did all the above with 1, 2 and 3 threads, on Linux. To check speculation was happening, we modified \texttt{gcc} to sleep for 1 second before starting. The numbers are:

% RAW RESULTS
% $ rattle-benchmark intro
%   make   -j1: 3.35s 0.00s 2.19s 2.19s 3.28s
%   rattle -j1: 3.28s 0.00s 1.12s 2.16s 3.29s
%   make   -j2: 2.18s 0.00s 2.19s 2.23s 2.22s
%   rattle -j2: 3.28s 0.00s 1.12s 2.21s 2.20s
%   make   -j3: 2.20s 0.00s 2.19s 2.19s 2.19s
%   rattle -j3: 3.36s 0.00s 1.10s 2.22s 2.20s
%   make   -j4: 2.20s 0.00s 2.19s 2.18s 2.19s
%   rattle -j4: 3.28s 0.00s 1.14s 2.21s 2.20s

\vspace{3mm}
\begin{tabular}{l|r|r||r|r||r|r}
Number of threads & \multicolumn{2}{c||}1 & \multicolumn{2}{c||}2 & \multicolumn{2}{c}3 \\
Tool & \Make & \Rattle & \Make & \Rattle & \Make & \Rattle \\
\hline
1) Initial build & 3.35s & 3.28s & 2.18s & 3.28s & 2.20s & 3.36s \\
2) Nothing changed & 0.00s & 0.00s & 0.00s & 0.00s & 0.00s & 0.00s \\
3) \texttt{main.c} changed whitespace & 2.19s & 1.12s & 2.19s & 1.12s & 2.19s & 1.10s \\
4) \texttt{main.c} changed & 2.19s & 2.16s & 2.23s & 2.21s & 2.19s & 2.22s \\
5) Both C files changed & 3.28s & 3.29s & 2.22s & 2.20s & 2.19s & 2.20s \\
\end{tabular}
\vspace{3mm}

As expected, we see that in the initial build \Rattle doesn't exhibit parallelism, but \Make can (1). In constrast, \Rattle can benefit when a file changes in whitespace only and the resulting object file doesn't change, while \Make can't (3). We see 3 threads has no benefit over 2 threads, as this build contains no more parallelism opportunities. Comparing the non-sleep portion of the build, \Make and \Rattle are quite evenly matched, typically within a few milliseconds, showing low overheads -- we focus on the overheads in the next section.

\subsection{Measuring overhead}
\label{sec:eval:overhead}

In order to determine what overhead \Rattle introduces, we ran a fixed set of commands with increasingly more parts of \Rattle enabled. \Rattle command execution builds on the command execution from \Shake \cite{shake}, which in turn uses \Fsatrace for tracing and the Haskell \texttt{process} library for command execution. Therefore, we ran the commands in a clean build directory in 7 different ways:

\begin{enumerate}
\item Using \texttt{make -j1}, as a baseline.
\item Using \texttt{System.Process} from the Haskell \texttt{process} library.
\item Using \texttt{cmd} from the Haskell \texttt{shake} library \cite{shake}, which builds on top of the \texttt{process} library.
\item Using \texttt{cmd} from \texttt{shake}, but wrapping the command with \Fsatrace for file tracing.
\item Using \texttt{cmd} from \texttt{shake} with the \texttt{Traced} setting, which runs \Fsatrace and collects the results.
\item Using \Rattle with no speculation or parallelism, and not storing any results to shared storage.
\item Using \Rattle with all features turned on, including shared storage.
\end{enumerate}

To obtain a set of commands typical of building, we took the latest version of \Fsatrace\footnote{\url{https://github.com/jacereda/fsatrace/commit/41fbba17da580f81ababb32ec7e6e5fd49f11473}} and ran \texttt{make -j1}, capturing the commands that were executed. On Windows \Fsatrace runs 25 commands (21 compiles, 4 links). On Linux \Fsatrace runs 9 commands (7 compiles, 2 links). On Linux the list of commands produces write-write hazards, because it compiles some files (e.g. \texttt{shm.c}) twice, once with \texttt{-fPIC} (position independent code), and once without. However, both times it passes \texttt{-MMD} to cause \texttt{gcc} to produce \texttt{shm.d} which is used for dependencies -- we removed the \texttt{-MMD} flag as it doesn't impact the benchmark. We ran all sets of commands five times, and took the average of the three fastest, on both Windows and Linux.

% RAW RESULTS (final number is avg of fastest 3, to ignore swapping etc)
% $ rattle-benchmark micro
%   WINDOWS (with Windows Defender)
%     make: 13.14s 10.47s 11.04s 11.67s 11.86s = 11.06s
%     System.Process: 12.28s 12.41s 12.90s 13.07s 13.24s = 12.53s
%     shake.cmd: 13.59s 13.83s 13.88s 13.99s 14.08s = 13.77s
%     shake.cmd fsatrace: 16.54s 17.06s 16.55s 16.41s 16.39s = 16.45s
%     shake.cmd traced: 16.52s 16.42s 16.84s 16.69s 16.70s = 16.54s
%     rattle: 20.70s 18.51s 18.93s 18.50s 18.51s = 18.51s
%     rattle share: 18.94s 19.10s 18.74s 18.74s 18.63s = 18.70s
%   WINDOWS (no Windows Defeneder)
%     make: 12.31s 9.73s 9.98s 14.97s 10.16s = 9.96s
%     System.Process: 10.20s 10.25s 10.32s 10.40s 10.44s = 10.26s
%     shake.cmd: 10.53s 10.68s 10.62s 10.59s 10.70s = 10.58s
%     shake.cmd fsatrace: 12.68s 12.58s 12.85s 12.72s 12.80s = 12.66s
%     shake.cmd traced: 13.31s 12.99s 12.86s 13.90s 16.26s = 13.06s
%     rattle: 16.59s 14.51s 14.48s 14.82s 14.30s = 14.43s
%     rattle share: 14.56s 14.48s 14.62s 14.62s 14.56s = 14.53s
%   LINUX
%     make: 1.26s 1.22s 1.19s 1.23s 1.17s = 1.19s
%     System.Process: 1.19s 1.19s 1.22s 1.24s 1.16s = 1.18s
%     shake.cmd: 1.20s 1.18s 1.17s 1.21s 1.18s = 1.17s
%     shake.cmd fsatrace: 1.26s 1.20s 1.26s 1.25s 1.25s = 1.23s
%     shake.cmd traced: 1.25s 1.23s 1.27s 1.21s 1.23s = 1.23s
%     rattle: 1.50s 1.29s 1.25s 1.25s 1.25s = 1.25s
%     rattle share: 1.31s 1.31s 1.31s 1.26s 1.24s = 1.27s

\vspace{3mm}
\begin{tabular}{l|rrr|rrr}
Commands & \multicolumn{3}{c|}{Windows} & \multicolumn{3}{c}{Linux} \\
\hline
1) Make                      &  9.96s & 100\% &       &    1.19s & 100\% & \\
2) process                   & 10.26s & 103\% &  +3\% &    1.18s &  99\% & -1\% \\
3) \Shake                    & 10.58s & 106\% &  +3\% &    1.17s &  98\% & -1\% \\
4) \Shake + \Fsatrace        & 12.66s & 127\% & +21\% &    1.23s & 103\% & +5\% \\
5) \Shake + \texttt{Traced}  & 13.06s & 131\% &  +4\% &    1.23s & 103\% & +0\% \\
6) \Rattle                   & 14.43s & 145\% & +14\% &    1.25s & 105\% & +2\% \\
7) \Rattle + everything      & 14.53s & 146\% &  +1\% &    1.27s & 107\% & +2\% \\
\end{tabular}
\vspace{3mm}

Both Windows and Linux have three columns -- the time taken (average of five runs), that time as a percentage of the \Make run, and the delta from the row above. The results are significantly different between platforms:

\paragraph{Windows} On Windows, we see that the total overhead of \Rattle makes the execution 46\% slower. Of the parts, 21\% of the slowdown is from \Fsatrace (due to hooking Windows kernel API), with the next greatest overhead being from \Rattle itself. Of the \Rattle overhead, the greatest slowdown is caused by canonicalising filepaths. Using the default NTFS file system, Windows considers paths to be case insensitive. As a result, we observe paths like \verb"C:\windows\system32\KERNELBASE.dll", which on disk are called \verb"C:\Windows\System32\KernelBase.dll", but can also be accessed by names such as \verb"C:\Windows\System32\KERNEL~1.DLL". Unfortunately, Windows also supports case sensitive file systems, so simply using case-insensitive equality is insufficient.

On Windows, enabling the anti-virus (Windows Defender) has a significant impact on the result, increasing the \Make baseline by 11\% and the final time by 29\%. These results were collected with the anti-virus disabled.

\paragraph{Linux} In contrast, on Linux, the total overhead is only 7\%, of which nearly all (5\%) comes from the tracing.

\postparagraphs

These results show that tracing has minor but not insignificant on Linux, whereas on Windows can be a substantial performance reduction. As a consequence, we focus on the results under Linux.

\subsection{\Fsatrace}
\label{sec:eval:fsatrace}

To compare \Make and \Rattle on \Fsatrace we took the commands we extracted for \S\ref{sec:eval:overhead} and ran the build script for the 100 previous commits in turn, starting with a clean build then performing incremental builds. To make the results readable, we hid any commands where all versions were < 0.02s, resulting in 26 interesting commits. We ran with 1 to 4 threads, but omit the 2 and 3 thread case as they typically fall either on or just above the 4 thread case.

\begin{tikzpicture}
\begin{axis}[
  title={Compile time at each successive commit},
  width=\textwidth,
  height=5cm,
  ylabel={Seconds},
  ymin=0,
  xmin=0,
  xmax=15,
]
\addplot [color=cyan, mark=o] table [x expr=\coordindex, y=make1] {data/fsatrace.dat};    \addlegendentry{\Make -j1}
\addplot [color=cyan, mark=*] table [x expr=\coordindex, y=make4] {data/fsatrace.dat};    \addlegendentry{\Make -j4}
\addplot [color=purple, mark=triangle] table [x expr=\coordindex, y=rattle1] {data/fsatrace.dat};  \addlegendentry{\Rattle -j1}
\addplot [color=purple, mark=triangle*] table [x expr=\coordindex, y=rattle4] {data/fsatrace.dat};  \addlegendentry{\Rattle -j4}
\end{axis}
\end{tikzpicture}

As we can see, the first build is always > 1s for \Rattle, but \Make is able to optimise it as low as 0.33s with 4 threads. Otherwise, \Rattle and \Make are competitive -- users would struggle to see the difference. The one commit that does show some variation is commit 2, where \Make at 1 thread matches all the \Rattle builds, but \Make at more than 1 thread goes slightly faster. The cause is a speculation leading to a write-write hazard. Concretely, the command for linking \texttt{fsatrace.so} changed to include a new file \texttt{proc.o}. \Rattle starts speculating on the old link, then gets the command for the new link -- they both write to \texttt{fsatrace.so}, leading to a hazard, and causing \Rattle to restart without speculation.

% WriteWriteHazard /tmp/extra-dir-39079319872788/fsatrace.so Cmd [EchoStderr False] ["cc","-shared","src/unix/fsatraceso.os","src/emit.os","src/unix/shm.os","-o","fsatrace.so","-ldl","-lrt"] Cmd [EchoStderr False] ["cc","-shared","src/unix/fsatraceso.os","src/emit.os","src/unix/shm.os","src/unix/proc.os","-o","fsatrace.so","-ldl","-lrt"] Restartable

\subsection{Redis}
\label{sec:eval:redis}

\begin{tikzpicture}
\begin{axis}[
  title={Compile time at each successive commit},
  width=\textwidth,
  height=5cm,
  ylabel={Seconds},
  ymin=0,
  xmin=0,
  xmax=30,
]
\addplot [color=cyan, mark=o] table [x expr=\coordindex, y=make1] {data/redis.dat}; 	\addlegendentry{\Make -j1}
\addplot [color=cyan, mark=*] table [x expr=\coordindex, y=make4] {data/redis.dat}; 	\addlegendentry{\Make -j4}
\addplot [color=purple, mark=triangle] table [x expr=\coordindex, y=rattle1] {data/redis.dat}; 	\addlegendentry{\Rattle -j1}
% \addplot table [x expr=\coordindex, y=rattle4] {data/redis.dat}; 	\addlegendentry{\Rattle -j4}
\addplot [color=purple, mark=triangle*] table [x expr=\coordindex, y=rattle4_noshared] {data/redis.dat}; 	\addlegendentry{\Rattle -j4}
\end{axis}
\end{tikzpicture}

In this graph the ``no shared'' variant is \Rattle with the cloud cache disabled. For local builds, the consequence is that if a file changes, then changes back, we will have to rebuild rather than get a cache hit -- something that never happens in this benchmark, making the copying of files to a shared cloud redundant work. The Redis project is structured as recursive \Make, which is known to be problematic \cite{miller:recursive_make} and how \Rattle is able to outperform \Make. Furthermore, the Redis build system is structured so as to write sentinel values to ignore some portions of the code unless they change, requiring users to manually clean when dependencies are upgraded. We avoid that entirely by running all the dependency commands every single time.

\begin{comment}
% I reimplemented Stack in Rattle. Not sure it's useful given how much other evaluation stuff we have.
\subsection{Reimplementing Stack}

\Rattle assumes that each command is atomic - it cannot be subdivided into smaller parts. If a command is secretly two independent commands then they should usually be expressed as such so they can be individually skipped.

Compound commands: Sometimes a command will produce something that is user specific (not great for caching), but the next step will remove the user specificity (good for caching). To fix that we allow compound commands, by conjoining two commands with \texttt{\&\&}. Sometimes the sole purpose of the second command can be to strip machine-unique data from the first command.

As another example, the GHC package database has additional entries added every time a package is installed, making the output a consequence of the original file\footnote{As a consequence many build systems, including \Bazel and \Rattle, use multiple package databases with only one entry per database}.

and some memoisation operations (\texttt{memo})
\end{comment}

\subsection{vim}
\label{sec:eval:vim}

\begin{tikzpicture}
\begin{axis}[
  title={Compile time at each successive commit},
  width=\textwidth,
  height=5cm,
  ylabel={Seconds},
  ymin=0,
  xmin=0,
  xmax=39,
]

\addplot [color=cyan, mark=o] table [x expr=\coordindex, y=make1] {data/vim.dat}; 	\addlegendentry{\Make -j1}
\addplot [color=cyan, mark=*] table [x expr=\coordindex, y=make4] {data/vim.dat}; 	\addlegendentry{\Make -j4}
\addplot [color=purple, mark=triangle] table [x expr=\coordindex, y=rattle1] {data/vim.dat}; 	\addlegendentry{\Rattle -j1}
\addplot [color=purple, mark=triangle*] table [x expr=\coordindex, y=rattle4] {data/vim.dat}; 	\addlegendentry{\Rattle -j4}
\end{axis}
\end{tikzpicture}

Vim is a popular text editor whose source code can be found on \href{https://github.com/vim/vim}{github}.  The majority of the source code is Vim script and C, and it is built with Make.  To build Make on a Unix based system one can merely call \Make from the top-level project directory.  To compare the original \Make based build to a new \Rattle build, we generated Vim over a series of commits checked-out from github with both the original build system and the adapted \Rattle build.  The \Rattle build script was created by recording every command executed by the original build, except for those invoking \Make.  Additionally, some commands which may have been separate shell commands in the original build were combined to create one command if they caused either \emph{read-write} or \emph{write-write} hazards.  A \Rattle build script was generated for each commit where the orginal build system changed or files were added to or removed from the project.  This was necessary since the \Rattle build was created from a literal list of hard-coded shell commands rather than the ideal Haskell program with variables.

% SS todo add list of the commits

Comparing building Vim \cite{} with \Make and with \Rattle for commits \emph{21109272f} to \emph{7cc96923c}.  For two commits \Make and \Rattle did noticeably different work; note, both of these commits were build incrementally, \emph{21109272f} was built from scratch by both build systems and all subsequent commits were built incrementally.

% todo compare sequential
% SS todo add timing data

\subsection{node}
\label{sec:eval:node}

% Brief into to the project
Node.js is a JavaScript runtime built on Chrome's V8 Javascript engine and can be found on \href{``https://github.com/nodejs/node''}{github}.  The project is largely written in JavaScript, C++, Python, and C, and is built using Make and a meta-build tool called \emph{Generate Your Projects (GYP)}.  To build Node.js from source, a user first runs \emph{./configure} which runs a python script that configures the build and runs GYP.  GYP generates the majority of the Makefiles used to build the project.

GYP takes a series of \emph{.gyp} files and produces the project build from those.  It puts all generated files in the project's \emph{out} directory and generates a separate \emph{*.mk} Makefile for each target, all of which are included by the top-level Makefile.  The generated top-level Makefile includes a \emph{do\_cmd} function which first checks if a command changed since the last time it was run, and if it has runs the command.  Additionally, it writes the command run to a generated dependency file, and if the command run was a \emph{g++} command that produced a file with dependencies, the dependencies in that file are cleaned up and written to the generated dependency file.

These dependency files are included in the top-level \emph{out/Makefile}, presumably so on future runs Make can consider the recorded dependencies of these object files as well as check whether the command has changed since it was previously run.  Most of these targets include the dependency \emph{FORCE\_DO\_CMD} which is an empty phony target and serves the purpose of forcing the build to check everytime whether the command has changed since it was previously run. Each time a command is re-run, these dependency files are re-generated.

% Explain the makefile insanity and all of this dependency stuff they do
% Still trying to make coherent since of what this gyp is doing, so here is my current understanding

The build both seems to be using a form of tracing to keep track of accurate dependencies as well as tracking whether or not the build itself has changed.  \Rattle would make it unnecessary for the build to explicitely do this, since it internally traces and tracks the depencencies of all commands run, and doesn't need to worry about when a build changes.

% The generator/make.py

% what is this
% # Helper that is non-empty when a prerequisite changes.
% # Normally make does this implicitly, but we force rules to always run
% # so we can check their command lines.
% #   $? -- new prerequisites
% #   $| -- order-only dependencies
% prereq_changed = $(filter-out FORCE_DO_CMD,$(filter-out $|,$?))

% # do_cmd: run a command via the above cmd_foo names, if necessary.
% # Should always run for a given target to handle command-line changes.

% Explain the intermediate files and how on a rebuild with no changes stuff still happens

When building certain targets, \emph{.intermediate} files are created and at the end of the build deleted.  These \emph{.intermediate} commands appear to be for doing code generation.  So, the generated files depend on the \emph{.intermediate} target which runs the code generator.  As with the object file targets mentioned, a \emph{.intermediate.d} dependency file is generated for the \emph{.intermediate} target.  This dependency file contains the \emph{.intermediate} code generation command run, so the build can check whether the command changed since it was last run.

These \emph{.intermediate} files are specified as prereqs of the special \Make target {.INTERMEDIATE}, which means that when one does not exist \Make won't bother updating it unless one of its prerequisites has changed.  Because these \emph{.intermediate} targets have \emph{DO\_/FORCE\_CMD} as a prereq they always run.  This would enable the build to check if the command changed since it was last run, but this only works if the Makefile includes the dependency file storing that information.  The dependency files of these \emph{.intermediate} targets, which run everytime, are not included in the Makefile.  Because of this the Makefile thinks the command has never run before, causing them to run everytime.  Even if the Makefile is modified to include the dependency files of these \emph{.intermediate} targets, the targets still execute the recipe which does the file generation even though the command has not changed. % SS comment on this further when I fully understand why the commands are registering as changed...

  % Other things like parallelism and rattle rebuilding less

It appears the authors of this build are attempting to keep track of accurate dependencies by recording them every time a command runs, they are also tracking the build by recording every command run and checking whether a command or its prerequisites have changed when deciding whether to run it.  The scheme to accomplish this is in my opinion quite complicated and confusing and relies on various hacks to work around \Make and the fact that this isn't a full-fledged programming language.  \Rattle accomplishes both of these things the build authors are trying to accomplish with \Make without forcing them to work so hard.

To compare the \Make based build to a \Rattle version, we built Node.js over a series of commits checked-out from github with both the original build system and the adapted \Rattle build.  The \Rattle build was created by recording every command executed by the original build, except for the commands creating dependency files and those invoking \Make.  A new such \Rattle build script was generated for any commit where the original build system changed or files were added to or removed from the project.
% data


% SS add the differences in commits
% 1. make list of commits run

% in order from latest to oldest.
% d80c40047b 0fe810168b 22724894c9 ab9e89439e cb210110e9 d10927b687
% 023ecbccc8 be6596352b [54c1a09202] 470511ae78 25c3f7c61a 13fe56bbbb
% [43fb6ffef7] a171314003 dd4c62eabe abe6a2e3d1 9225939528 d4c81be4a0
% 38aa31554c 1d9511127c d227d22bd9 5cf789e554 d65e6a5017 24e81d7c5a
% 2cd9892425 [64161f2a86] 0f8941962d 2170259940 32f63fcf0e 2462a2c5d7
% b851d7b986 70c32a6d19 3d456b1868 f2ec64fbcf 59a1981a22 [7b7e7bd185]
% 78743f8e39 a5d4a397d6
% [] indicate a commit where build appears to have changed. so need new rattle build too


% So, pretty much everything needs to be re-run. awesome.

% 5. Record here
% 6. Run on tank? with various threads; ask sam


% Comment from the Makefile
%# .buildstamp needs $(NODE_EXE) but cannot depend on it
%# directly because it calls make recursively.  The parent make cannot know
%# if the subprocess touched anything so it pessimistically assumes that
%# .buildstamp is out of date and need a rebuild.
%# Just goes to show that recursive make really is harmful...
%# TODO(bnoordhuis) Force rebuild after gyp update.


\subsection{tmux}

\begin{tikzpicture}
\begin{axis}[
  title=Compile time in seconds,
  width=\textwidth,
  height=7cm,
  xlabel={Time at each successive commit},
  ylabel={Seconds},
  ymin=0,
  xmin=0,
]

\addplot table [x expr=\coordindex, y=make1] {data/tmux.dat}; 	\addlegendentry{\Make -j1}
\addplot table [x expr=\coordindex, y=make4] {data/tmux.dat}; 	\addlegendentry{\Make -j4}
\addplot table [x expr=\coordindex, y=rattle1] {data/tmux.dat}; 	\addlegendentry{\Rattle -j1}
\addplot table [x expr=\coordindex, y=rattle4_noshared] {data/tmux.dat}; 	\addlegendentry{\Rattle -j4 (no shared)}
\end{axis}
\end{tikzpicture}


Tmux is a terminal multiplexer whose sourcecode is available on  \href{``https://github.com/vim/vim''}{github}.  It can be built from source using \emph{autoconf}, \emph{automake}, \emph{pkg-config}, and \emph{make}.  To build Tmux from source grabbed from github one does the following:

\begin{verbatim}
sh autogen.sh
./configure
make
\end{verbatim}

To compare a \Rattle based Tmux build to the \Make based one, we first ran \emph{sh autogen.sh} and \emph{./configure}, and then used the resulting Makefile(s) to generate the Rattle build scripts.  A new Rattle build script was generated for any commit where the build appeared to have changed or a file was added or deleted from the repository.

Tmux was built over a series of 40 commits with both \Make and \Rattle.  The commands run by each build system for each command were then compared.  For single threaded \Rattle and \Make, for 33 out of 40 commits, \Rattle and \Make ran the same commands, for 4 commits, \Make ran 1 more non-trivial command than \Rattle, but for 3 commits \Rattle ran substantially more cmds than \Make did.

These 3 commits were ones where new \Rattle build scripts were generated because something about the build or project had changed.  For two of these commits the version number of Tmux had changed which caused the textual contents of majority of the commands to change.  \Rattle viewed these as new commans and therefore ran them.  For the 3rd commit, \Rattle ran 125 commands more than \Make did because a command flag changed, which caused the textual contents of the majority of the commands to change, so \Rattle viewed them as new commands.

% todo add details about parallel version and hazards
% todo timing data



% 47174f51
% 150 cmds in file - 10 for make
% 140 cmds run by rattle not run by make
% how many did rattle run period?
% there were 141 cmds in the script.
% so what command did make run too?

%4822130b was a new build script

% 685eb381 also a new build script

% I wonder if I don't do sh autogen and configure if that wouldn't happen?



% For 1 thread

% commits where did same things: close enough; ignoring commands like rm tmux

% obf153da 0c6c8c4e 0eb7b547 19d5f4a0 22e9cf04 24cd726d 32be954b 37919a6b 3e701309
% 400750bb 43b36752 4694afb  470cba35 54553903 60ab7144 61b075a2 6c28d0dd 6f0241e6
% 74b42407 7cdf5ee9 7f3feb18 8457f54e 8b22da69 9900ccd0 a01c9ffc a4d8437b ba542e42
% bc36700d c391d50c c915cfc7 cdf13837 e9b12943 fdbc1116


% commits with differences
% 47174f51 (commit where make did automake stuff and rattle re-ran everything because a version number changed)
% 4822130b
% 685eb381

% 7eada28f: make did ./etc/ylwrap ....
% ed16f51e: make did ./etc/ylwrap ...
% ee3d3db3: make built tmux.o
% f3ea318a: make did ./etc/ylwrap ...


% commits
% ed16f51e 61b075a2 e9b12943 3e701309 8457f54e a01c9ffc cdf13837 74b42407 0eb7b547 f3ea318a 7cdf5ee9 ee3d3db3 685eb381 60ab7144 7eada28f 7f3feb18 8b22da69 bc36700d 32be954b 6f0241e6 19d5f4a0 43b36752 0bf153da 4822130b 47174f51 c915cfc7 54553903 400750bb 470cba35 a4d8437b 6c28d0dd 24cd726d 9900ccd0 c391d50c 0c6c8c4e fdbc1116 37919a6b 22e9cf04 ba542e42 4694afb

% \subsection{openssl}

\section{Related work}
\label{sec:related}

The vast majority of existing build systems are \emph{backward build systems} -- they start at the final target, and recursively determine the dependencies required for that target. In contrast, \Rattle is a \emph{forward build system}---the script executes sequentially in the order given by the user.

\subsection{Comparison to forward build systems}

The idea of treating a script as a build system, omitting commands that have not changed, was pioneered by \Memoize \cite{memoize} and popularised by \Fabricate \cite{fabricate}. In both cases the build script was written in Python, where cheap logic was specified in Python and commands were run in a traced environment. If a traced command hadn't changed inputs  since it was last run, then it was skipped. We focus on \Fabricate, which came later and offered more features. \Fabricate uses \texttt{strace} on Linux, and on Windows uses either file access times (which are either disabled or very low resolution on modern Windows installations) or a proprietary and unavailable \texttt{tracker} program. Parallelism can be annotated explicitly, but often is omitted.

These systems did not seem to have much adoption -- in both cases the original sources are no longer available, and knowledge of them survives only as GitHub copies. \Rattle differs from an engineering perspective by the tracing mechanisms available (see \S\ref{sec:tracing}) and the availability of cloud build (see \S\ref{sec:cloud_builds}) -- both of which are likely just a consequence of being developed a decade later. \Rattle extends these systems in a more fundamental way with the notion of hazards, which both allows the detection of bad build scripts, and allows for speculation -- overcoming the main disadvantage of earlier forward build systems. Stated alternatively, \Rattle takes the delightfully simple approach of these build systems, and tries a more sophisticated execution strategy.

Recently there have been three other implementations of forward build systems we are aware of.

\begin{enumerate}
\item \Shake \cite{shake} provides a forward mode implemented in terms of a backwards build system. The approach is similar to the \Fabricate design, offering skipping of repeated commands and explicit parallelism. In addition, \Shake allows caching custom functions as though they were commands, relying on the explicit dependency tracking functions such as \texttt{need} already built into \Shake. The forward mode has been adopted by a few projects, notably a library for generating static websites/blogs. The tracing features are provided by a combination of \Shake and \Fsatrace, and are the ones we reuse in \Rattle.
\item \Fac \cite{fac} is based on the \Bigbro tracing library. Commands are given in a custom file format as a static list of commands (i.e. no monadic expressive power as per \S\ref{sec:monadic}), but may optionally include a subset of their inputs or outputs. The commands are not given in any order, but the specified inputs/outputs are used to form a dependency order which \Fac uses. If the specified inputs/outputs are insufficient to give a working order, then \Fac will fail but record the \emph{actual} dependencies which will be used next time -- a build with no dependencies can usually be made to work by running \Fac multiple times.
\item \Stroll \cite{stroll} takes a static set of commands, without either a valid sequence or any input/output information, and keeps running commands until they have all succeeded. As a consequence, \Stroll may run the same command multiple times, using tracing to figure out what might have changed to turn a previous failure into a future success. \Stroll also reuses the tracing features of \Shake and \Fsatrace.
\end{enumerate}

There are significantly fewer forward build systems than backwards build systems, but the interesting dimension starting to emerge is how an ordering is specified. The three current alternatives are the user specifies a valid order (\Fabricate and \Rattle), the user specifies partial dependencies which are used to calculate an order (\Fac) or the user specifies no ordering and search is used (\Stroll and some aspects of \Fac).

\subsection{Comparison to backward build systems}
\label{sec:remote_execution}

The design space of backward build systems is discussed in \cite{build_systems_a_la_carte}. In that paper it is notable that forward build systems do not naturally fit into the design space, lacking the features that a build system requires. We feel that omission points to an interesting gap in the landscape of build systems. We think that it is likely forward build systems could be characterised similarly, but that we have yet to develop the necessary variety of forward build systems to do so. There are two dimensions used to classify backward build systems:

\textbf{Ordering} Looking at the features of \Rattle, the ordering is a sequential list, representing an excessively strict ordering given by the user. The use of speculation is an attempt to weaken that ordering into one that is less linear and more precise.

\textbf{Rebuild} For rebuilding, \Rattle looks a lot like the constructive trace model -- traces are made, stored in a cloud, and available for future use. The one wrinkle is that a trace may be later invalidated if it turns out a hazard occurred (see \S\ref{sec:choices}).
% In particular, the correspondence to constructive traces illuminates the consequences of moving to a deep constructive trace model (see \S\ref{sec:forward_hashes}) -- it solves non-determinism at the cost of losing unchanging builds.

\postparagraphs

\noindent There are three features present in some backward build systems that are particularly relevant to \Rattle:

\textbf{Sandboxing} Some backward build systems (e.g. \Bazel \cite{bazel}) run processes in a sandbox, where access to files which weren't declared as dependencies are blocked -- ensuring dependencies are always sufficient. A consequence is that it can be harder to write a \Bazel build script, requiring users to declare dependencies like \texttt{gcc} and system headers that are often overlooked. The sandbox doesn't prevent the reverse problem of too many dependencies.

\textbf{Remote Execution} Some build systems (e.g. \Bazel and \BuildXL \cite{buildxl}) allow running commands on a remote machine, usually with a much higher degree of parallelism than is available on the users machine. If \Rattle was able to leverage remote execution then speculative commands could be used to fill up the cloud cache, and \emph{not} cause local writes to disk, eliminating all speculative hazards -- a very attractive property. Remote execution in \Bazel sends all required files along with the command, but since \Rattle doesn't know the files accessed in advance, that model is infeasible. Remote execution in \BuildXL sends the files it thinks the command will need, augments the file system to block if other files are accessed, and sends requests for additional files back to the originating machine -- which would fit nicely with \Rattle.

\textbf{Hazards} Some build systems detect when certain types of hazard occur. For example, \Pluto \cite{erdweg2015sound} builds are constructed from builders and it is a requirement that no two builders generate the same file, and if they do the build is aborted. Similarly, if two commands in a \Rattle build write to the same file a write-write hazard occurs and the build is terminated. If a command in a \Rattle build writes to a file after another command has already read it a read-write hazard occurs. Analogously in \Pluto, if a builder requires a file generated by another builder, then the builder which generated the file must be required first and if it is not the build is aborted.

% This related work isn't relevant, and doesn't say anything interesting. It feels like a forced way of including Pluto.
%
% \textbf{Fine-grained dependencies} Pluto \cite{erdweg2015sound} is another backward build system that aims to
% provide fine-grained dependencies and optimal incremental rebuilding.  It supports dynamic
% dependencies, allows users to specify how they would like to depend on a file, and track build
% rule definitions so a build will re-run correctly when changed.  Rattle in contrast doesn't need
% to explicitly track whether a build script changes, and because all dependencies are implicitly
% tracked \Rattle does not provide the ability for a build script author to specify how a command
% should depend on any file.

\subsection{Analysis of existing build systems}

We aren't the first to observe that existing build systems often have incorrect dependencies.  \citet{bezemer2017empirical} performed an analysis of the missing dependencies in \Make build scripts, finding over \emph{1.2 million unspecified dependencies} among four projects. To detect missing dependencies, \citet{detecting_incorrect_build_rules} introduced a concept called \emph{build fuzzing}, finding race-conditions and errors in 30 projects. It has also been shown that build maintenance requires as much as a 27\% overhead on software development \cite{build_maintenance}, a substantial proportion of which is devoted to dependency management. Our anecdotes from \S\ref{sec:evaluation} all reinforce these messages.

% SS TODO maybe mention something about a project like node trying to generate dependencies within make;
\subsection{Speculation}

Speculation is used extensively in many other areas of computer science, from processors to distributed systems. If an action is taken before it is known to be required, that can increase performance, but undesired side-effects can occur. Most speculating systems attempt to block the side-effects from happening, or roll them back if they do.

The most common use of speculation in computer science is the CPU -- \citet{swanson_cpu_speculation} found that 93\% of useful CPU instructions were evaluated speculatively. CPUs use hazards to detect incorrect speculation, with similar types of read/write, write/write and write/read hazards \cite{patterson_cpu_design} -- our terminology is inspired by their approaches. For CPUs many solutions to hazards are available, e.g. stalling the pipeline if a hazard is detected in advance or the Tomasulo algorithm \cite{tomasulo}. \Rattle also stalls the pipeline (stops speculating) if it detects potential hazards, although does so with incomplete information, unlike a CPU. The Tomasulo algorithm involves writing results into temporary locations and the moving them over afterwards -- use of remote execution (\S\ref{sec:remote_execution}) might be a way to apply similar ideas to build systems.

Looking towards software systems, \citet{welc2005safe} showed how to add speculation to Java programs, by marking certain parts of the program as worth speculating with a future. Similar to our work, they wanted Java with speculation to respect the semantics of the sequential version of the program, which required two main techniques. First, all data accesses to shared state are tracked and recorded, and if a dependency violation occurs, the offending code is restarted.  Second, shared state is versioned using a copy-on-write invariant to ensure threads write to their own copies, preventing a future from seeing its continuation's writes.

Thread level speculation \cite{steffan1998potential} is used by compilers to automatically parallelise programs, often by executing multiple iterations of a loop body simultaneously. As before, the goal is to maintain the semantics of single-threaded execution. Techniques commonly involve buffering speculative writes \cite{steffan2000scalable} and ensuring that a read reflects the speculative writes of threads that logically precede it.

Speculation has also been investigated for distributed systems. \citet{nightingale2005speculative} showed that adding speculation to distributed file systems such as NFS can make some benchmarks over 10 times faster, by allowing multiple file system operations to occur concurrently. A model allowing more distributed speculation, even in the presence of message passing between speculated distributed processes, is presented by \citet{tapus2006distributed}. Both these pieces of work involve modifying the Linux kernel with a custom file system to implement roll backs transparently.

All these approaches rely on the ability to trap writes, either placing them in a buffer and applying them later or rolling them back. Unfortunately, such facilities, while desirable, are currently difficult to achieve in portable cross-platform abstractions (\S\ref{sec:tracing}). We have used the ideas underlying speculative execution, but if the necessary trapping/rollback facilities became available in future, it's possible we could follow the approaches more directly.

\section{Conclusion and future work}
\label{sec:conclusion}

In this paper we present a build system that can take a sequence of actions and treat them as a build script. From the user perspective, they get most of the benefits of a conventional build system (incrementality, parallelism) but with lower cost (less time thinking about dependencies, only needing to supply a valid ordering). Comparing \Rattle to other build systems, e.g. \Make, it is fair to say that \Rattle presents a simpler user interface (no dependencies), but a more complex implementation model.

Our evaluation in \S\ref{sec:evaluation} shows that for some popular real-world projects, switching to \Rattle would bring about simplicity and correctness benefits, with negligible performance cost. The two places where builds aren't roughly equivalent to \Make are the very first build (which could be solved with a global shared cache) and when speculation leads to a hazard. There are various approaches to improving speculation, including giving \Rattle a list of commands that should not be speculated (which can be a perfect list for non-monadic builds), or giving \Rattle a subset of commands' inputs/outputs (like \Fac does). It is also possible to have better recovery strategies from speculation errors.

Our evaluation focuses on projects whose build times are measured in seconds or minutes, not hours. It is as yet unclear whether similar benefits could be achieved on larger code bases, and whether the \Rattle approach of ``any valid ordering'' is easy to describe compositionally for large projects.  But, it does seem clear that existing projects built with \Make seek some form of automatic dependency detection, with most projects using some form of \texttt{gcc} dependency generation manually wired into the build system.

Our next steps are scaling \Rattle and incorporating feedback from actual users.

% SS probably say something else here but not sure at moment.

\paragraph{Acknowledgements} We'd like to thank Jorge Acereda and David Roundy for their work on \Fsatrace and \Bigbro respectively.



\bibliography{paper}

\end{document}
