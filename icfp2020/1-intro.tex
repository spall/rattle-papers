\section{Introduction}
\label{sec:introduction}

Build systems \citep{build_systems_a_la_carte} describe \emph{commands to run} and \emph{dependencies to respect}. For example, a simple \Make build system might look like:

\begin{verbatim}
main.o: main.c
    gcc -c main.c
util.o: util.c
    gcc -c util.c
main.exe: main.o util.o
    gcc -o main.exe main.o util.o
\end{verbatim}

It contains three rules - taking a look at the first it says \texttt{main.o} depends on \texttt{main.c}, and is produced by running \texttt{gcc -c main.c}. But, what if we just wrote a shell script? We'd end up with:

\begin{verbatim}
gcc -c main.c
gcc -c util.c
gcc -o main.exe main.o util.o
\end{verbatim}

That's shorter, simpler and easier to follow. Instead of declaring dependencies, we've merely given one valid ordering of the commands (we could equally have put \texttt{gcc -c util.c} first). There are two main downsides: 1) everything will always rebuild even if it's dependencies haven't changed; 2) while \Make could run the two \texttt{gcc -c} commands in parallel, they will now be run sequentially. But at the same time, we've fixed an inadvertent bug -- these commands all depend on the undeclared dependency \texttt{gcc}, and potentially whatever header files are imported by \texttt{main.c} and \texttt{util.c}. Furthermore, as the files \texttt{main.c} and \texttt{util.c} evolve, and their dependencies change (by adding additional \texttt{\#include} directives), the shell script remains correct, while the \Make script \emph{must} always be consistent (or builds will become stagnant).

In this paper we introduce a build system \Rattle\footnote{\url{https://github.com/ndmitchell/rattle}}, which can take the above shell script, and gain the benefits of the \Make script -- skipping those dependencies that haven't changed and running it in parallel. Furthermore, it can source compilation results from a shared cloud store. It performs this feat by \emph{tracing} which files are read/written by each command and using this information to offer a build-like experience.

This paper makes the following contributions:

\begin{itemize}
\item In \S\ref{sec:rattle} we describe how tracing can turn a shell script into a build system, while skipping dependencies and using a cloud store. The major inovation is to explicitly model hazards \S\ref{sec:hazards}, which enables speculative execution \S\ref{sec:speculation}. To make this transformation requires \emph{dependency tracing}, which we describe in \S\ref{sec:tracing}.
\item The alternative interpretations of a script described in \S\ref{sec:rattle} are only safe because of our underlying theory of hazards, which we prove in \S\ref{sec:proof}.
\item In \S\ref{sec:evaluation} we describe how we converted existing \Make scripts into \Rattle scripts, and show the performance characteristics and bugs uncovered.
\item In \S\ref{sec:related} we describe related work, in particular the \Fabricate system which pioneered the way.
\item Finally, in \S\ref{sec:conclusion} we conclude and describe future work.
\end{itemize}
