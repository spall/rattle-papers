\section{Conclusion and future work}
\label{sec:conclusion}

In this paper we present a build system that can take a sequence of actions and treat them as a build script. From the user perspective, they get most of the benefits of a conventional build system (incrementality, parallelism) but with lower cost (less time thinking about dependencies, only needing to supply a valid ordering). Comparing \Rattle to other build systems, e.g. \Make, it is fair to say that \Rattle presents a simpler interface (no dependencies), but a more complex implementation model.

Our evaluation shows that for some popular real-world projects switching to \Rattle would bring about performance and correctness benefits (excluding the very first build). Our evaluation focuses on projects who build times are measured in seconds or minutes, not hours. It is as yet unclear whether similar benefits could be acheived on larger code bases, and whether the \Rattle approach of ``any valid ordering'' is easy to describe compositionally for large projects. Scaling \Rattle and actual users are our next steps.

\paragraph{Acknowledgements} We'd like to thank Jorge Acereda and David Roundy for their work on \Fsatrace and \libbigbro respectively.
