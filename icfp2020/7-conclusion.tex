\section{Conclusion and future work}
\label{sec:conclusion}

Persuade people to use it!

Rattle build systems are simple to write - just write commands to build your project. There is only one "Rattle" specific thing, and that's running a command, giving it a tiny API. Rattle build systems can delegate to other build systems quite happily -- if a dependency is built with `make`, just shell out to `make` -- no need to recreate the dependencies with Rattle. You don't need to think about dependencies, and because you aren't writing dependencies, there's no way to get them wrong.

The disadvantages of Rattle are that only system commands are skipped, so the zero build time might not be as quick as it could. Rattle also has to guess at useful commands to get parallelism, and that might go wrong.

Compare Rattle to Shake. The interface to Rattle is basically "run this command", providing a vastly simpler API. In contast, Shake has a lot of API to learn and use.

It works well in the small to medium sized builds, but its less compositional than backward build systems, hard to do remote execution and requires more effort to get less parallelism.
