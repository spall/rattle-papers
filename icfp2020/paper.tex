% DEADLINE: Tue 3 Mar 2020
% LIMITS: 25 pages, bibliography does not count
% APPROACH: Double blind
% CFP: https://icfp20.sigplan.org/track/icfp-2020-papers#Call-for-Papers

\documentclass[acmsmall]{acmart}

\bibliographystyle{ACM-Reference-Format}
\citestyle{acmauthoryear}

\usepackage{xspace}


\begin{document}

\newcommand{\Make}{\textsc{Make}\xspace}
\newcommand{\Rattle}{\textsc{Rattle}\xspace}
\newcommand{\Fabricate}{\textsc{Fabricate}\xspace}
\newcommand{\Bazel}{\textsc{Bazel}\xspace}
\newcommand{\Buck}{\textsc{Buck}\xspace}
\newcommand{\Shake}{\textsc{Shake}\xspace}
\newcommand{\libbigbro}{\textsc{libbigbro}\xspace}
\newcommand{\Fsatrace}{\textsc{Fsatrace}\xspace}
\newcommand{\tracedfs}{\textsc{Traced-Fs}\xspace}
\newcommand{\BuildXL}{\textsc{BuildXL}\xspace}
\newcommand{\Nix}{\textsc{Nix}\xspace}
\newcommand{\Memoize}{\textsc{Memoize}\xspace}


\title{Build Systems with Perfect Dependencies}

\author{Author name}
\affiliation{
  \institution{Institution name}
  \country{Country name}
}
\email{email@address}


\begin{abstract}
Users of build systems have to write down the dependencies of each step - but they get it wrong.
Most build systems have both \emph{too few dependencies} (leading to stagnant build outputs) and \emph{too many dependencies} (leading to excessive rebuilds and reduced parallelism).
In this paper we outline a build system where dependencies are \emph{not specified}, but instead \emph{captured by tracing execution}.
The consequence is that dependencies are always correct, build systems are easier to specify, but speculative execution is required to recover parallelism.
\end{abstract}

%% 2012 ACM Computing Classification System (CSS) concepts
%% Generate at 'http://dl.acm.org/ccs/ccs.cfm'.
\begin{CCSXML}
<ccs2012>
<concept>
<concept_id>10011007</concept_id>
 <concept_desc>Software and its engineering</concept_desc>
<concept_significance>500</concept_significance>
</concept>
<concept>
<concept_id>10002950</concept_id>
 <concept_desc>Mathematics of computing</concept_desc>
<concept_significance>300</concept_significance>
</concept>
</ccs2012>
\end{CCSXML}
\ccsdesc[500]{Software and its engineering}
\ccsdesc[300]{Mathematics of computing}
%% End of generated code
\keywords{build systems, functional programming, algorithms}

\maketitle

\section{Introduction}
\label{sec:introduction}

Build scripts \cite{build_systems_a_la_carte} describe \emph{commands to run} and \emph{dependencies to respect}. For example, using the \Make build system \cite{make}, a build script might look like:

\vspace{3mm}
\begin{verbatim}
main.o: main.c
    gcc -c main.c
util.o: util.c
    gcc -c util.c
main.exe: main.o util.o
    gcc -o main.exe main.o util.o
\end{verbatim}
\vspace{3mm}

This script contains three rules. Looking at the first rule, it says \texttt{main.o} depends on \texttt{main.c}, and is produced by running \texttt{gcc -c main.c}. But what if we copied the commands into a shell script? We get:

\vspace{3mm}
\begin{verbatim}
gcc -c main.c
gcc -c util.c
gcc -o main.exe main.o util.o
\end{verbatim}
\vspace{3mm}

That's shorter, simpler and easier to follow. Instead of declaring the outputs and dependencies of each command, we've merely given one valid ordering of the commands (we could equally have put \texttt{gcc -c util.c} first). There are two main downsides: 1) everything will always rebuild even if it's dependencies haven't changed; 2) the commands will run sequentially, while \Make can run the two \texttt{gcc -c} commands in parallel. But there are some benefits. We've fixed an inadvertent bug -- these commands depend on the undeclared dependency \texttt{gcc}, and potentially whatever header files are used by \texttt{main.c} and \texttt{util.c}. Furthermore, as the files \texttt{main.c} and \texttt{util.c} evolve, and their dependencies change (by changing the \texttt{\#include} directives), the shell script remains correct, while the \Make script \emph{must} be kept consistent (or builds will become stagnant).

\subsection{Overview}

% SS I changed script -> build; not sure if this is correct or wrong?
In this paper we show how to take the above shell script and gain the benefits of a \Make build (\S\ref{sec:design}). Firstly, we can skip those commands whose dependencies haven't changed by \emph{tracing} which files they read and write (\S\ref{sec:skipping_unnecessary}). Secondly, we can run some commands in parallel, using \emph{speculation} to guess which future commands won't interfere with things already running (\S\ref{sec:speculation}). The key to speculation is a robust model of what ``interfering'' means -- we call a problematic interference a \emph{hazard}, which we define in \S\ref{sec:hazards} and use to prove important properties about the safety of speculation in \S\ref{sec:proof}.

We have implemented these techniques in a build system called \Rattle\footnote{\url{https://github.com/author_name_omitted/rattle}}, introduced in \S\ref{sec:implementation}, which embeds commands in a Haskell script. A key part of the implementation is the ability to trace commands, using techniques we describe in \S\ref{sec:tracing}. To evaluate our claims, and properly understand the subtlties of our design, we converted existing \Make scripts into \Rattle scripts, and discuss the performance characteristics and \Make script bugs uncovered in \S\ref{sec:evaluation}. Our design can be considered a successor to the \Memoize build system \cite{memoize}, and we compare the differences between that and other related work in \S\ref{sec:related}. Finally, in \S\ref{sec:conclusion} we conclude and describe future work.

\section{Rattle}
\label{sec:rattle}

We want to take a list of commands, and turn it into a build system. To do that, we assume the existence of a \emph{tracing oracle} which can, after a command completes, tell us which files that command read and which files it wrote -- we discuss the implementation and limitations of the tracing oracle in \S\ref{sec:tracing}. In this section we start by describing the most simple system possible, and progressively add features to obtain the benefits of a full build system.

\subsection{Executing commands}

In it's simplest variant, a build script can be a list of shell commands, like in \S\ref{sec:introduction}. To execute such commands correctly, it is sufficient to run each command sequentially in the order they were given. We consider this the reference semantics, and want any optimised/cached implementation to give the same results. For now, we assume commands are deterministic, but discuss how to weaken that assumption in \S\ref{sec:determinism}. We also assume that each command is atomic - it cannot be subdivided into smaller parts.

\subsection{Monadic builds}
\label{sec:monadic}

A build system which can only execute a static list of commands is restricted in its expressive power. Taking the build system from \S\ref{sec:introduction} - it would be better to compile and link \emph{all} \texttt{*.c} files -- not just those listed in the script. A more plausible script might be:

\begin{verbatim}
FILES=$(ls *.c)
for FILE in $FILES; do
    gcc -c $FILE
done
gcc -o main.exe $FILES{/.c/.o/}
\end{verbatim}

\begin{figure}
\begin{verbatim}
import Development.Rattle

main = runRattle $ do
    -- TODO: Convert
    FILES=$(ls *.c)
    for FILE in $FILES; do
        gcc -c $FILE
    done
    gcc -o main.exe $FILES{/.c/.o/}
\end{verbatim}
\caption{A Haskell/\Rattle version of the script from \S\ref{sec:monadic}}
\label{fig:monadic}
\end{figure}

Now we have a curious mixture of build system commands (e.g. \texttt{ls}, \texttt{gcc}), some control logic (e.g. \texttt{for}) and simple manipulation (e.g. changing file extension). The way we cope with this is to consider the build system as a series of commands, where the future commands may depend on the results of previous commands. These are glued together with ``cheap'' functions like the control logic and simple manipulations. We take the approach that the cheap commands are fixed overhead, run on every build, and not cached or parallelised in any way. If any of these cheap manipulations becomes expensive, they can be replaced by a command, which will then be cached and parallelised.

To implement \Rattle we treat the script as a sequence of commands with no observation on which commands are coming next. As a consequence, even if you are using a simple static script, and then manually edit it, the \Rattle build remains correct -- it has no knowledge that you manually edited the script, or if it was instead conditional on something it didn't observe.

\subsection{Skipping unnecessary commands}

When \Rattle runs a command, it records the files it reads and writes, and their hashes at that point. If you subsequently run the same command, and the inputs and outputs haven't changed (same hashes), it can be skipped. This is mostly true -- you get old but correct values. If you are running \texttt{date} it won't work, and if you are running \texttt{gcc} it will have an old compilation time. But it's good enough.

Given a command, if you know what files it reads and writes, and none of those files have changed, you can skip running it providing that if it were deterministic it wouldn't be harmful. As an example, if the compiler produces a different object file each time, but it would be OK with not doing that, then you're fine. In practice this means that any command that isn't relied on to produce fresh entropy (the time, a GUID, a random number) is fine to skip subsequent times. Those commands that do produce fresh entropy are not well suited to a build system anyway, so aren't typically used in build systems.

This got solved by fabricate. You can use a system access tracer to watch what a process reads/writes, and have a database storing the command, previous reads/writes, and then skip it if the values haven't changed. This step is not complex.

\subsection{Cloud builds}

If \Rattle matches on the files read, but not on the files written, but a previous run had the same files read/written and cached the written files, you can copy them over. If that store is in a cloud, you end up with cloud build functionality. It's pretty simple. Moreover, whether a command uses a cloud oracle or runs locally, that's not observable to the rest of the system, so we ignore it for proofs.

Given knowledge of the reads/writes of a command, it's not too hard to store them in a cloud, and satisfy commands not by rerunning them, but by copying their result from a shared cache. While this works beautifully in theory, in practice it leads to at least three separate problems which we solve.

Relative build dirs

Often the current directory, or users profile directory, will be accessed by commands. These change either if the user has two working directories, or if they use different machines. We solve this by having a substitution table.

Compound commands

Sometimes a command will produce something that is user specific, but the next step will make it not user specific. To fix that we add compound commands. You can do it with a shell script, using `pipeline`, or with TemplateHaskell blocks.

Non-deterministic builds

We solve this by having `.rattle.hash` files which we substitute for reading.

\subsection{Build consistency}

However, it a build writes a file \emph{twice} in a run, or reads it then writes to it, the result will be that a subsequent rebuild will have stuff to do, as it won't end at a quiescent state. To detect that, we introduce the concept of hazards.

For a \Make build system to be stable, it must be the case that after a build, a rebuild will not execute any other commands. It's easy to construct examples of sequences of commands that violate this property, for example:

\begin{verbatim}
gcc -c foo.c
echo x >> foo.c
\end{verbatim}

This script compiles \texttt{foo.c}, then appends the character \texttt{x} to \texttt{foo.c}. Each time around, the input to \texttt{gcc} will have changed, necessitating a recompile. We define such a build as \emph{hazardous}, because it violates one of our consistency rules:

\begin{description}
\item[read-write hazard] When one command reads from a file, and a subsequent command writes to that file. On a future build, the first command will have to be rerun.
\item[write-write hazard] When two commands both write to the same file. On a future build, the first will be rerun (it's output has changed), which is likely to then cause the second to be rerun.
\end{description}

We assume that if a single command both reads and writes to the same file (as \texttt{echo x >> foo.c} does), then within that command they were correctly sequenced.

Using the tracing we spot hazards and raise errors if they occur. We proove that a build system with deterministic control logic with no hazards always results in no rebuilds in \S\ref{sec:proof:no_rebuild}. In a build system without hazards there is at most one write to any file, which occurs before any reads of that file. We can therefore prove there are no rebuilds by showing the first command can't rebuild, and proceeding by induction.

\subsection{Parallelism}

Given a script with no hazards when executed sequentially, we can show that any interleaving of those commands that also has no hazards will result in an equivalent output in \S\ref{sec:proof:reorder}. Moreover, any parallel execution without hazards will remain consistent, see \S\ref{sec:proof:parallel}. And even further, if we execute any additional commands that don't cause hazards, they can be shown to have no impact on the rest of the build, see \S\ref{sec:proof:additional}.

As a consequence of the above, it is important that we check for hazards, but are otherwise free to run things in parallel. There are two ways we allow parallelism.

\subsubsection{Explicit Parallelism}

In the Haskell API for \Rattle there is a parallel combinator \texttt{forP}. Replacing \texttt{forM} with \texttt{forP} in Figure \ref{fig:monadic} causes the commands to be given to \Rattle in parallel. As a consequence, they can be executed in parallel. The use of explicit parallelism is convenient when replacing a loop, but harder when expressing that three executables can be built in parallel, but that a single shared object common to all of them must complete before starting, so it is difficult to provide \emph{complete} parallelism annotations.

Interestingly, given complete \emph{dependency} information (e.g. as available to \Make) it is possible to infer complete \emph{parallelism} information. However, the difficult of writing such information is the attraction to a tracing approach in the first place.

\subsubsection{Implicit Parallelism}

The other source of parallelism is implicit, using speculation. If we can predict what commands are coming up next, and predict that their execution will not cause hazards, then we could speculatively execute them. Such predictions can be made by simply recording the last known execution and using that. If speculation suggests that running a command would be beneficial there are two possible approaches to take.

Firstly, we can execute the command remotely, or in a sandboxed manner - ensuring all writes do \emph{not} end up on the file system, but are recorded to the cloud cache. In such a mode we are filling up the coud cache speculative, with the hope that when the future command does arrive it can be satisfied from the cache. However, running remotely requires syncronising the files across, or using a sycronise on demand approach (CITE Microsoft Remote Execution talk). Running with a sandbox which intercepts writes is not an easily available cross-platform feature.

Alternatively, we can execute the command locally, which is what \Rattle does. However, if the execution leads to a hazard, it is possible that the hazard is entirely an artefact of speculation. One simplistic approach is to simply rerun without speculation if speculation leads to a hazard. A more refined approach is to determine whether classification may have impacted the hazard, which can be categorised as per \S\ref{sec:proof:classify_hazard}, and either raise the error immediately (if the speculation was not at fault) or selectively eliminte a subset of commands from speculation (if speculation was at fault).

\section{Tracing}
\label{sec:tracing}

In \S\ref{sec:assume_tracing} we assume the existence of \emph{dependency tracing} which can, after a command completes, tell us which files that command read and which files it wrote. Unfortunately, such an API is \emph{not} part of the POSIX standard, and is not easily available on any standard platform. We aim to make \Rattle work on Linux, Mac and Windows, which requires using a variety of approaches.

\subsection{Tracing approaches}

In this section we outline some of the approaches that can be used for tracing, their advantages and disadvantages:

\begin{description}
\item[Syscall tracing] On Linux you can trace every system call made, examine it's arguments, and thus record which files are accessed. Moreover, you can tell which have information about them queried (via the \texttt{stat} system call). The syscall tracking approach can be made complete, but because \emph{every} syscall must be hooked, can end up being a bit slow. This approach is used by \libbigbro \cite{libbigbro}.
\item[Library preload] On both Linux and Mac most programs refer to a dynamically linked C library for their file accesses. By using \texttt{LD\_LIBRARY\_PRELOAD} it is possible to inject another dynamic library into the program memory which intercepts the relevant C library calls, recording which files are read and written. This approach is simpler than hooking syscalls, but only works if all access to syscalls is made through the C library. For programs written in Go \cite{go}, the design is that the syscalls are invoked directly, meaning that nothing is observed by tracing the C library. For statically linked programs, the location of the C library is not available, and similarly they will not be traced. While the technique works on a Mac, from Mac OS X 1.10 or higher system binaries can't be traced due to System Integrity Protection\footnote{\url{https://developer.apple.com/library/content/documentation/Security/Conceptual/System_Integrity_Protection_Guide/ConfiguringSystemIntegrityProtection/ConfiguringSystemIntegrityProtection.html}}. As an example, the C compiler is typically installed as a system binary. It is possible to disable System Integrity Protection (but not recommended by Apple), to use non-system binaries (e.g. those supplied by \Nix \cite{nix} work well), or to copy the system binary to a temporary directory (which works provided the binary does not afterwards invoke another system binary to do its work). The library preload mechanism is implemented by \Fsatrace \cite{fsatrace} and the copying system binaries trick is implemented by \Shake \cite{shake}.
\item[File system tracing] An alternative approach on Linux would be to implement a file system that reported which files were accessed. We are aware of one such implementation, \tracedfs \cite{tracedfs}, which is unfortunately not yet complete. Such an approach would track all accesses, but may require privileges to mount a file system.
\item[Custom Mac tracing] \BuildXL \cite{buildxl}\footnote{\url{https://github.com/Microsoft/BuildXL/blob/master/Documentation/Specs/Sandboxing.md\#macos-sandboxing}} uses a Mac sandbox based on KAuth combined with TrustedBSD Mandatory Access Control (MAC) to both detect which files are accessed and also block access to specific files. The approach is based on internal Mac OS X detils which have been reversed engineered, some of which are deprecated and scheduled for removal.
\item[Windows Kernel API hooking] On Windows it is possible to hook the Kernel API, which can be used to detect when files are accessed. Hooking the kernel detects everything, but is not trivial. In particular, 32bit v 64bit differences are problematic, as Windows trampolines between the two and the hooks must cope with the differences using assembly code. Furthermore, some antivirus products are more likely to (incorrectly) detect such programs as viruses. The Windows kernel hooking is available in both \Fsatrace and \libbigbro, although without the 32bit calling 64bit assembly code.
\end{description}

\Rattle currently uses \Fsatrace as wrapped by \Shake for tracing. That means it uses library preloading on Linux/Mac and kernel hooking on Windows. The biggest limitations in practice vary by OS:

\begin{itemize}
\item On \textbf{Linux} it can't trace into Go programs and statically linked binaries. We are planning to integrate \libbigbro as an alternative, to address these concerns.
\item On \textbf{Mac} it can't trace into system binaries called from other system binaries. We recommend using \Nix binaries if this limitation is problematic.
\item On \textbf{Windows} it can't trace into 64bit programs invoked by 32bit programs. In most cases the 32bit binaries can easily be replaced by 64bit binaries. The only instance we've seen thus far was an outdated version of \texttt{sh} one of the authors of this paper downloaded over five years ago, which was easily remedied with a newer version.
\end{itemize}

However, in practice, none of the limitations have been overly problematic in the examples we have explored.

\subsection{Possible tracing enhancements}

Using the approaches listed above we are able to detect when files are opened for read or write access. After the command has finished, we then report which files were accessed and in what mode. That restricted interface has driven the design of \Rattle to ensure that it is possible to implement \Rattle robustly in realistic situations. However, there are places where a more refined interface would be useful.

Tracing file accesses is the most important file system information, but there is other information about a file a command may query -- for example the existence of a file or its modification time and size. We model these as reads of a file, with all the associated read/write hazard detection. % SS I am confused by the following line; the bit about being a proxy for incremental building is confusing. 
We don't rerun if a file has changed modification time as we suspect it likely that modification time is used as a proxy for incremental building, which \Rattle already does. By virtue of \Rattle storing the hash and using it to determine if a file has changed, a size change is going to cause a rebuild. For the existence test, as a consequence of treating it as a read, \Rattle will rebuild slightly too much since the dependency is now on the entire contents rather than just the existence.  \Rattle also has to deal with files that are read but don't exist.

% SS elaborate what it means for rattle to deal with files that don't exist.  I'm assuming this just mean doing things like trying to get modtime/hash for files that don't exist etc.


The current mechanisms provide information about which files were accessed at the end. If they could interactively stream back which files were read, it would lead to better guesses in speculation (see \S\ref{sec:speculation}) about what files had been accessed by running system commands. With the tracing mechanisms available, there is no reason they couldn't report on which files had been read at any point via a socket, but such engineering work has not been done.

The approaches only report whether a file was accessed or not -- not at what time the access occurred. With a precise time it would be possible to have a more refined notion of hazard when commands are run in parallel. However, assuming commands are relatively short and that generally commands should not be relying on race conditions to avoid hazards, we consider the current approach quite reasonable. The enhancement could be implemented if desired.

It would be nice if the accesses could be intercepted and denied if there was a hazard violation, but that would require two way communication from the tracing to \Rattle. In many of the contexts being called (e.g. kernel tracing or syscall hooking) such communication is problematic, so while useful, we don't have it.

Finally, it would be useful to know the purpose of a file system access, but that is generally not available to the tracing system. As an example, many programs create, use and then delete temporary files. If multiple programs are run in a single \Rattle build then it is not unreasonable (and in fact, somewhat common) to see them access the \emph{same} temporary file, but distinct instances of it, which should not be reported as a hazard despite multiple writes. We solve such issues by ignoring reads and writes to the temporary directory.

\subsection{Tracing standardisation}

The tracing pieces of \Rattle have been some of the most difficult engineering pieces to develop. We suggest that such features should be standardised and made available in a cross-platform way.

\section{Proofs}
\label{sec:proof}

\subsection{Hazards, formally}

A hazard occurs when a Build command writes to a file that a previous command has already read from or written to.  So, hazards can be classified as either: \emph{read-write hazards} or \emph{write-write hazards}.  

\subsection{No rebuilds}
\label{sec:proof:no_rebuild}

We prove that a build system with deterministic control logic with no hazards always results in no rebuilds.
In a build system without hazards there is at most one write to any file, which occurs before any reads of that file. We can therefore prove there are no rebuilds by showing the first command can't rebuild, and proceeding by induction.

Let us have a build of \emph{n} commands that meets the above restrictions.  

Base case: Command $1$ of $1$  will not be re-run because from the definition of the claim none of its inputs has changed, and the build system's control logic is deterministic.  Because, command $1$ is not re-running, it will not write to any files.  

Induction step:  Let us assume the above claim is true for a build with $n$ commands.  Let us show the claim is true for a build of $n+1$ commands.

If the first $n$ commands of the build did not run and did not write to any files, then the dependencies of command $n+1$ could not have changed during the build, and from the claim above we know they did not change before the build, therefore, command $n+1$ will not run.

\subsection{Reordered builds}
\label{sec:proof:reorder}

Given a script with no hazards when executed sequentially, we can show that any interleaving of those commands that also has no hazards will result in an equivalent output.

Proof by contradiction.

Let us assume we have a build script $A$, with no hazards.  Let us also assume there exists a build script $B$, which is an alternative ordering of the build script $A$, which also has no hazards and produces an output different from the output produced by $A$.  Here, the output of a build script is the set of files written and their hashes.

Therefore, $B$ produces a different set of file and hashes than $A$ does when running the same set of commands.

\begin{description}
\item [$B$ writes to a file $A$ does not write to. Or $A$ writes to a file $B$ does not write to]
  There must exist a cmd $c$ which when run by $A$ and $B$ writes to a different set of files.  The set of files written to by $c$ is affected by the content of the cmd, and the set of input files.  Since, the content of the cmd is the same between $A$ and $B$, the input files must differ.

  By the definition of $A$ and $B$ having no hazards, any files read by $c$ must be written to by commands that precede $c$ in the build. And because $A$ and $B$ contain the same commands, $c$'s input files must be the same regardless of build.  Therefore, a contradiction.

\item [$B$ writes to a file $f$ whose hash is $h$ and $A$ writes to the same file but its hash is $K$; $h \neq k$]
  There must exist a cmd $c$ which when run by $A$ and $B$ writes to the file $f$ and produces the different hashes, $k$ and $h$ respectively.  The files written to by $c$ is affected by the content of the cmd and the set of input files. Since $c$ is the same between $A$ and $B$, the input files must differ.

  By the definition of $A$ and $B$ having no hazards, any files read by $c$ must be written to by commands that precede $c$ in the build.  And because $A$ and $B$ containt he same commands, $c$'s input files must be the same regardless of build.  Therefore, a contradiction.
  
\end{description}

\subsection{Parallel commands}
\label{sec:proof:parallel}


\subsection{Additional commands have no effect}
\label{sec:proof:additional}

\subsection{Classifying Hazards}
\label{sec:proof:classify_hazard}

Hazards are first classified as either \emph{read-write} or \emph{write-write}.  The command execution that led to the hazard classifies it further.

\paragraph{Non Recoverable}
Commands in the build script when executed sequentially violate the consistency rules in \ref{sec:hazards}.  A non-recoverable hazard results in \Rattle terminating the build immediately with an error.  See the two examples from \ref{sec:hazards}.

\paragraph{Recoverable}
Speculatively executed command(s) read file(s) which were later written to during the build.  If \Rattle speculatively executed a command which read from a file which was later written to, that command
probably read stale data.  \Rattle can re-execute the speculatively executed command with no loss of correctness.  A proof of this claim is provided in \ref{}.

\begin{verbatim}
cp foo.o baz.o [speculate]
gcc -c foo.c
cp foo.o baz.o [re-execute]
\end{verbatim}

\paragraph{Restartable} % when speculation causes a consistency violation. or a command to read incorrect data.
If \Rattle speculatively executed a command that wrote to a file that was later written to or read by another command, then incorrectness was potentially introduced into the build.  In the following example \emph{cp foo.o baz.o} was speculated, and wrote to \emph{baz.o}, then \emph{gcc -c baz.c} was executed and also wrote to \emph{baz.o}.  Normally a build script that executed these two commands would violate \Rattle's consistency properties, but in this case the consistency violation might have been avoided if \emph{cp foo.o baz.o} were not executed by the build.  Therefore, if \Rattle were to re-execute the build script and not speculate \emph{cp foo.o baz.o} then the consistency violation might not occur.

\begin{verbatim}
cp foo.o baz.o [speculate]
gcc -c baz.c
\end{verbatim}

Another possible situation is \Rattle executing something like the following:

\begin{verbatim}
cp old-foo.c foo.c [speculate]
gcc -c foo.c
\end{verbatim}

Maybe the build script originally included \emph{cp old-foo.c foo.c}, but it was removed in the most recent version of the build.  By speculating this old command \Rattle caused \emph{gcc -c foo.c} to
potentially build the wrong \emph{foo.c}.  If \Rattle were to re-execute the build and not speculate \emph{cp old-foo.c foo.c} then \emph{gcc -c foo.c} would build the correct \emph{foo.c} during the next build.  % This is a bad example since Rattle would have corrupted foo.c with the bad copy.  Need a new example or address this

\section{Evaluation}
\label{sec:evaluation}

In this section we evaluate the design from \S\ref{sec:design}, specifically our implemenation from \S\ref{sec:implementation}. We show how the implementation performs on the example from \S\ref{sec:introduction} in \S\ref{sec:eval:introduction}, on microbenchmarks in \S\ref{sec:eval:overhead}, and then on real projects that currently use \Make{} -- namely FSATrace (\S\ref{sec:eval:fsatrace}), Redis (\S\ref{sec:eval:redis}), Vim (\S\ref{sec:eval:vim}) and Node (\S\ref{sec:eval:node}). For larger projects we look at both whether the dependencies are correct and the performance.

For benchmarks, the first three (\S\ref{sec:eval:introduction}, \S\ref{sec:eval:fsatrace} and \S\ref{sec:eval:redis}) were run on a 4 core Intel i7-4790 3.6Ghz CPU, with 16Gb of RAM. The remaining benchmarks were run on TODO: SARAH MACHINE SPECS HERE.

\subsection{Validating the claims from \S\ref{sec:introduction}}
\label{sec:eval:introduction}

In \S\ref{sec:introduction} we claimed that the following build script is ``just as good'' as a proper \Make script.

\begin{verbatim}
gcc -c main.c
gcc -c util.c
gcc -o main.exe main.o util.o
\end{verbatim}

There are two axes on which to measure ``just as good'' -- correctness and performance. Performance can be further broken down into how much rebuilds, how much parallelism can be acheived, and how much overhead there is.

\paragraph{Correctness} \Rattle is correct, in that the reference semantics is running all the commands, and as we have shown in \S\ref{sec:design} and \S\ref{sec:proof}, and tested for with examples, \Rattle obeys those semantics. In contrast, the \Make version may have missing dependencies which causes it not to rebuild. Examples of failure to rebuild include both if \texttt{gcc} changes, one of the system headers used by \texttt{gcc} or any headers included but not listed in the \Make script.

\paragraph{Rebuilding too much} \Rattle only rebuilds a command if some of its inputs have changed. It is possible that a command only depends on a subset of those inputs, but at the level of abstraction \Rattle works, it never rebuilds too much. As a matter of implementation, to implement cloud builders as per \ref{sec:cloud_builds}, \Rattle uses hashes of the file contents. In contrast, \Make uses the modification time, so if a file is modified, but it's contents do not change (e.g. using \texttt{touch}), \Make will rebuild but \Rattle will not. It would be possible for \Make to use hashes, if it chose to store additional metadata.

\paragraph{Parallelism} The script from \S\ref{sec:introduction} has three commands, the first two of which can run in parallel, the the third must wait for the first two to finish. \Make is given all this information by dependencies, and will always acheive as much parallelism as possible. In constrast, \Rattle has no such knowledge, so has to recover the parallelism by speculation, as per \ref{sec:speculation}. During the first execution, \Rattle has no knowledge about even which commands are coming next (as described in \S\ref{sec:monadic}), so has no choice but to execute each command serially, with less parallelism than \Make. In subsequent executions \Rattle uses speculation to always speculate on the second command (as it never has a hazard with the first), but never speculate on the third until the first two have finished (as they are known to conflict). Interestingly, sometimes \Rattle executes the third command (because it got to that point in the script), and sometimes it speculates it (because the previous two have finished) -- it is a race condition where both alternatives are equivalent. While \Rattle has less parallelism on the first execution, using shared storage for speculation traces, that can be reduced to the first execution \emph{ever}, rather than the first execution for a given user.

\paragraph{Overhead} The overhead inherent in \Rattle is greater than that of \Make as it hashes files, traces command execution, computes potential hazards to figure out speculation and writes to a shared cloud store. To measure the overhead, and prove the other claims in this section, we created a very simple pair of \texttt{main.c} and \texttt{util.c} files where \texttt{main.c} calls \texttt{printf} using a string computed by a function in \texttt{util.c}. We then measured the time to do 1) an initial build; 2) a rebuild when nothing had changed; 3) a rebuild with whitespace changes to \texttt{main.c}; 4) a rebuild with meaningful changes to \texttt{main.c}; 5) a rebuild with meaningful changes to both C files. We did all the above with 1, 2 and 3 threads, on Linux. To check speculation was happening, we modified \texttt{gcc} to sleep for 1 second before starting. The numbers are:

% RAW RESULTS
% $ rattle-benchmark intro
%   make   -j1: 3.35s 0.00s 2.19s 2.19s 3.28s
%   rattle -j1: 3.28s 0.00s 1.12s 2.16s 3.29s
%   make   -j2: 2.18s 0.00s 2.19s 2.23s 2.22s
%   rattle -j2: 3.28s 0.00s 1.12s 2.21s 2.20s
%   make   -j3: 2.20s 0.00s 2.19s 2.19s 2.19s
%   rattle -j3: 3.36s 0.00s 1.10s 2.22s 2.20s
%   make   -j4: 2.20s 0.00s 2.19s 2.18s 2.19s
%   rattle -j4: 3.28s 0.00s 1.14s 2.21s 2.20s

\vspace{3mm}
\begin{tabular}{l|r|r||r|r||r|r}
Number of threads & \multicolumn{2}{c||}1 & \multicolumn{2}{c||}2 & \multicolumn{2}{c}3 \\
Tool & \Make & \Rattle & \Make & \Rattle & \Make & \Rattle \\
\hline
1) Initial build & 3.35s & 3.28s & 2.18s & 3.28s & 2.20s & 3.36s \\
2) Nothing changed & 0.00s & 0.00s & 0.00s & 0.00s & 0.00s & 0.00s \\
3) \texttt{main.c} changed whitespace & 2.19s & 1.12s & 2.19s & 1.12s & 2.19s & 1.10s \\
4) \texttt{main.c} changed & 2.19s & 2.16s & 2.23s & 2.21s & 2.19s & 2.22s \\
5) Both C files changed & 3.28s & 3.29s & 2.22s & 2.20s & 2.19s & 2.20s \\
\end{tabular}
\vspace{3mm}

As expected, we see that in the initial build \Rattle doesn't exhibit parallelism, but \Make can (1). In constrast, \Rattle can benefit when a file changes in whitespace only and the resulting object file doesn't change, while \Make can't (3). We see 3 threads has no benefit over 2 threads, as this build contains no more parallelism opportunities. Comparing the non-sleep portion of the build, \Make and \Rattle are quite evenly matched, typically within a few milliseconds, showing low overheads -- we focus on the overheads in the next section.

\subsection{Measuring overhead}
\label{sec:eval:overhead}

In order to determine what overhead \Rattle introduces, we ran a fixed set of commands with increasingly more parts of \Rattle enabled. \Rattle command execution builds on the command execution from \Shake \cite{shake}, which in turn uses \Fsatrace for tracing and the Haskell \texttt{process} library for command execution. Therefore, we ran the commands in a clean build directory in 7 different ways:

\begin{enumerate}
\item Using \texttt{make -j1}, as a baseline.
\item Using \texttt{System.Process} from the Haskell \texttt{process} library.
\item Using \texttt{cmd} from the Haskell \texttt{shake} library \cite{shake}, which builds on top of the \texttt{process} library.
\item Using \texttt{cmd} from \texttt{shake}, but wrapping the command with \Fsatrace for file tracing.
\item Using \texttt{cmd} from \texttt{shake} with the \texttt{Traced} setting, which runs \Fsatrace and collects the results.
\item Using \Rattle with no speculation or parallelism, and not storing any results to shared storage.
\item Using \Rattle with all features turned on, including shared storage.
\end{enumerate}

To obtain a set of commands typical of building, we took the latest version of \Fsatrace\footnote{\url{https://github.com/jacereda/fsatrace/commit/41fbba17da580f81ababb32ec7e6e5fd49f11473}} and ran \texttt{make -j1}, capturing the commands that were executed. On Windows \Fsatrace runs 25 commands (21 compiles, 4 links). On Linux \Fsatrace runs 9 commands (7 compiles, 2 links). On Linux the list of commands produces write-write hazards, because it compiles some files (e.g. \texttt{shm.c}) twice, once with \texttt{-fPIC} (position independent code), and once without. However, both times it passes \texttt{-MMD} to cause \texttt{gcc} to produce \texttt{shm.d} which is used for dependencies -- we removed the \texttt{-MMD} flag as it doesn't impact the benchmark. We ran all sets of commands five times, and took the average of the three fastest, on both Windows and Linux.

% RAW RESULTS (final number is avg of fastest 3, to ignore swapping etc)
% $ rattle-benchmark micro
%   WINDOWS (with Windows Defender)
%     make: 13.14s 10.47s 11.04s 11.67s 11.86s = 11.06s
%     System.Process: 12.28s 12.41s 12.90s 13.07s 13.24s = 12.53s
%     shake.cmd: 13.59s 13.83s 13.88s 13.99s 14.08s = 13.77s
%     shake.cmd fsatrace: 16.54s 17.06s 16.55s 16.41s 16.39s = 16.45s
%     shake.cmd traced: 16.52s 16.42s 16.84s 16.69s 16.70s = 16.54s
%     rattle: 20.70s 18.51s 18.93s 18.50s 18.51s = 18.51s
%     rattle share: 18.94s 19.10s 18.74s 18.74s 18.63s = 18.70s
%   WINDOWS (no Windows Defeneder)
%     make: 12.31s 9.73s 9.98s 14.97s 10.16s = 9.96s
%     System.Process: 10.20s 10.25s 10.32s 10.40s 10.44s = 10.26s
%     shake.cmd: 10.53s 10.68s 10.62s 10.59s 10.70s = 10.58s
%     shake.cmd fsatrace: 12.68s 12.58s 12.85s 12.72s 12.80s = 12.66s
%     shake.cmd traced: 13.31s 12.99s 12.86s 13.90s 16.26s = 13.06s
%     rattle: 16.59s 14.51s 14.48s 14.82s 14.30s = 14.43s
%     rattle share: 14.56s 14.48s 14.62s 14.62s 14.56s = 14.53s
%   LINUX
%     make: 1.26s 1.22s 1.19s 1.23s 1.17s = 1.19s
%     System.Process: 1.19s 1.19s 1.22s 1.24s 1.16s = 1.18s
%     shake.cmd: 1.20s 1.18s 1.17s 1.21s 1.18s = 1.17s
%     shake.cmd fsatrace: 1.26s 1.20s 1.26s 1.25s 1.25s = 1.23s
%     shake.cmd traced: 1.25s 1.23s 1.27s 1.21s 1.23s = 1.23s
%     rattle: 1.50s 1.29s 1.25s 1.25s 1.25s = 1.25s
%     rattle share: 1.31s 1.31s 1.31s 1.26s 1.24s = 1.27s

\vspace{3mm}
\begin{tabular}{l|rrr|rrr}
Commands & \multicolumn{3}{c|}{Windows} & \multicolumn{3}{c}{Linux} \\
\hline
1) Make                      &  9.96s & 100\% &       &    1.19s & 100\% & \\
2) process                   & 10.26s & 103\% &  +3\% &    1.18s &  99\% & -1\% \\
3) \Shake                    & 10.58s & 106\% &  +3\% &    1.17s &  98\% & -1\% \\
4) \Shake + \Fsatrace        & 12.66s & 127\% & +21\% &    1.23s & 103\% & +5\% \\
5) \Shake + \texttt{Traced}  & 13.06s & 131\% &  +4\% &    1.23s & 103\% & +0\% \\
6) \Rattle                   & 14.43s & 145\% & +14\% &    1.25s & 105\% & +2\% \\
7) \Rattle + everything      & 14.53s & 146\% &  +1\% &    1.27s & 107\% & +2\% \\
\end{tabular}
\vspace{3mm}

Both Windows and Linux have three columns -- the time taken (average of five runs), that time as a percentage of the \Make run, and the delta from the row above. The results are significantly different between platforms:

\paragraph{Windows} On Windows, we see that the total overhead of \Rattle makes the execution 46\% slower. Of the parts, 21\% of the slowdown is from \Fsatrace (due to hooking Windows kernel API), with the next greatest overhead being from \Rattle itself. Of the \Rattle overhead, the greatest slowdown is caused by canonicalising filepaths. Using the default NTFS file system, Windows considers paths to be case insensitive. As a result, we observe paths like \verb"C:\windows\system32\KERNELBASE.dll", which on disk are called \verb"C:\Windows\System32\KernelBase.dll", but can also be accessed by names such as \verb"C:\Windows\System32\KERNEL~1.DLL". Unfortunately, Windows also supports case sensitive file systems, so simply using case-insensitive equality is insufficient.

On Windows, enabling the anti-virus (Windows Defender) has a significant impact on the result, increasing the \Make baseline by 11\% and the final time by 29\%. These results were collected with the anti-virus disabled.

\paragraph{Linux} In contrast, on Linux, the total overhead is only 7\%, of which nearly all (5\%) comes from the tracing.

\postparagraphs

These results show that tracing has minor but not insignificant on Linux, whereas on Windows can be a substantial performance reduction. As a consequence, we focus on the results under Linux.

\subsection{\Fsatrace}
\label{sec:eval:fsatrace}

To compare \Make and \Rattle on \Fsatrace we took the commands we extracted for \S\ref{sec:eval:overhead} and ran the build script for the 100 previous commits in turn, starting with a clean build then performing incremental builds. To make the results readable, we hid any commands where all versions were < 0.02s, resulting in 26 interesting commits. We ran with 1 to 4 threads, but omit the 2 and 3 thread case as they typically fall either on or just above the 4 thread case.

\begin{tikzpicture}
\begin{axis}[
  title={Compile time at each successive commit},
  width=\textwidth,
  height=5cm,
  ylabel={Seconds},
  ymin=0,
  xmin=0,
  xmax=15,
]
\addplot [color=cyan, mark=o] table [x expr=\coordindex, y=make1] {data/fsatrace.dat};    \addlegendentry{\Make -j1}
\addplot [color=cyan, mark=*] table [x expr=\coordindex, y=make4] {data/fsatrace.dat};    \addlegendentry{\Make -j4}
\addplot [color=purple, mark=triangle] table [x expr=\coordindex, y=rattle1] {data/fsatrace.dat};  \addlegendentry{\Rattle -j1}
\addplot [color=purple, mark=triangle*] table [x expr=\coordindex, y=rattle4] {data/fsatrace.dat};  \addlegendentry{\Rattle -j4}
\end{axis}
\end{tikzpicture}

As we can see, the first build is always > 1s for \Rattle, but \Make is able to optimise it as low as 0.33s with 4 threads. Otherwise, \Rattle and \Make are competitive -- users would struggle to see the difference. The one commit that does show some variation is commit 2, where \Make at 1 thread matches all the \Rattle builds, but \Make at more than 1 thread goes slightly faster. The cause is a speculation leading to a write-write hazard. Concretely, the command for linking \texttt{fsatrace.so} changed to include a new file \texttt{proc.o}. \Rattle starts speculating on the old link, then gets the command for the new link -- they both write to \texttt{fsatrace.so}, leading to a hazard, and causing \Rattle to restart without speculation.

% WriteWriteHazard /tmp/extra-dir-39079319872788/fsatrace.so Cmd [EchoStderr False] ["cc","-shared","src/unix/fsatraceso.os","src/emit.os","src/unix/shm.os","-o","fsatrace.so","-ldl","-lrt"] Cmd [EchoStderr False] ["cc","-shared","src/unix/fsatraceso.os","src/emit.os","src/unix/shm.os","src/unix/proc.os","-o","fsatrace.so","-ldl","-lrt"] Restartable

\subsection{Redis}
\label{sec:eval:redis}

\begin{tikzpicture}
\begin{axis}[
  title={Compile time at each successive commit},
  width=\textwidth,
  height=5cm,
  ylabel={Seconds},
  ymin=0,
  xmin=0,
  xmax=30,
]
\addplot [color=cyan, mark=o] table [x expr=\coordindex, y=make1] {data/redis.dat}; 	\addlegendentry{\Make -j1}
\addplot [color=cyan, mark=*] table [x expr=\coordindex, y=make4] {data/redis.dat}; 	\addlegendentry{\Make -j4}
\addplot [color=purple, mark=triangle] table [x expr=\coordindex, y=rattle1] {data/redis.dat}; 	\addlegendentry{\Rattle -j1}
% \addplot table [x expr=\coordindex, y=rattle4] {data/redis.dat}; 	\addlegendentry{\Rattle -j4}
\addplot [color=purple, mark=triangle*] table [x expr=\coordindex, y=rattle4_noshared] {data/redis.dat}; 	\addlegendentry{\Rattle -j4}
\end{axis}
\end{tikzpicture}

In this graph the ``no shared'' variant is \Rattle with the cloud cache disabled. For local builds, the consequence is that if a file changes, then changes back, we will have to rebuild rather than get a cache hit -- something that never happens in this benchmark, making the copying of files to a shared cloud redundant work. The Redis project is structured as recursive \Make, which is known to be problematic \cite{miller:recursive_make} and how \Rattle is able to outperform \Make. Furthermore, the Redis build system is structured so as to write sentinel values to ignore some portions of the code unless they change, requiring users to manually clean when dependencies are upgraded. We avoid that entirely by running all the dependency commands every single time.

\begin{comment}
% I reimplemented Stack in Rattle. Not sure it's useful given how much other evaluation stuff we have.
\subsection{Reimplementing Stack}

\Rattle assumes that each command is atomic - it cannot be subdivided into smaller parts. If a command is secretly two independent commands then they should usually be expressed as such so they can be individually skipped.

Compound commands: Sometimes a command will produce something that is user specific (not great for caching), but the next step will remove the user specificity (good for caching). To fix that we allow compound commands, by conjoining two commands with \texttt{\&\&}. Sometimes the sole purpose of the second command can be to strip machine-unique data from the first command.

As another example, the GHC package database has additional entries added every time a package is installed, making the output a consequence of the original file\footnote{As a consequence many build systems, including \Bazel and \Rattle, use multiple package databases with only one entry per database}.

and some memoisation operations (\texttt{memo})
\end{comment}

\subsection{vim}
\label{sec:eval:vim}

\begin{tikzpicture}
\begin{axis}[
  title={Compile time at each successive commit},
  width=\textwidth,
  height=5cm,
  ylabel={Seconds},
  ymin=0,
  xmin=0,
  xmax=39,
]

\addplot [color=cyan, mark=o] table [x expr=\coordindex, y=make1] {data/vim.dat}; 	\addlegendentry{\Make -j1}
\addplot [color=cyan, mark=*] table [x expr=\coordindex, y=make4] {data/vim.dat}; 	\addlegendentry{\Make -j4}
\addplot [color=purple, mark=triangle] table [x expr=\coordindex, y=rattle1] {data/vim.dat}; 	\addlegendentry{\Rattle -j1}
\addplot [color=purple, mark=triangle*] table [x expr=\coordindex, y=rattle4] {data/vim.dat}; 	\addlegendentry{\Rattle -j4}
\end{axis}
\end{tikzpicture}

Vim is a popular text editor whose source code can be found on \href{https://github.com/vim/vim}{github}.  The majority of the source code is Vim script and C, and it is built with Make.  To build Make on a Unix based system one can merely call \Make from the top-level project directory.  To compare the original \Make based build to a new \Rattle build, we generated Vim over a series of commits checked-out from github with both the original build system and the adapted \Rattle build.  The \Rattle build script was created by recording every command executed by the original build, except for those invoking \Make.  Additionally, some commands which may have been separate shell commands in the original build were combined to create one command if they caused either \emph{read-write} or \emph{write-write} hazards.  A \Rattle build script was generated for each commit where the orginal build system changed or files were added to or removed from the project.  This was necessary since the \Rattle build was created from a literal list of hard-coded shell commands rather than the ideal Haskell program with variables.

% SS todo add list of the commits

Comparing building Vim \cite{} with \Make and with \Rattle for commits \emph{21109272f} to \emph{7cc96923c}.  For two commits \Make and \Rattle did noticeably different work; note, both of these commits were build incrementally, \emph{21109272f} was built from scratch by both build systems and all subsequent commits were built incrementally.

% todo compare sequential
% SS todo add timing data

\subsection{node}
\label{sec:eval:node}

% Brief into to the project
Node.js is a JavaScript runtime built on Chrome's V8 Javascript engine and can be found on \href{``https://github.com/nodejs/node''}{github}.  The project is largely written in JavaScript, C++, Python, and C, and is built using Make and a meta-build tool called \emph{Generate Your Projects (GYP)}.  To build Node.js from source, a user first runs \emph{./configure} which runs a python script that configures the build and runs GYP.  GYP generates the majority of the Makefiles used to build the project.

GYP takes a series of \emph{.gyp} files and produces the project build from those.  It puts all generated files in the project's \emph{out} directory and generates a separate \emph{*.mk} Makefile for each target, all of which are included by the top-level Makefile.  The generated top-level Makefile includes a \emph{do\_cmd} function which first checks if a command changed since the last time it was run, and if it has runs the command.  Additionally, it writes the command run to a generated dependency file, and if the command run was a \emph{g++} command that produced a file with dependencies, the dependencies in that file are cleaned up and written to the generated dependency file.

These dependency files are included in the top-level \emph{out/Makefile}, presumably so on future runs Make can consider the recorded dependencies of these object files as well as check whether the command has changed since it was previously run.  Most of these targets include the dependency \emph{FORCE\_DO\_CMD} which is an empty phony target and serves the purpose of forcing the build to check everytime whether the command has changed since it was previously run. Each time a command is re-run, these dependency files are re-generated.

% Explain the makefile insanity and all of this dependency stuff they do
% Still trying to make coherent since of what this gyp is doing, so here is my current understanding

The build both seems to be using a form of tracing to keep track of accurate dependencies as well as tracking whether or not the build itself has changed.  \Rattle would make it unnecessary for the build to explicitely do this, since it internally traces and tracks the depencencies of all commands run, and doesn't need to worry about when a build changes.

% The generator/make.py

% what is this
% # Helper that is non-empty when a prerequisite changes.
% # Normally make does this implicitly, but we force rules to always run
% # so we can check their command lines.
% #   $? -- new prerequisites
% #   $| -- order-only dependencies
% prereq_changed = $(filter-out FORCE_DO_CMD,$(filter-out $|,$?))

% # do_cmd: run a command via the above cmd_foo names, if necessary.
% # Should always run for a given target to handle command-line changes.

% Explain the intermediate files and how on a rebuild with no changes stuff still happens

When building certain targets, \emph{.intermediate} files are created and at the end of the build deleted.  These \emph{.intermediate} commands appear to be for doing code generation.  So, the generated files depend on the \emph{.intermediate} target which runs the code generator.  As with the object file targets mentioned, a \emph{.intermediate.d} dependency file is generated for the \emph{.intermediate} target.  This dependency file contains the \emph{.intermediate} code generation command run, so the build can check whether the command changed since it was last run.

These \emph{.intermediate} files are specified as prereqs of the special \Make target {.INTERMEDIATE}, which means that when one does not exist \Make won't bother updating it unless one of its prerequisites has changed.  Because these \emph{.intermediate} targets have \emph{DO\_/FORCE\_CMD} as a prereq they always run.  This would enable the build to check if the command changed since it was last run, but this only works if the Makefile includes the dependency file storing that information.  The dependency files of these \emph{.intermediate} targets, which run everytime, are not included in the Makefile.  Because of this the Makefile thinks the command has never run before, causing them to run everytime.  Even if the Makefile is modified to include the dependency files of these \emph{.intermediate} targets, the targets still execute the recipe which does the file generation even though the command has not changed. % SS comment on this further when I fully understand why the commands are registering as changed...

  % Other things like parallelism and rattle rebuilding less

It appears the authors of this build are attempting to keep track of accurate dependencies by recording them every time a command runs, they are also tracking the build by recording every command run and checking whether a command or its prerequisites have changed when deciding whether to run it.  The scheme to accomplish this is in my opinion quite complicated and confusing and relies on various hacks to work around \Make and the fact that this isn't a full-fledged programming language.  \Rattle accomplishes both of these things the build authors are trying to accomplish with \Make without forcing them to work so hard.

To compare the \Make based build to a \Rattle version, we built Node.js over a series of commits checked-out from github with both the original build system and the adapted \Rattle build.  The \Rattle build was created by recording every command executed by the original build, except for the commands creating dependency files and those invoking \Make.  A new such \Rattle build script was generated for any commit where the original build system changed or files were added to or removed from the project.
% data


% SS add the differences in commits
% 1. make list of commits run

% in order from latest to oldest.
% d80c40047b 0fe810168b 22724894c9 ab9e89439e cb210110e9 d10927b687
% 023ecbccc8 be6596352b [54c1a09202] 470511ae78 25c3f7c61a 13fe56bbbb
% [43fb6ffef7] a171314003 dd4c62eabe abe6a2e3d1 9225939528 d4c81be4a0
% 38aa31554c 1d9511127c d227d22bd9 5cf789e554 d65e6a5017 24e81d7c5a
% 2cd9892425 [64161f2a86] 0f8941962d 2170259940 32f63fcf0e 2462a2c5d7
% b851d7b986 70c32a6d19 3d456b1868 f2ec64fbcf 59a1981a22 [7b7e7bd185]
% 78743f8e39 a5d4a397d6
% [] indicate a commit where build appears to have changed. so need new rattle build too


% So, pretty much everything needs to be re-run. awesome.

% 5. Record here
% 6. Run on tank? with various threads; ask sam


% Comment from the Makefile
%# .buildstamp needs $(NODE_EXE) but cannot depend on it
%# directly because it calls make recursively.  The parent make cannot know
%# if the subprocess touched anything so it pessimistically assumes that
%# .buildstamp is out of date and need a rebuild.
%# Just goes to show that recursive make really is harmful...
%# TODO(bnoordhuis) Force rebuild after gyp update.


\subsection{tmux}

\begin{tikzpicture}
\begin{axis}[
  title=Compile time in seconds,
  width=\textwidth,
  height=7cm,
  xlabel={Time at each successive commit},
  ylabel={Seconds},
  ymin=0,
  xmin=0,
]

\addplot table [x expr=\coordindex, y=make1] {data/tmux.dat}; 	\addlegendentry{\Make -j1}
\addplot table [x expr=\coordindex, y=make4] {data/tmux.dat}; 	\addlegendentry{\Make -j4}
\addplot table [x expr=\coordindex, y=rattle1] {data/tmux.dat}; 	\addlegendentry{\Rattle -j1}
\addplot table [x expr=\coordindex, y=rattle4_noshared] {data/tmux.dat}; 	\addlegendentry{\Rattle -j4 (no shared)}
\end{axis}
\end{tikzpicture}


Tmux is a terminal multiplexer whose sourcecode is available on  \href{``https://github.com/vim/vim''}{github}.  It can be built from source using \emph{autoconf}, \emph{automake}, \emph{pkg-config}, and \emph{make}.  To build Tmux from source grabbed from github one does the following:

\begin{verbatim}
sh autogen.sh
./configure
make
\end{verbatim}

To compare a \Rattle based Tmux build to the \Make based one, we first ran \emph{sh autogen.sh} and \emph{./configure}, and then used the resulting Makefile(s) to generate the Rattle build scripts.  A new Rattle build script was generated for any commit where the build appeared to have changed or a file was added or deleted from the repository.

Tmux was built over a series of 40 commits with both \Make and \Rattle.  The commands run by each build system for each command were then compared.  For single threaded \Rattle and \Make, for 33 out of 40 commits, \Rattle and \Make ran the same commands, for 4 commits, \Make ran 1 more non-trivial command than \Rattle, but for 3 commits \Rattle ran substantially more cmds than \Make did.

These 3 commits were ones where new \Rattle build scripts were generated because something about the build or project had changed.  For two of these commits the version number of Tmux had changed which caused the textual contents of majority of the commands to change.  \Rattle viewed these as new commans and therefore ran them.  For the 3rd commit, \Rattle ran 125 commands more than \Make did because a command flag changed, which caused the textual contents of the majority of the commands to change, so \Rattle viewed them as new commands.

% todo add details about parallel version and hazards
% todo timing data



% 47174f51
% 150 cmds in file - 10 for make
% 140 cmds run by rattle not run by make
% how many did rattle run period?
% there were 141 cmds in the script.
% so what command did make run too?

%4822130b was a new build script

% 685eb381 also a new build script

% I wonder if I don't do sh autogen and configure if that wouldn't happen?



% For 1 thread

% commits where did same things: close enough; ignoring commands like rm tmux

% obf153da 0c6c8c4e 0eb7b547 19d5f4a0 22e9cf04 24cd726d 32be954b 37919a6b 3e701309
% 400750bb 43b36752 4694afb  470cba35 54553903 60ab7144 61b075a2 6c28d0dd 6f0241e6
% 74b42407 7cdf5ee9 7f3feb18 8457f54e 8b22da69 9900ccd0 a01c9ffc a4d8437b ba542e42
% bc36700d c391d50c c915cfc7 cdf13837 e9b12943 fdbc1116


% commits with differences
% 47174f51 (commit where make did automake stuff and rattle re-ran everything because a version number changed)
% 4822130b
% 685eb381

% 7eada28f: make did ./etc/ylwrap ....
% ed16f51e: make did ./etc/ylwrap ...
% ee3d3db3: make built tmux.o
% f3ea318a: make did ./etc/ylwrap ...


% commits
% ed16f51e 61b075a2 e9b12943 3e701309 8457f54e a01c9ffc cdf13837 74b42407 0eb7b547 f3ea318a 7cdf5ee9 ee3d3db3 685eb381 60ab7144 7eada28f 7f3feb18 8b22da69 bc36700d 32be954b 6f0241e6 19d5f4a0 43b36752 0bf153da 4822130b 47174f51 c915cfc7 54553903 400750bb 470cba35 a4d8437b 6c28d0dd 24cd726d 9900ccd0 c391d50c 0c6c8c4e fdbc1116 37919a6b 22e9cf04 ba542e42 4694afb

% \subsection{openssl}

\section{Related work}
\label{sec:related}

The vast majority of existing build systems are \emph{backward build systems} -- they start at the final target, and recursively determine the dependencies required for that target. In contrast, \Rattle is a \emph{forward build system}---the script executes sequentially in the order given by the user.

\subsection{Comparison to forward build systems}

The idea of treating a script as a build system, omitting commands that have not changed, was pioneered by \Memoize \cite{memoize} and popularised by \Fabricate \cite{fabricate}. In both cases the build script was written in Python, where cheap logic was specified in Python and commands were run in a traced environment. If a traced command hadn't changed inputs  since it was last run, then it was skipped. We focus on \Fabricate, which came later and offered more features. \Fabricate uses \texttt{strace} on Linux, and on Windows uses either file access times (which are either disabled or very low resolution on modern Windows installations) or a proprietary and unavailable \texttt{tracker} program. Parallelism can be annotated explicitly, but often is omitted.

These systems did not seem to have much adoption -- in both cases the original sources are no longer available, and knowledge of them survives only as GitHub copies. \Rattle differs from an engineering perspective by the tracing mechanisms available (see \S\ref{sec:tracing}) and the availability of cloud build (see \S\ref{sec:cloud_builds}) -- both of which are likely just a consequence of being developed a decade later. \Rattle extends these systems in a more fundamental way with the notion of hazards, which both allows the detection of bad build scripts, and allows for speculation -- overcoming the main disadvantage of earlier forward build systems. Stated alternatively, \Rattle takes the delightfully simple approach of these build systems, and tries a more sophisticated execution strategy.

Recently there have been three other implementations of forward build systems we are aware of.

\begin{enumerate}
\item \Shake \cite{shake} provides a forward mode implemented in terms of a backwards build system. The approach is similar to the \Fabricate design, offering skipping of repeated commands and explicit parallelism. In addition, \Shake allows caching custom functions as though they were commands, relying on the explicit dependency tracking functions such as \texttt{need} already built into \Shake. The forward mode has been adopted by a few projects, notably a library for generating static websites/blogs. The tracing features are provided by a combination of \Shake and \Fsatrace, and are the ones we reuse in \Rattle.
\item \Fac \cite{fac} is based on the \Bigbro tracing library. Commands are given in a custom file format as a static list of commands (i.e. no monadic expressive power as per \S\ref{sec:monadic}), but may optionally include a subset of their inputs or outputs. The commands are not given in any order, but the specified inputs/outputs are used to form a dependency order which \Fac uses. If the specified inputs/outputs are insufficient to give a working order, then \Fac will fail but record the \emph{actual} dependencies which will be used next time -- a build with no dependencies can usually be made to work by running \Fac multiple times.
\item \Stroll \cite{stroll} takes a static set of commands, without either a valid sequence or any input/output information, and keeps running commands until they have all succeeded. As a consequence, \Stroll may run the same command multiple times, using tracing to figure out what might have changed to turn a previous failure into a future success. \Stroll also reuses the tracing features of \Shake and \Fsatrace.
\end{enumerate}

There are significantly fewer forward build systems than backwards build systems, but the interesting dimension starting to emerge is how an ordering is specified. The three current alternatives are the user specifies a valid order (\Fabricate and \Rattle), the user specifies partial dependencies which are used to calculate an order (\Fac) or the user specifies no ordering and search is used (\Stroll and some aspects of \Fac).

\subsection{Comparison to backward build systems}
\label{sec:remote_execution}

The design space of backward build systems is discussed in \cite{build_systems_a_la_carte}. In that paper it is notable that forward build systems do not naturally fit into the design space, lacking the features that a build system requires. We feel that omission points to an interesting gap in the landscape of build systems. We think that it is likely forward build systems could be characterised similarly, but that we have yet to develop the necessary variety of forward build systems to do so. There are two dimensions used to classify backward build systems:

\textbf{Ordering} Looking at the features of \Rattle, the ordering is a sequential list, representing an excessively strict ordering given by the user. The use of speculation is an attempt to weaken that ordering into one that is less linear and more precise.

\textbf{Rebuild} For rebuilding, \Rattle looks a lot like the constructive trace model -- traces are made, stored in a cloud, and available for future use. The one wrinkle is that a trace may be later invalidated if it turns out a hazard occurred (see \S\ref{sec:choices}).
% In particular, the correspondence to constructive traces illuminates the consequences of moving to a deep constructive trace model (see \S\ref{sec:forward_hashes}) -- it solves non-determinism at the cost of losing unchanging builds.

\postparagraphs

\noindent There are three features present in some backward build systems that are particularly relevant to \Rattle:

\textbf{Sandboxing} Some backward build systems (e.g. \Bazel \cite{bazel}) run processes in a sandbox, where access to files which weren't declared as dependencies are blocked -- ensuring dependencies are always sufficient. A consequence is that it can be harder to write a \Bazel build script, requiring users to declare dependencies like \texttt{gcc} and system headers that are often overlooked. The sandbox doesn't prevent the reverse problem of too many dependencies.

\textbf{Remote Execution} Some build systems (e.g. \Bazel and \BuildXL \cite{buildxl}) allow running commands on a remote machine, usually with a much higher degree of parallelism than is available on the users machine. If \Rattle was able to leverage remote execution then speculative commands could be used to fill up the cloud cache, and \emph{not} cause local writes to disk, eliminating all speculative hazards -- a very attractive property. Remote execution in \Bazel sends all required files along with the command, but since \Rattle doesn't know the files accessed in advance, that model is infeasible. Remote execution in \BuildXL sends the files it thinks the command will need, augments the file system to block if other files are accessed, and sends requests for additional files back to the originating machine -- which would fit nicely with \Rattle.

\textbf{Hazards} Some build systems detect when certain types of hazard occur. For example, \Pluto \cite{erdweg2015sound} builds are constructed from builders and it is a requirement that no two builders generate the same file, and if they do the build is aborted. Similarly, if two commands in a \Rattle build write to the same file a write-write hazard occurs and the build is terminated. If a command in a \Rattle build writes to a file after another command has already read it a read-write hazard occurs. Analogously in \Pluto, if a builder requires a file generated by another builder, then the builder which generated the file must be required first and if it is not the build is aborted.

% This related work isn't relevant, and doesn't say anything interesting. It feels like a forced way of including Pluto.
%
% \textbf{Fine-grained dependencies} Pluto \cite{erdweg2015sound} is another backward build system that aims to
% provide fine-grained dependencies and optimal incremental rebuilding.  It supports dynamic
% dependencies, allows users to specify how they would like to depend on a file, and track build
% rule definitions so a build will re-run correctly when changed.  Rattle in contrast doesn't need
% to explicitly track whether a build script changes, and because all dependencies are implicitly
% tracked \Rattle does not provide the ability for a build script author to specify how a command
% should depend on any file.

\subsection{Analysis of existing build systems}

We aren't the first to observe that existing build systems often have incorrect dependencies.  \citet{bezemer2017empirical} performed an analysis of the missing dependencies in \Make build scripts, finding over \emph{1.2 million unspecified dependencies} among four projects. To detect missing dependencies, \citet{detecting_incorrect_build_rules} introduced a concept called \emph{build fuzzing}, finding race-conditions and errors in 30 projects. It has also been shown that build maintenance requires as much as a 27\% overhead on software development \cite{build_maintenance}, a substantial proportion of which is devoted to dependency management. Our anecdotes from \S\ref{sec:evaluation} all reinforce these messages.

% SS TODO maybe mention something about a project like node trying to generate dependencies within make;
\subsection{Speculation}

Speculation is used extensively in many other areas of computer science, from processors to distributed systems. If an action is taken before it is known to be required, that can increase performance, but undesired side-effects can occur. Most speculating systems attempt to block the side-effects from happening, or roll them back if they do.

The most common use of speculation in computer science is the CPU -- \citet{swanson_cpu_speculation} found that 93\% of useful CPU instructions were evaluated speculatively. CPUs use hazards to detect incorrect speculation, with similar types of read/write, write/write and write/read hazards \cite{patterson_cpu_design} -- our terminology is inspired by their approaches. For CPUs many solutions to hazards are available, e.g. stalling the pipeline if a hazard is detected in advance or the Tomasulo algorithm \cite{tomasulo}. \Rattle also stalls the pipeline (stops speculating) if it detects potential hazards, although does so with incomplete information, unlike a CPU. The Tomasulo algorithm involves writing results into temporary locations and the moving them over afterwards -- use of remote execution (\S\ref{sec:remote_execution}) might be a way to apply similar ideas to build systems.

Looking towards software systems, \citet{welc2005safe} showed how to add speculation to Java programs, by marking certain parts of the program as worth speculating with a future. Similar to our work, they wanted Java with speculation to respect the semantics of the sequential version of the program, which required two main techniques. First, all data accesses to shared state are tracked and recorded, and if a dependency violation occurs, the offending code is restarted.  Second, shared state is versioned using a copy-on-write invariant to ensure threads write to their own copies, preventing a future from seeing its continuation's writes.

Thread level speculation \cite{steffan1998potential} is used by compilers to automatically parallelise programs, often by executing multiple iterations of a loop body simultaneously. As before, the goal is to maintain the semantics of single-threaded execution. Techniques commonly involve buffering speculative writes \cite{steffan2000scalable} and ensuring that a read reflects the speculative writes of threads that logically precede it.

Speculation has also been investigated for distributed systems. \citet{nightingale2005speculative} showed that adding speculation to distributed file systems such as NFS can make some benchmarks over 10 times faster, by allowing multiple file system operations to occur concurrently. A model allowing more distributed speculation, even in the presence of message passing between speculated distributed processes, is presented by \citet{tapus2006distributed}. Both these pieces of work involve modifying the Linux kernel with a custom file system to implement roll backs transparently.

All these approaches rely on the ability to trap writes, either placing them in a buffer and applying them later or rolling them back. Unfortunately, such facilities, while desirable, are currently difficult to achieve in portable cross-platform abstractions (\S\ref{sec:tracing}). We have used the ideas underlying speculative execution, but if the necessary trapping/rollback facilities became available in future, it's possible we could follow the approaches more directly.

\section{Conclusion and future work}
\label{sec:conclusion}

In this paper we present a build system that can take a sequence of actions and treat them as a build script. From the user perspective, they get most of the benefits of a conventional build system (incrementality, parallelism) but with lower cost (less time thinking about dependencies, only needing to supply a valid ordering). Comparing \Rattle to other build systems, e.g. \Make, it is fair to say that \Rattle presents a simpler user interface (no dependencies), but a more complex implementation model.

Our evaluation in \S\ref{sec:evaluation} shows that for some popular real-world projects, switching to \Rattle would bring about simplicity and correctness benefits, with negligible performance cost. The two places where builds aren't roughly equivalent to \Make are the very first build (which could be solved with a global shared cache) and when speculation leads to a hazard. There are various approaches to improving speculation, including giving \Rattle a list of commands that should not be speculated (which can be a perfect list for non-monadic builds), or giving \Rattle a subset of commands' inputs/outputs (like \Fac does). It is also possible to have better recovery strategies from speculation errors.

Our evaluation focuses on projects whose build times are measured in seconds or minutes, not hours. It is as yet unclear whether similar benefits could be achieved on larger code bases, and whether the \Rattle approach of ``any valid ordering'' is easy to describe compositionally for large projects.  But, it does seem clear that existing projects built with \Make seek some form of automatic dependency detection, with most projects using some form of \texttt{gcc} dependency generation manually wired into the build system.

Our next steps are scaling \Rattle and incorporating feedback from actual users.

% SS probably say something else here but not sure at moment.

\paragraph{Acknowledgements} We'd like to thank Jorge Acereda and David Roundy for their work on \Fsatrace and \Bigbro respectively.


\bibliography{paper}

\end{document}
