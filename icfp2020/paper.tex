\documentclass[acmsmall]{acmart}

\begin{document}

\title{Build Systems with Perfect Dependencies}

\author{Sarah Spall}
\affiliation{
  \institution{Indiana University}
  \country{USA}
}
\email{sjspall@iu.edu}

\author{Neil Mitchell}
\affiliation{
  \institution{Facebook}
  \country{United Kingdom}
}
\email{ndmitchell@gmail.com}

\author{Sam Tobin-Hochstadt}
\affiliation{
  \institution{Indiana University}
  \country{USA}
}
\email{samth@cs.indiana.edu}

\begin{abstract}
Writing a build system for your project is tricky. You have to list the dependencies for every command, but these commands offer a level of encapsulation you may not understand. If you overspecify dependencies then it will rebuild too often, and without enough parallelism. If you underspecify dependencies, it will not rebuild enough. What if you could \emph{avoid} specifying dependencies and guarantee you got it right? This paper explores that approach, what it means, and how you recover what you lose.
\end{abstract}

\maketitle

\section{Introduction}
\label{sec:introduction}


Every non-trivial piece of software includes a ``build system'', describing how to set up the system from source code.
Build scripts \cite{build_systems_a_la_carte} describe \emph{commands to run} and \emph{dependencies to respect}. For example, using the \Make build system \cite{make}, a build script might look like:

\vspace{3mm}
\begin{verbatim}
main.o: main.c
    gcc -c main.c
util.o: util.c
    gcc -c util.c
main.exe: main.o util.o
    gcc -o main.exe main.o util.o
\end{verbatim}
\vspace{3mm}

This script contains three rules. Looking at the first rule, it says \texttt{main.o} depends on \texttt{main.c}, and is produced by running \texttt{gcc -c main.c}. What if we copied the commands into a shell script? We get:

\vspace{3mm}
\begin{verbatim}
gcc -c main.c
gcc -c util.c
gcc -o main.exe main.o util.o
\end{verbatim}
\vspace{3mm}

That's shorter, simpler and easier to follow. Instead of declaring the outputs and dependencies of each command, we've merely given one valid ordering of the commands (we could equally have put \texttt{gcc -c util.c} first). This simpler specification has additional benefits. First, we've fixed an inadvertent bug -- these commands depend on the undeclared dependency \texttt{gcc}, and potentially whatever header files are used by \texttt{main.c} and \texttt{util.c}. Furthermore, as the files \texttt{main.c} and \texttt{util.c} evolve, and their dependencies change (by changing the \texttt{\#include} directives), the shell script remains correct, while the \Make script \emph{must} be kept consistent or builds will become incorrect.

\subsection{Why dependencies?}

Given the manifest simplicity of the shell script, why write a \texttt{Makefile}? Build systems such as \Make have two primary advantages, both provided by dependency specification: \textbf{incrementality} and \textbf{parallelism}. \Make is able to re-run only the commands needed when only some of the files change, saving significant work when only \texttt{util.c} changes. \Make can also run multiple commands in parallel when neither depends on the other, such as the two invocations of \texttt{gcc -c}. With a shell script, these are both challenging research problems in incremental computing and automatic parallelization, respectively, and unlikely to be solvable for arbitrary programs such as \texttt{gcc}. 

\subsection{Builds without dependencies}

In this paper we show how to take the above shell script and gain the benefits of a \Make build (\S\ref{sec:design}). Firstly, we can skip those commands whose dependencies haven't changed by \emph{tracing} which files they read and write (\S\ref{sec:skipping_unnecessary}) and keeping a history of such traces. Secondly, we can run some commands in parallel, using \emph{speculation} to guess which future commands won't interfere with things already running (\S\ref{sec:speculation}). The key to speculation is a robust model of what ``interfering'' means -- we call a problematic interference a \emph{hazard}, which we define  in \S\ref{sec:hazards}; we formalize hazards and show important properties about the safety of speculation in \S\ref{sec:proof}.

We have implemented these techniques in a build system called \Rattle\footnote{\url{https://github.com/author_name_omitted/rattle}}, introduced in \S\ref{sec:implementation}, which embeds commands in a Haskell script. A key part of the implementation is the ability to trace commands, using techniques we describe in \S\ref{sec:tracing}. To evaluate our claims, and properly understand the subtleties of our design, we converted existing \Make scripts into \Rattle scripts, and discuss the performance characteristics and \Make script issues uncovered in \S\ref{sec:evaluation}. We also implement two small but non-trivial builds from scratch in \Rattle and report on the lessons learned. Our design can be considered a successor to the \Memoize build system \cite{memoize}, and we compare \Rattle with it and other related work in \S\ref{sec:related}. Finally, in \S\ref{sec:conclusion} we conclude and describe future work.


\section{Rattle}
\label{sec:rattle}

Our goal is to take a list of commands, and turn that into a build system. In this section we start by describing the most simple system possible, and progressively add features to obtain the benefits of a conventional build system.

\subsection{Executing commands}

In it's simplest variant, a build script can be a list of shell commands, like in \S\ref{sec:introduction}. To execute such commands correctly, it is sufficient to run each command sequentially in the order they were given. We consider this the reference semantics, and want any optimised/cached implementation to give the same results. We assume that each command is atomic - it cannot be subdivided into smaller parts.

\subsection{Monadic builds}
\label{sec:monadic}

A build system which can only execute a static list of commands is restricted in its expressive power. Taking the build system from \S\ref{sec:introduction} - it would be better to compile and link \emph{all} \texttt{*.c} files -- not just those explicitly listed in the script. A more plausible script might be:

\begin{verbatim}
FILES=$(ls *.c)
for FILE in $FILES; do
    gcc -c $FILE
done
gcc -o main.exe $FILES{/.c/.o/}
\end{verbatim}

\begin{figure}
\begin{verbatim}
import Development.Rattle

main = runRattle $ do
    -- TODO: Convert
    FILES=$(ls *.c)
    for FILE in $FILES; do
        gcc -c $FILE
    done
    gcc -o main.exe $FILES{/.c/.o/}
\end{verbatim}
\caption{A Haskell/\Rattle version of the script from \S\ref{sec:monadic}}
\label{fig:monadic}
\end{figure}

Now we have a curious mixture of build system commands (e.g. \texttt{ls}, \texttt{gcc}), some control logic (e.g. \texttt{for}) and simple manipulation (e.g. changing file extension). The way we cope with this is to consider the build system as a series of commands, where the future commands may depend on the results of previous commands. These are glued together with ``cheap'' functions like the control logic and simple manipulations. We take the approach that the cheap commands are fixed overhead, run on every build, and not cached or parallelised in any way. If any of these cheap manipulations becomes expensive, they can be replaced by a command, which will then be cached and parallelised.

To implement \Rattle we treat the script as a sequence of commands with no knowledge of which commands are coming next. As a consequence, even if you are using a simple static script, and then manually edit it, the \Rattle build remains correct -- it has no knowledge that you manually edited the script, or if it was instead conditional on something it didn't observe. This approach solves the problem listed in \cite{build_systems_a_la_carte} \S?? of determining when the build system changes.

\subsection{Dependency tracing}

For the rest of this section we assume the existence of \emph{depdendency tracing} which can, after a command completes, tell us which files that command read and which files it wrote -- we discuss the implementation and limitations of dependency tracing in \S\ref{sec:tracing}. Concretely, we can run a command in a special mode such that when the command completes (and not before) we can determine which files it read and which it wrote. We cannot determine at which point during the execution these files were accessed, or which order they were accessed in. We cannot prevent or otherwise redirect an in-progress access. These limitations are a (frustrating!) consequence of the tracing techology used.

\subsection{Skipping unnecessary commands}
\label{sec:skipping_unnecessary}

When \Rattle runs a command, it uses tracing to capture the files read and written, and then captures their hashes after the command finishes. If you subsequently run the same command, and the inputs and outputs haven't changed (same hashes), it can be skipped. This approach is the fundamental aspect of \Fabricate\citep{fabricate}. However, this approach has some issues that can make it incorrect.

\begin{itemize}
\item If the command is deterministic, then running it again will change nothing. However, many commands are not deterministic -- e.g. the output of \texttt{ghc} object files contains random values within it. We assume that where such non-determinism exists any possible output is equally valid.
\item If the command incorporates information such as the timestamp, then a cached value will represent the first time the command was run, not the current time. For commands like compilation that embed the timestamp in an object header, the original timestamp is probably fine. For commands that want to actually get the current time, we suggest they be made part of the control logic (as per \S\ref{sec:monadic}). The same applies to commands that require unique information, e.g. the generation of a GUID or random number.
\item If the command both reads and writes the same file, and the information written is fundamentally influenced by the file that was read, then the command never results in a stable state. As an example, \texttt{echo x >> foo.txt} will append the character \texttt{x} every time the command is run. As another example, the GHC package database has additional entries added every time a package is installed, making the output a consequence of the original file\footnote{As a consequence many build systems, including \Bazel and \Rattle, use multiple package databases with only one entry per database}. Equally, there are also commands that read the existing file to avoid rewriting a file that hasn't changed (e.g. \texttt{ghc} generating a \texttt{.hi} file) and commands that can cheaply update an existing file in some circumstances (the Micrsoft C++ linker). We make the assumption that if a command both reads and writes to a file that the read does not meaningfully influence the write. Thought of another way, we can consider the write non-deterministic anve the previous points apply.
\item If a command reads a file that gets modified by something not under the observation of \Rattle (e.g. a human or the untracked control logic), then the command may have seen multiple distinct values of the file, and the final hash will not match what the command saw. Therefore, we make the assumption that only \Rattle tracked commands are working on the relevant files at the time. We can detect such errors by checking the modification time of the read file pre-dates the start of the command execution.
\end{itemize}

In general we assume individual commands are well behaved, and if not, they can be lifted into the control logic.

\subsection{Cloud builds}

When skipping an execution \Rattle checks that all files accessed have the same hashes as a previous execution. However, if the files read all match a previous execution, and the corresponding output files have been cached, those output files can be copied over \emph{without} rerunning the command. If that cache is in on a server, you end up with cloud build functionality. Assuming commands that can be cached, whether a command uses a cloud oracle or runs locally is not observable to the rest of the system, so we don't need to consider it for the rest of the system. While this approach works beautifully in theory, in practice it leads to at least three separate problems:

\paragraph{Relative build directories} Often the current directory, or users profile directory, will be accessed by commands. These change either if the user has two working directories, or if they use different machines. We solve this by having a substitution table, replacing values such as the users home directory with \texttt{\$HOME}.

\paragraph{Compound commands} Sometimes a command will produce something that is user specific (not great for caching), but the next step will remove the user specificity (good for caching). To fix that we allow compound commands, by conjoining two commands with \texttt{\&\&}. Sometimes the sole purpose of the second command can be to strip machine-unique data from the first command.

\paragraph{Non-deterministic builds} If a command has non-deterministic output then if a user is temporarily disconnected from the cloud storage, and try and build something, then it will result in a different hash. If the user then reconnects to the cloud all actions depending on that non-deterministic build will fail to get a cache hit. There is one solution in \Rattle, where the hash of an output can be considered to be the hash of its inputs, but this setting is enabled on a per-file basis, and has all the problems of deep-constructive traces from \cite{build_systems_a_la_carte} \S?? and cannot benefit from unchanging outputs.

\subsection{Build consistency}
\label{sec:hazards}

For a \Make build system to be stable, it must be the case that after a build, a rebuild will not execute any further commands. It's easy to construct examples of sequences of commands that violate this property, for example:

\begin{verbatim}
echo x >> foo.txt
\end{verbatim}

As discussed in \S\ref{sec:skipping_unnecessary}, this command does not meet our restrictions. But we can find other sequences of command, where each command is fine in isolation, but the conjunction is problematic, for example:

\begin{verbatim}
echo 1 > foo.txt
echo 2 > foo.txt
\begin{verbatim}

This program writes \texttt{1} to \texttt{foo.txt}, then writes \texttt{2}. If the commands are reexecuted then the first command is invalid by virtue of its output changing, and after the first command has run, now the second command is invalid. More generally, if a build writes different values to the same file multiple times, it will not have a consistent build. But even without writing twice, it is possible to have an inconsistent build:

\begin{verbatim}
sha1sum foo.txt > bar.txt
sha1sum bar.txt > foo.txt
\end{verbatim}

Here \texttt{sha1sum} takes the SHA1 hash of a file, firstly taking the SHA of \texttt{foo.txt} and storing it in \texttt{bar.txt}, then using the same files the other way around. The root of the problem is that the script first reads from \texttt{foo.txt} on the first line, then writes to \texttt{foo.txt} on the second line, meaning that when the script is rerun the read will have to be repeated as the value has changed.

It turns out these are the only circumstances in which a build where every individual command is well-forced (as per \S\ref{sec:skipping_unnecessary}). We define such a build as \emph{hazardous} if it violates one of these consistency rules:

\begin{description}
\item[read-write hazard] When one command reads from a file, and a subsequent command writes to that file. On a future build, the first command will have to be rerun.
\item[write-write hazard] When two commands both write to the same file. On a future build, the first will be rerun (it's output has changed), which is likely to then cause the second to be rerun.
\end{description}

Using the tracing we spot hazards and raise errors if they occur. We prove that a build system with deterministic control logic with no hazards always results in no rebuilds in \S\ref{sec:proof:no_rebuild}. In a build system without hazards there is at most one write to any file, which occurs before any reads of that file. We can therefore prove there are no rebuilds by showing the first command can't rebuild, and proceeding by induction. It is not the case that the presence of hazards guarantees subsequent executions will definitely rebuild, for example if the a write doesn't actually change the value, but such a build system is dangerous -- if the write definitely can't change the output, the write should not be present.

\subsection{Parallelism}

Given a script with no hazards when executed sequentially, we can show that any interleaving of those commands that also has no hazards will result in an equivalent output in \S\ref{sec:proof:reorder}. Moreover, any parallel or interleaved execution without hazards will remain also be equivalent, see \S\ref{sec:proof:parallel}. And even further, if we execute any additional commands that don't cause hazards, they can be shown to have no impact on the rest of the build, see \S\ref{sec:proof:additional}.

As a consequence of the above, we have quite a lot of freedom to optimise the commands, provided they do not cause hazards. There are two ways we allow parallelism.

\subsubsection{Explicit Parallelism}

In the Haskell API for \Rattle there is a parallel combinator \texttt{forP}. Replacing \texttt{forM} with \texttt{forP} in Figure \ref{fig:monadic} causes the commands to be given to \Rattle in parallel. As a consequence, they can be executed in parallel. The use of explicit parallelism is convenient when replacing a loop, but harder when expressing that three executables can be built in parallel, but that a single shared object common to all of them must complete before starting, so it is difficult to provide \emph{complete} parallelism annotations.

Interestingly, given complete \emph{dependency} information (e.g. as available to \Make) it is possible to infer complete \emph{parallelism} information. However, the difficult of writing such information is the attraction to a tracing approach in the first place.

\subsubsection{Implicit Parallelism}
\label{sec:speculation}

The other source of parallelism is implicit, using speculation. If we can predict what commands are coming up next, and predict that their execution will not cause hazards, then we can speculatively execute them. Such predictions can be made by simply recording the last known execution and using that. If speculation suggests that running a command would be beneficial there are two possible approaches to take.

Firstly, we can execute the command remotely, or in a sandboxed manner - ensuring all writes do \emph{not} end up on the file system, but are recorded to the cloud cache. In such a mode we are filling up the coud cache speculative, with the hope that when the future command does arrive it can be satisfied from the cache. However, running remotely requires syncronising the files across, or using a sycronise on demand approach (CITE Microsoft Remote Execution talk). Running with a sandbox which intercepts writes is not an easily available cross-platform feature, so we have not explored it further.

Alternatively, we can execute the command locally, which is what \Rattle does. However, if the execution leads to a hazard, it is possible that the hazard is entirely an artefact of speculation. One simplistic approach is to simply rerun without speculation if speculation leads to a hazard. A more refined approach is to determine whether classification may have impacted the hazard, which can be categorised as per \S\ref{sec:proof:classify_hazard}, and either raise the error immediately (if the speculation was not at fault) or selectively eliminte a subset of commands from speculation (if speculation was at fault).


\section{Evaluation}

\section{Related Work}

\section{Future Work}

\section{Conclusion}

\end{document}
