\documentclass[acmsmall]{acmart}

\begin{document}

\title{Build Systems with Perfect Dependencies}

\author{Sarah Spall}
\affiliation{
  \institution{Indiana University}
  \country{USA}
}
\email{sjspall@iu.edu}

\author{Neil Mitchell}
\affiliation{
  \institution{Facebook}
  \country{United Kingdom}
}
\email{ndmitchell@gmail.com}

\author{Sam Tobin-Hochstadt}
\affiliation{
  \institution{Indiana University}
  \country{USA}
}
\email{samth@cs.indiana.edu}

\begin{abstract}
Writing a build system for your project is tricky. You have to list the dependencies for every command, but these commands offer a level of encapsulation you may not understand. If you overspecify dependencies then it will rebuild too often, and without enough parallelism. If you underspecify dependencies, it will not rebuild enough. What if you could \emph{avoid} specifying dependencies and guarantee you got it right? This paper explores that approach, what it means, and how you recover what you lose.
\end{abstract}

\maketitle

\section{Introduction}
\label{sec:introduction}

Build scripts \cite{build_systems_a_la_carte} describe \emph{commands to run} and \emph{dependencies to respect}. For example, using the \Make build system \cite{make}, a build script might look like:

\vspace{3mm}
\begin{verbatim}
main.o: main.c
    gcc -c main.c
util.o: util.c
    gcc -c util.c
main.exe: main.o util.o
    gcc -o main.exe main.o util.o
\end{verbatim}
\vspace{3mm}

This script contains three rules. Looking at the first rule, it says \texttt{main.o} depends on \texttt{main.c}, and is produced by running \texttt{gcc -c main.c}. But what if we copied the commands into a shell script? We get:

\vspace{3mm}
\begin{verbatim}
gcc -c main.c
gcc -c util.c
gcc -o main.exe main.o util.o
\end{verbatim}
\vspace{3mm}

That's shorter, simpler and easier to follow. Instead of declaring the outputs and dependencies of each command, we've merely given one valid ordering of the commands (we could equally have put \texttt{gcc -c util.c} first). There are two main downsides: 1) everything will always rebuild even if it's dependencies haven't changed; 2) the commands will run sequentially, while \Make can run the two \texttt{gcc -c} commands in parallel. But there are some benefits. We've fixed an inadvertent bug -- these commands depend on the undeclared dependency \texttt{gcc}, and potentially whatever header files are used by \texttt{main.c} and \texttt{util.c}. Furthermore, as the files \texttt{main.c} and \texttt{util.c} evolve, and their dependencies change (by changing the \texttt{\#include} directives), the shell script remains correct, while the \Make script \emph{must} be kept consistent (or builds will become stagnant).

\subsection{Overview}

% SS I changed script -> build; not sure if this is correct or wrong?
In this paper we show how to take the above shell script and gain the benefits of a \Make build (\S\ref{sec:design}). Firstly, we can skip those commands whose dependencies haven't changed by \emph{tracing} which files they read and write (\S\ref{sec:skipping_unnecessary}). Secondly, we can run some commands in parallel, using \emph{speculation} to guess which future commands won't interfere with things already running (\S\ref{sec:speculation}). The key to speculation is a robust model of what ``interfering'' means -- we call a problematic interference a \emph{hazard}, which we define in \S\ref{sec:hazards} and use to prove important properties about the safety of speculation in \S\ref{sec:proof}.

We have implemented these techniques in a build system called \Rattle\footnote{\url{https://github.com/author_name_omitted/rattle}}, introduced in \S\ref{sec:implementation}, which embeds commands in a Haskell script. A key part of the implementation is the ability to trace commands, using techniques we describe in \S\ref{sec:tracing}. To evaluate our claims, and properly understand the subtlties of our design, we converted existing \Make scripts into \Rattle scripts, and discuss the performance characteristics and \Make script bugs uncovered in \S\ref{sec:evaluation}. Our design can be considered a successor to the \Memoize build system \cite{memoize}, and we compare the differences between that and other related work in \S\ref{sec:related}. Finally, in \S\ref{sec:conclusion} we conclude and describe future work.


\section{Rattle}
\label{sec:rattle}

We want to take a list of commands, and turn it into a build system. To do that, we assume the existence of a \emph{tracing oracle} which can, after a command completes, tell us which files that command read and which files it wrote -- we discuss the implementation and limitations of the tracing oracle in \S\ref{sec:tracing}. In this section we start by describing the most simple system possible, and progressively add features to obtain the benefits of a full build system.

\subsection{Executing commands}

In it's simplest variant, a build script can be a list of shell commands, like in \S\ref{sec:introduction}. To execute such commands correctly, it is sufficient to run each command sequentially in the order they were given. We consider this the reference semantics, and want any optimised/cached implementation to give the same results. For now, we assume commands are deterministic, but discuss how to weaken that assumption in \S\ref{sec:determinism}. We also assume that each command is atomic - it cannot be subdivided into smaller parts.

\subsection{Monadic builds}
\label{sec:monadic}

A build system which can only execute a static list of commands is restricted in its expressive power. Taking the build system from \S\ref{sec:introduction} - it would be better to compile and link \emph{all} \texttt{*.c} files -- not just those listed in the script. A more plausible script might be:

\begin{verbatim}
FILES=$(ls *.c)
for FILE in $FILES; do
    gcc -c $FILE
done
gcc -o main.exe $FILES{/.c/.o/}
\end{verbatim}

\begin{figure}
\begin{verbatim}
import Development.Rattle

main = runRattle $ do
    -- TODO: Convert
    FILES=$(ls *.c)
    for FILE in $FILES; do
        gcc -c $FILE
    done
    gcc -o main.exe $FILES{/.c/.o/}
\end{verbatim}
\caption{A Haskell/\Rattle version of the script from \S\ref{sec:monadic}}
\label{fig:monadic}
\end{figure}

Now we have a curious mixture of build system commands (e.g. \texttt{ls}, \texttt{gcc}), some control logic (e.g. \texttt{for}) and simple manipulation (e.g. changing file extension). The way we cope with this is to consider the build system as a series of commands, where the future commands may depend on the results of previous commands. These are glued together with ``cheap'' functions like the control logic and simple manipulations. We take the approach that the cheap commands are fixed overhead, run on every build, and not cached or parallelised in any way. If any of these cheap manipulations becomes expensive, they can be replaced by a command, which will then be cached and parallelised.

To implement \Rattle we treat the script as a sequence of commands with no observation on which commands are coming next. As a consequence, even if you are using a simple static script, and then manually edit it, the \Rattle build remains correct -- it has no knowledge that you manually edited the script, or if it was instead conditional on something it didn't observe.

\subsection{Skipping unnecessary commands}

When \Rattle runs a command, it records the files it reads and writes, and their hashes at that point. If you subsequently run the same command, and the inputs and outputs haven't changed (same hashes), it can be skipped. This is mostly true -- you get old but correct values. If you are running \texttt{date} it won't work, and if you are running \texttt{gcc} it will have an old compilation time. But it's good enough.

Given a command, if you know what files it reads and writes, and none of those files have changed, you can skip running it providing that if it were deterministic it wouldn't be harmful. As an example, if the compiler produces a different object file each time, but it would be OK with not doing that, then you're fine. In practice this means that any command that isn't relied on to produce fresh entropy (the time, a GUID, a random number) is fine to skip subsequent times. Those commands that do produce fresh entropy are not well suited to a build system anyway, so aren't typically used in build systems.

This got solved by fabricate. You can use a system access tracer to watch what a process reads/writes, and have a database storing the command, previous reads/writes, and then skip it if the values haven't changed. This step is not complex.

\subsection{Cloud builds}

If \Rattle matches on the files read, but not on the files written, but a previous run had the same files read/written and cached the written files, you can copy them over. If that store is in a cloud, you end up with cloud build functionality. It's pretty simple. Moreover, whether a command uses a cloud oracle or runs locally, that's not observable to the rest of the system, so we ignore it for proofs.

Given knowledge of the reads/writes of a command, it's not too hard to store them in a cloud, and satisfy commands not by rerunning them, but by copying their result from a shared cache. While this works beautifully in theory, in practice it leads to at least three separate problems which we solve.

Relative build dirs

Often the current directory, or users profile directory, will be accessed by commands. These change either if the user has two working directories, or if they use different machines. We solve this by having a substitution table.

Compound commands

Sometimes a command will produce something that is user specific, but the next step will make it not user specific. To fix that we add compound commands. You can do it with a shell script, using `pipeline`, or with TemplateHaskell blocks.

Non-deterministic builds

We solve this by having `.rattle.hash` files which we substitute for reading.

\subsection{Build consistency}

However, it a build writes a file \emph{twice} in a run, or reads it then writes to it, the result will be that a subsequent rebuild will have stuff to do, as it won't end at a quiescent state. To detect that, we introduce the concept of hazards.

For a \Make build system to be stable, it must be the case that after a build, a rebuild will not execute any other commands. It's easy to construct examples of sequences of commands that violate this property, for example:

\begin{verbatim}
gcc -c foo.c
echo x >> foo.c
\end{verbatim}

This script compiles \texttt{foo.c}, then appends the character \texttt{x} to \texttt{foo.c}. Each time around, the input to \texttt{gcc} will have changed, necessitating a recompile. We define such a build as \emph{hazardous}, because it violates one of our consistency rules:

\begin{description}
\item[read-write hazard] When one command reads from a file, and a subsequent command writes to that file. On a future build, the first command will have to be rerun.
\item[write-write hazard] When two commands both write to the same file. On a future build, the first will be rerun (it's output has changed), which is likely to then cause the second to be rerun.
\end{description}

We assume that if a single command both reads and writes to the same file (as \texttt{echo x >> foo.c} does), then within that command they were correctly sequenced.

Using the tracing we spot hazards and raise errors if they occur. We proove that a build system with deterministic control logic with no hazards always results in no rebuilds in \S\ref{sec:proof:no_rebuild}. In a build system without hazards there is at most one write to any file, which occurs before any reads of that file. We can therefore prove there are no rebuilds by showing the first command can't rebuild, and proceeding by induction.

\subsection{Parallelism}

Given a script with no hazards when executed sequentially, we can show that any interleaving of those commands that also has no hazards will result in an equivalent output in \S\ref{sec:proof:reorder}. Moreover, any parallel execution without hazards will remain consistent, see \S\ref{sec:proof:parallel}. And even further, if we execute any additional commands that don't cause hazards, they can be shown to have no impact on the rest of the build, see \S\ref{sec:proof:additional}.

As a consequence of the above, it is important that we check for hazards, but are otherwise free to run things in parallel. There are two ways we allow parallelism.

\subsubsection{Explicit Parallelism}

In the Haskell API for \Rattle there is a parallel combinator \texttt{forP}. Replacing \texttt{forM} with \texttt{forP} in Figure \ref{fig:monadic} causes the commands to be given to \Rattle in parallel. As a consequence, they can be executed in parallel. The use of explicit parallelism is convenient when replacing a loop, but harder when expressing that three executables can be built in parallel, but that a single shared object common to all of them must complete before starting, so it is difficult to provide \emph{complete} parallelism annotations.

Interestingly, given complete \emph{dependency} information (e.g. as available to \Make) it is possible to infer complete \emph{parallelism} information. However, the difficult of writing such information is the attraction to a tracing approach in the first place.

\subsubsection{Implicit Parallelism}

The other source of parallelism is implicit, using speculation. If we can predict what commands are coming up next, and predict that their execution will not cause hazards, then we could speculatively execute them. Such predictions can be made by simply recording the last known execution and using that. If speculation suggests that running a command would be beneficial there are two possible approaches to take.

Firstly, we can execute the command remotely, or in a sandboxed manner - ensuring all writes do \emph{not} end up on the file system, but are recorded to the cloud cache. In such a mode we are filling up the coud cache speculative, with the hope that when the future command does arrive it can be satisfied from the cache. However, running remotely requires syncronising the files across, or using a sycronise on demand approach (CITE Microsoft Remote Execution talk). Running with a sandbox which intercepts writes is not an easily available cross-platform feature.

Alternatively, we can execute the command locally, which is what \Rattle does. However, if the execution leads to a hazard, it is possible that the hazard is entirely an artefact of speculation. One simplistic approach is to simply rerun without speculation if speculation leads to a hazard. A more refined approach is to determine whether classification may have impacted the hazard, which can be categorised as per \S\ref{sec:proof:classify_hazard}, and either raise the error immediately (if the speculation was not at fault) or selectively eliminte a subset of commands from speculation (if speculation was at fault).


\section{Evaluation}

\section{Related Work}

\section{Future Work}

\section{Conclusion}

\end{document}
