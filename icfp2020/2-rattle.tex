\section{Rattle}

% Rattle is a build system a la fabricate, where instead of describing what you want to build
% you just describe how to build it.

% The author of a Rattle build script just what and how to run the tools that build their
% project, in an order they know is correct.

% Rattle then learns while running the script what the true file dependencies are for the build.
% So, the build script author cannot get them wrong.

% How does Rattle work?

% It uses system level tracing to record the files read and written by the commands specified in
% the haskell build script.

% Rattle records this tracing information for use during subsequent runs.  On subsequent runs
% The tracing information informs Rattle of which cmds should safely run in parallel.

\subsection{Speculation}

% Maybe this should be earlier.

% Because Rattle doesn't know what cmds are in the build script being run, it cannot perform
% the same sort of graph traversal Make can to determine which commands to run in order.

% Speculation is used instead to achieve parallelism.  If Rattle believes it is safe, it will run
% the commands from the previous build in parallel with the current build script.  When a cmd it
% ran speculatively is encountered in the build script it won't have to run the command.

% If the build script hasn't changed between runs and the file level dependencies of each cmd
% haven't changed, then Rattle should have no problems running cmds in parallel.

% But what if the build script or file level dependencies have changed?  Then Rattle might
% run cmds in an order that doesn't respect their new dependencies.  To deal with this Rattle
% has a concept of hazards, because speculation can go wrong.

\subsection{Hazards}

% already need to have explained the properties

% Read Write hazard
% Write Write hazard

% ReadWrite hazard occurs when two commands read and write to the same file at the same time
% or a command reads from a file before it is written to

% WriteWrite hazard occurs when two commands write to the same file during the build.

% When a hazard occurs Rattle will terminate, restart, or continue.  If a property violating hazard
% occurred Rattle will terminate and alert the build user.  If Rattle speculated a command that
% caused a property violation the build will be restarted with no speculation.
% If Rattle speculated a read cmd before a write then Rattle can continue and later re-execute the
% read command again.


\subsection{Rebuilds}

% using hashes to determine if files have changed

% only rerun cmds whose dependencies have changed

\subsection{Shared build cache}

% Store output files to copy if they are missing but cmd's dependencies haven't changed

% Other things Neil mentioned in the Rattle readme... 



