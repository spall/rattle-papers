\section{Related work}
\label{sec:related}

Of the build systems in existence, the vast majority can be thought of as backward build systems - they start at the final target, and recursively determine the dependencies required for that target. In contrast, \Rattle can be thought of as a forward build system -- the script executes sequentially in the order given by the user.

\subsection{Comparison to forward build systems}

The idea of treating a script as a build system, omitting execution of commands that have not changed, was first pioneered by \Memoize \cite{memoize} then later popularised by \Fabricate \cite{fabricate}. In both cases the build script was written in Python, where cheap logic was specified in Python and commands were run in a traced enviroment. If a traced command hadn't changed inputs or outputs since it was last run, then it was skipped. Of these two incarnations, \Fabricate came later and offered more features, so we focus on it. \Fabricate uses \texttt{strace} on Linux, and on Windows uses either file access times (which are either disabled or very low resolution on modern Windows installations) or a proprietory and non-distributable \texttt{tracker} program. Parallelism can be annotated explicitly, but is considered an advanced feature.

These systems did not seem to have much adoption -- in both cases the original sources are no longer available, and knowledge of them survives on GitHub copies. \Rattle differs from an engineering perspective by the tracing mechanisms available (see \S\ref{sec:tracing}) and the use of cloud build (see \S\ref{sec:cloud_build}) -- both of which are likely just a consequence of being developed a decade later. \Rattle extends these systems in a more fundamental way with the notion of hazards, which both allow the detection of bad build scripts, and allow for speculation -- overcoming the main disadvantage of earlier foward build systems. Stated alternatively, \Rattle takes the delightfully simple approach of these build systems, and tries a more sophisticated execution strategy.

\subsection{Comparison to backward build systems}

The design space of backward build systems is discussed in \cite{build_systems_a_la_carte}. In that paper it is notable that \Fabricate does not naturally fit into the design space, lacking the features that a build system requires. We feel that points to an interesting gap in the landscape of build systems. We think that it is likely forward build systems could be characterised similarly, but that we have yet to develop the necessary variety of forward build systems to do so. There are two dimensions in the paper:

\paragraph{Ordering} Looking at the features of \Rattle, we would say the ordering is incomplete and given by a user. The use of speculation is an attempt to refine that ordering into one that is more precise.

\paragraph{Rebuild} In this sense \Rattle looks a lot like the constructive trace model -- traces are made, stored in a cloud, and accessible for future use. The one wrinkle is that a trace may be later invalidated, if it turns out a hazard occurred. In particular, the correspondence to constructive traces illuminates the consequences of moving to a deep constructive trace model (see \S\ref{sec:cloud_builds}) -- it solves non-determinism at the cost of losing unchanging builds.

Some existing backward build systems have notions for sandboxing to ensure all commands are present. e.g. \Bazel. That often results in a lot of work, and doesn't prevent the reverse problem of too many dependencies.

Remote execution is a useful thing that we can't get.

\subsection{Analysis of existing build systems}

We aren't the first to point out that existing build systems are wrong. See \url{https://ieeexplore.ieee.org/document/8812082}.

\subsection{Other related work}

The notion of hazards is used extensively in processor design. This is standard terminology.
